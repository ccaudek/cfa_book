\chapter{Oggettività specifica e sufficienza }
\label{chapter:ogg_specifica} 

%% un esempio di analisi è fornito nella pagina
%% http://wiki.r-project.org/rwiki/doku.php?id=packages:cran:ltm
%% può essere usato come esempio conclusivo dei modelli irt

%%--------------------------------------------------------------
%\section{La misurazione}
%
%Le scienze naturali si pongono l'obiettivo di operare valutazioni dei 
%fenomeni empirici che siano \emph{oggettive} (ovvero, non dipendenti
%dalle caratteristiche dell'osservatore), \emph{inequivocabili} (cioè
%tali per cui la valutazione di un osservatore sia comprensibile a
%tutti gli utilizzatori dell'osservazione) e \emph{riproducibili} (nel
%senso che in ogni tempo e luogo l'osservazione dello stesso fenomeno
%deve produrre lo stesso risultato). Per ottenere tale scopo, nelle
%scienze naturali i fenomeni empirici sono descritti in maniera
%quantitativa.  L'operazione di attribuire dei numeri ai fenomeni
%empirici viene detta operazione di misurazione. Cenni alla teoria
%della misura sono forniti nell'Appendice A.


%--------------------------------------------------------------
\section{I livelli di scala}

Secondo Michell (1999), l'obiettivo della scienze naturali è quello di indagare le relazioni tra i cambiamenti d'intensità degli attributi dei fenomeni, laddove i cambiamenti d'intensità sono additivi e possono essere espressi in termini di rapporti. Per Michell, ogni misurazione richiede il confronto con un riferimento noto: misurare una grandezza significa trovare il {\it rapporto} fra l'intensità considerata ed un termine di confronto (cioè, un'unità di misura).
Per misurare la lunghezza, per esempio, si può usare un righello graduato: il righello (che rappresenta la grandezza di riferimento) viene accostato all'oggetto da misurare; la misura si ricava dal confronto tra la lunghezza dell'oggetto e la scala graduata del righello.

Nel caso delle proprietà  del mondo fisico è ``facile'' determinare l'unità di misura  per i fenomeni di interesse (ad esempio, la temperatura o la lunghezza). Più difficile è stabilire un'unità misura per proprietà psicologiche quali l'abilità, l'autostima, l'intelligenza, ecc. Non essendo possibile stabilire un'unità di misura appropriata, le misurazioni degli attributi psicologici non possono essere espresse in termini di rapporti. Di conseguenza viene a mancare, secondo Michell, il requisito fondamentale proprio di una scienza quantitativa.

È possibile fornire una risposta convincente alla critica che Michell rivolge alla psicologia? In realtà, è ovvio che, nella maggior parte dei casi, non sia possibile costruire scale psicologiche al livello di una scala a rapporti equivalenti. Tuttavia, la maggior parte degli psicologi ritiene che, nonostante questo limite, la psicologia possa  essere considerata  una scienza empirica. Scoperte importanti sui fenomeni empirici, compresi quelli psicologici, possono essere compiute anche se le proprietà di tali fenomeni vengono misurate a livelli più bassi di scala, quali quelli delle scale nominali o ad intervalli equivalenti. Le misurazioni psicologiche a tali livelli di scala sono interpretabili, purché i dati vengano  analizzati nella maniera appropriata (ovvero, utilizzando le statistiche idonee). 

A tale ovvia considerazione si deve aggiungere una precisazione. La procedura di analisi dei dati più usata in psicologia è il modello lineare generale. Affinché tale procedura di analisi dei dati possa essere utilizzata è necessario che le misure siano almeno a livello di scala ad intervalli equivalenti. Gli psicologi devono dunque giustificare il fatto che le loro misure si collocano almeno a tale  livello di  scala, ovvero, devono essere in grado di costruire ``a livello empirico una variabile sufficientemente sensibile e valida per la quale le differenze tra le misure che la rappresentano devono avere il medesimo significato, attraverso tutta la gamma di valori della variabile'' (Cristante e Mannarini, 2004, p. 5). Il modello di Rasch rappresenta uno dei tentativi più riusciti di fornire una tale giustificazione. 

%--------------------------------------------------------------
\section{Oggettività specifica}

Rasch (1960) cita un attacco che Skinner ha rivolto all'applicazione delle tradizionali tecniche statistiche alla ricerca psicologica:

\begin{quote}
The order to be found in human and animal behavior should be extracted from investigations into individuals \dots psychometric methods are inadequate for such purposes since they deal with groups of individuals (Skinner, 1956, p. 221).
\end{quote}
La critica di Skinner si può estendere alle correnti applicazioni statistiche alla psicologia (analisi di regressione, analisi della varianza, ecc.), le quali si pongono l'obiettivo di stimare il ``comportamento medio'' degli individui piuttosto che le caratteristiche particolari di ciascun individuo.  L'alternativa proposta da Rasch allo studio del ``comportamento medio'' è quella di sviluppare una metodologia che possa essere applicata direttamente ai singoli individui.
\begin{quote}
Individual-centered statistical techniques require models in which each individual is characterized separately and from which, given adequate data, the individual parameters can be estimated. It is further essential that comparisons between individuals become independent of which particular instrument -- test, or items or other stimuli -- within the class considered have been used.  Symmetrically, it ought to be possible to compare stimuli belonging to the same class -- measuring the same thing -- independent of which particular individuals within the class considered were instrumental for the comparison (Rasch, 1960, p. vii).
\end{quote}
%L'invarianza delle misure rispetto allo strumento di misura viene chiamata da Rasch \emph{oggettività specifica}\footnote{Tale requisito era già stato discusso da Thurstone: ``{\it A measuring instrument must not be seriously affected in its measuring functions by the object of measurement. To the extent that its measurement function is so affected, the validity of the instrument is impaired or limited.  If a yardstick measured differently because of the fact that it was a rug, a picture, or a piece of paper that was being measured, then to that extent the trustworthiness of that yardstick as a measuring device would be impaired. Within the range of objects for which the measuring instrument is intended its functions must be independent of the object of measurement}'', Thurstone, 1928, p. 547}.
%
%Secondo Rasch (1967) il postulato dell'oggettività specifica è un principio generale della ricerca scientifica: all'interno di condizioni chiaramente specificate, le affermazioni scientifiche devono essere generalizzabili al di là delle specifiche condizioni in cui si svolge l'osservazione. 
Questa citazione introduce il concetto di \emph{oggettività specifica}. 
Secondo Rasch (1960), ciò che caratterizza la superiorità delle scienze naturali rispetto a quelle umane è legato alla possibilità di sviluppare metodi per trasformare osservazioni in misure, in base a regole che soddisfano il principio della \emph{oggettività specifica}. In termini intuitivi tale principio si riferisce al fatto che i metodi di misurazione delle scienze naturali consentono di misurare caratteristiche specifiche di un soggetto senza che il processo di misurazione risulti influenzato da caratteristiche del soggetto diverse da quella di interesse, da altri soggetti e da peculiarità dello strumento di misurazione.
L'invarianza delle misure rispetto allo strumento di misura è, secondo Rasch (1967), un principio generale della ricerca scientifica: all'interno di condizioni chiaramente specificate, le affermazioni scientifiche devono essere generalizzabili al di là delle specifiche condizioni in cui si svolge l'osservazione. Tale requisito era già stato discusso da Thurstone: 
\begin{quote}
A measuring instrument must not be seriously affected in its measuring functions by the object of measurement. To the extent that its measurement function is so affected, the validity of the instrument is impaired or limited.  If a yardstick measured differently because of the fact that it was a rug, a picture, or a piece of paper that was being measured, then to that extent the trustworthiness of that yardstick as a measuring device would be impaired. Within the range of objects for which the measuring instrument is intended its functions must be independent of the object of measurement (Thurstone, 1928, p. 547). 
\end{quote}
Il modello di Rasch soddisfa la proprietà dell'oggettività specifica descritta sopra, in quanto le misure ottenute suo tramite sono indipendenti degli item utilizzati e dal campione di soggetti impiegato per selezionare gli item. Esso dunque rappresenta una forma di ``misurazione oggettiva''. Rasch riporta una serie di esempi in cui mostra come tutte le misure fondamentali della fisica soddisfino il principio dell'oggettività specifica. Il maggior contributo di Rasch consiste nell'avere indicato alle scienze umane la via per elevarsi al rango delle scienze naturali. Tale obiettivo può essere raggiunto solo costruendo teorie che, ove coinvolgano entità latenti per le quali non sia stato ancora individuato un adeguato strumento di misura, utilizzino solo misure che soddisfano il principio della oggettività specifica. Rasch riassunse le conseguenze della violazione di tale principio nel modo seguente:
\begin{quote}
Thus, if a set of empirical data cannot be described by that model then complete specifically objective statements cannot be derived from them.  Firstly, the failing of specific objectivity means that the conclusions about, say, any set of person parameters will depend on which other persons are also compared.  As a parody we might think of the comparison of the volumes of a glass and a bottle as being influenced by the heights of some of the books on a shelf. Secondly, the conclusion about the persons would depend on just which items were chosen for the comparison, a situation to which a parallel would be, that the relative height of two persons would depend on whether the measuring stick was calibrated in inches of in centimeters. Avoiding such irrelevant dependences is just my reason for recommending the use of models for measuring whenever they may be utilized  (Rasch, 1968, p. 27--28).
\end{quote}

%--------------------------------------------------------------
\subsection{Invarianza rispetto all'insieme degli item}

%  Nel contesto della misurazione psicologica, l'oggettività specifica richiede che (i) i confronti tra rispondenti risultino invarianti rispetto gli item usati per misurarli, e (ii) i confronti tra gli item risultino invarianti rispetto ai rispondenti usati per la calibrazione del questionario.

% L'invarianza rispetto agli item significa che la differenza rispetto al tratto latente tra due rispondenti deve essere indipendentemente dagli item utilizzati nel confronto.  L'invarianza rispetto gli item è implicita nel modello di Rasch. 

Seguendo Embretson e Reise (2000), consideriamo due rispondenti aventi livelli di abilità pari a $\theta_1=-2.20$ e  $\theta_2=-1.10$ e un item $i$ con  difficoltà pari a $\beta_i$. In base all'equazione~\ref{eq:rasch_logit}, i logit dei due soggetti sono:
\begin{align}
\ln \left[\frac{P(X_{1i}=1)}{1-P(X_{1i}=1)}\right] &= \theta_1-\beta_i\notag\\[10pt]
\ln \left[\frac{P(X_{2i}=1)}{1-P(X_{2i}=1)}\right] &= \theta_2-\beta_i\notag
\end{align}
La differenza tra i due logit è pari a:
\begin{align}
&\ln \left[\frac{P(X_{1i}=1)}{1-P(X_{1i}=1)}\right] - \ln \left[ \frac{P(X_{2i}=1)}{1-P(X_{2i}=1)} \right]\notag\\ 
 &=(\theta_1-\beta_i)-(\theta_2-\beta_i)
=\theta_1-\theta_2\notag\\ 
&=-2.20-(-1.10)=-1.10. \notag
\end{align}

Si noti che il confronto non dipende dalla difficoltà dell'item esaminato: infatti la differenza tra i due logit non dipende dal parametro $\beta_i$.  Ciò significa che la differenza tra le prestazioni dei rispondenti non dipende dagli item che sono stati usati: il confronto tra i rispondenti è dunque invariante rispetto all'insieme di item utilizzato; la differenza nelle prestazioni dei rispondenti dipende unicamente dai loro diversi livelli di abilità.

Nell'esempio appena discusso abbiamo considerato due rispondenti con un livello di abilità basso. Consideriamo ora due rispondenti con livelli alti di tratto,  $\theta_3=1.10$ e  $\theta_4=2.20$. La differenza tra i due logit 
\begin{align}
&\ln \left[\frac{P(X_{3i}=1)}{1-P(X_{3i}=1)}\right] - \ln \left[ \frac{P(X_{4i}=1)}{1-P(X_{4i}=1)}\right]\notag\\
&=(\theta_3-\beta_j)-(\theta_4-\beta_j)=\theta_3-\theta_4\notag\\
&=1.10-2.20=-1.10\notag
\end{align}
\noindent
è sempre uguale a $-1.10$. Anche in questo caso, la differenza delle prestazioni di due rispondenti dipende unicamente dai livelli di abilità. Le differenze misurate sulla scala dei logit sono dunque invarianti rispetto agli item utilizzati e non dipendono dalla posizione sulla scala di abilità. Le stime della prestazione sulla scala dei logit, dunque, possono essere considerate come delle misure a livello di scala ad intervalli equivalenti\footnote{Ricordiamo che, a livello di scala ad intervalli, non sono i valori della scala ad essere dotati di significato, ma le differenze tra valori della scala.}. 

%--------------------------------------------------------------
\subsection{Invarianza rispetto al campione di rispondenti}

In maniera analoga, le differenze nel livello di difficoltà degli item sono indipendentemente dal campione di rispondenti.  Si consideri, ad esempio, il confronto tra gli item 1 e 2, per un qualunque rispondente:
\begin{align}
&\ln \left[\frac{P(X_{v1}=1)}{1-P(X_{v1}=1)}\right] - \ln \left[ \frac{P(X_{v2}=1)}{1-P(X_{v2}=1)} \right] \notag\\
&=(\theta_v-\beta_1)-(\theta_v-\beta_2) =\beta_2-\beta_1.\notag
\end{align}
In altri termini, la differenza tra i logit di una risposta corretta a due diversi item dipende unicamente dalla differenza tra i livelli di difficoltà di tali item e non è influenzata dal livello di abilità dei rispondenti. Campioni diversi di rispondenti produrranno le stesse differenze tra i livelli di difficoltà degli item. Dunque, il confronto tra i livelli di difficoltà degli item risulta invariante rispetto al campione di rispondenti.  

%--------------------------------------------------------------
\subsection{Stima della difficoltà e del livello di abilità}

Nella Teoria Classica la difficoltà di un item è calcolata come la proporzione di individui che rispondono correttamente all'item nel campione. È facile capire come tale definizione della difficoltà di un item dipenda dalle caratteristiche del campione: uno stesso item può essere definito come facile in un campione composto da soggetti con un livello di abilità elevato (in cui quindi ci sarà una maggiore proporzione di risposte corrette), ma allo stesso tempo può essere definito come difficile in un campione con distribuzione di abilità bassa (che avrà una minore proporzione di risposte corrette). Nella Teoria Classica, quindi, la stima della difficoltà di un item è dipendente dal campione su cui viene stimata. Nei modelli IRT, invece, la difficoltà dell'item non dipende dalla distribuzione di abilità del campione a cui viene somministrato l'item. Nei modelli IRT, la difficoltà degli item viene inferita dalla ICC. In particolare, nel modello di Rasch la difficoltà viene interpretata come quel punto nella scala di abilità in cui la probabilità di risposta corretta è uguale a 0.5. 

Nella Teoria Classica, la stima del livello di abilità dipende dal test che viene somministrato: se si somministrano due test di diversa difficoltà allo stesso rispondente, la stima del livello di abilità sarà diversa. Al contrario nel modello di Rasch la stima dell'abilità è indipendente dal test che viene somministrato: il livello stimato di abilità rimane identico sia che venga somministrato un test facile sia che venga somministrato un test difficile. Poiché nella Teoria Classica il livello di abilità stimato è test dipendente, è possibile confrontare i punteggi di due rispondenti se e solo se sono stati sottoposti allo stesso test o a forme parallele di esso. Al contrario nel modello di Rasch è possibile confrontare i punteggi ottenuti a test diversi purché i punteggi grezzi siano trasformati in una unità di misura comune sia per il livello di abilità che per i parametri degli item.

Anche l'interpretazione dei punteggi ottenuti varia nelle due teorie. Nella Teoria Classica i punteggi ottenuti da un rispondente non vengono interpretati in modo assoluto ma vengono confrontati con il gruppo normativo. Nei modelli IRT, invece, i punteggi del rispondente vengono interpretati considerando la sua posizione rispetto al tratto latente. Poiché il modello di Rasch gode della proprietà di invarianza della misurazione, non è necessario interpretare la stima del livello di abilità del rispondente facendo riferimento ad un gruppo normativo. 


%--------------------------------------------------------------
\subsection{Invarianza dei confronti nei modelli 2PL e 3PL}

La caratteristica principale del modello di Rasch è l'invarianza dei confronti (\emph{invariance  of  comparison}) in base alla quale il confronto fra item risulta indipendente dagli individui a cui sono somministrati e, in modo simmetrico, il confronto fra individui risulta indipendente dagli item utilizzati. Questa desiderabile proprietà statistica del modello di Rasch non è invece presente nei modelli 2PL e 3PL. In tali modelli le curve caratteristiche degli item non sono parallele e, di conseguenza, il confronto tra item non dipende unicamente dalla difficoltà degli item, ma è influenzata anche dal potere discriminante e dal livello di guessing.  Infatti, si considerino due rispondenti con livelli di abilità $\theta_1$ e $\theta_2$, e un item $i$ con parametro di difficoltà $\beta_i$ e parametro di discriminazione $\alpha_i$. Per una risposta corretta, la differenza tra i logit di un modello 2PL 
\begin{align}
&\ln \left[\frac{P(X_{1i}=1)}{1-P(X_{1i}=1)}\right] - \ln \left[ \frac{P(X_{2i}=1)}{1-P(X_{2i}=1)} \right]\notag\\
&=\alpha_i(\theta_1 -\beta_i) -\alpha_i(\theta_2 -\beta_i)\notag\\
&= \alpha_i(\theta_1-\theta_2) \notag
\end{align}  
non dipende soltanto dal livello di abilità dei due rispondenti, ma anche dalla capacità discriminante dell'item considerato.  Ciò significa che l'invarianza rispetto agli individui e agli item esibita dal modello di Rasch non è presente nel modello 2PL. Un discorso analogo si può fare per il modello 3PL. 

%--------------------------------------------------------------
\subsection{Limiti del modello di Rasch}

Il termine ``invarianza'' usato nella discussione precedente sta a significare che il valore dei parametri del modello di Rasch non è influenzato da altri fattori di disturbo. Tuttavia, questo non significa che le conclusioni che traiamo dalla somministrazione di un questionario non dipendano dagli item che sono stati usati.  

In primo luogo, il modello di Rasch è molto restrittivo e non si adatta necessariamente ai dati empirici. Accade spesso, infatti, che le assunzioni del modello di Rasch risultino violate in molte ``scale'' con item dicotomici usate in psicologia. Il modello di Rasch, quindi, non è un metodo sempre applicabile per l'analisi di insiemi arbitrari di item dicotomici. 

In secondo luogo, posto che il modello di Rasch sia adatto a descrivere i dati empirici, la scelta degli item influenza l'errore di misurazione.   Una maggiore accuratezza della stima di livelli bassi di tratto, ad esempio, si ottiene mediante l'utilizzo di item aventi bassi livelli di difficoltà. In maniera equivalente, se il campione di rispondenti sul quale il questionario è stato calibrato esibisce bassi livelli del tratto esaminato, allora  i parametri degli item ``facili'' verranno  stimati con maggiore accuratezza dei parametri degli item ``difficili''. Questi  effetti si manifestano negli errori standard delle stime dei parametri del modello.

%------------------------------------------------------------------------
\section{Doppia cancellazione} 

Quali condizioni devono essere soddisfatte affinché un attributo psicologico  possa essere considerato a livello di scala ad intervalli? Per costruire una scala a tale livello di misurazione, devono essere soddisfatte appropriate condizioni, la principale delle quali è quella dell'additività.  La condizione dell'additività può essere verificata empiricamente mediante il principio della ``doppia cancellazione'' (Krantz, Luce, Suppes, \& Tversky, 1971; Luce, \& Tukey, 1964). 
Per illustrare il principio della doppia cancellazione, consideriamo nuovamente la tabella \ref{tab_rasch60}.  Nel caso di questi dati, l'ordinamento delle proporzioni di risposte corrette per qualunque coppia di item risulta essere lo stesso, indipendentemente dalla colonna che rappresenta il livello dell'abilità latente. Allo stesso modo, l'ordinamento relativo delle  proporzioni di risposte corrette rispecchia l'ordinamento relativo dei punteggi totali, indipendentemente da quale item (riga) venga considerato. 

\begin{table}
\centering
\begin{tabular}{ccccc}
\toprule
&\multicolumn{4}{c}{Punteggio Totale} \\
\cmidrule(r){2-5}
Item &  1  &   2 &   3  &  4 \\
\midrule
1 & 0.43 & 0.73 & 0.89 & 0.98\\
5 & 0.26 & 0.58&  0.79 & 0.96\\
4 & 0.09 & 0.33 & 0.59 & 0.91\\
2 & 0.04 & 0.28 & 0.46 & 0.88\\
3 & 0.04 & 0.08 & 0.27 & 0.74\\
\bottomrule
\end{tabular}
\caption{{\it Proporzione di risposte corrette per  ciascun item e per ciascun valore del punteggio totale nel test LSAT.} }
\label{tab_rasch60}
\end{table}

Il fatto che queste due proprietà vengano congiuntamente soddisfatte definisce il principio detto della ``doppia cancellazione''.  Tale principio implica dunque che, in una tabella quale la~\ref{tab_rasch60}, le  proporzioni di risposte corrette devono  aumentare, procedendo su ciascuna diagonale della tabella nella direzione che va da sinistra in basso a destra in alto. Di conseguenza, le curve caratteristiche degli item, che rappresentano tali proporzioni di risposte corrette, saranno parallele (vedi Figura~\ref{fig:RPlotLTA_2}).
%\ref{fig:icc1}). 
Il principio della ``doppia cancellazione'' che sta alla base della quantificazione gli attributi psicologici risulta dunque soddisfatto nel modello di Rasch.

Dalle considerazioni precedenti è facile desumere che modelli IRT più complessi, a due o tre parametri, non potranno soddisfare il principio della doppia cancellazione.  Embretson e Reise (2000) riportano quale esempio i dati della Tabella~\ref{tab_embretson6.7}, nella quale le probabilità predette da un modello IRT 2PL sulle diagonali della tabella talvolta aumentano, talvolta dimibuiscono.  

\begin{table}
\centering
\begin{tabular}{ccccccc}
\toprule
\multicolumn{2}{c}{} &\multicolumn{5}{c}{Abilità} \\
\cmidrule(r){3-7}
Difficoltà &  Discriminazione  &  -1.00 &   0.00  &  1.25 & 1.50  & 1.75\\
\midrule
-1.00 & 1.00  & 0.50 & 0.73 & 0.88 & 0.91 & 0.92\\
0.00 & 1.50   & 0.38&  0.50 & 0.62 & 0.65 & 0.68\\
0.25 & 0.50   & 0.13 & 0.41 & 0.76 & 0.82 & 0.87\\
1.00 & 1.00   & 0.12 & 0.27 & 0.50 & 0.56 & 0.62\\
\bottomrule
\end{tabular}
\caption{{\it Probabilità generate da un modello 2PL (Embretson e Reise, 2000).} }
\label{tab_embretson6.7}
\end{table}

%------------------------------------------------------------------------
\section{La critica di Michell}

Michell (2000) ha affermato che la psicometria dovrebbe essere classificata come un caso di ``patologia della scienza''. Secondo Michell l'indagine scientifica manifesta forme di patologia quando  si evidenziano i seguenti due problemi. Il primo si verifica quando un'ipotesi è ``accepted as 
true without a serious attempt being made to test it''. Il secondo consiste nell'atteggiamento che porta ad ignorare il limite precedente, così che il primo limite ``is not acknowledged or, in extreme cases, is disguised'' (p. 641). Secondo Michell, nella psicometria si manifesta questa patologia in quanto
\begin{quote}
[\dots] (a) a basic, empirical hypothesis (namely the hypothesis that psychological attributes are quantitative) is accepted as true without it ever having been seriously tested for its empirical adequacy, and (b) the fact that this hypothesis has never been satisfactorily tested is disguised (p. 650).  
\end{quote}

Sono state fornite varie risposte alle critiche di Michell (2000). Borsboom e Mellenbergh (2004), per esempio, ritengono che tali critiche non si applichino al modello di Rasch. Il modello di Rasch, infatti, non dà per scontata l'assunzione che gli attributi psicologici siano quantitativi (anziché qualitativi) ma sottopone a verifica empirica tale ipotesi. 

Tali critiche sono invece rilevanti per la Teoria Classica dei test e per la gran parte delle pratiche di misurazione in psicologia.  Non potendo essere sottoposta a verifica, la Teoria Classica dovrebbe essere considerata una tautologia piuttosto che un modello scientifico (Lord \& Novick, 1968, p. 48). Il concetto centrale di tale teoria, ovvero il punteggio vero, è concettualizzato come il valore atteso del punteggio di un  rispondente che, a causa degli errori di misurazione, non può essere direttamente osservato. Dato che è impossibile ottenere una serie infinita di repliche indipendenti del punteggio osservato di un rispondente, il valore atteso non può essere calcolato. Per ricavare il punteggio vero, la Teoria Classica deve  basarsi su un esperimento di pensiero ({\it gedanken experiment}) (Lord \& Novick, 1968, p. 29). Dato che è privo di qualunque restrizione empirica, un tale esperimento può sempre essere eseguito (ovvero, non può mai fallire).  

Una volta stabiliti i punteggi veri, tutti i teoremi della Teoria Classica risultano verificati (Lord \& Novick, 1968). Dato che tali teoremi rappresentano l'associazione tra i punteggi veri e quelli osservati mediante una relazione lineare, i punteggi veri devono essere concepiti a livello di scala ad intervalli.  Ma questa è soltanto un'assunzione che non può essere verificata, a meno di introdurre introdotte ulteriori assunzioni nel modello (quali ad esempio l'eguaglianza tra le varianze dei rispondenti). Il punto cruciale è che il livello di misurazione della Teoria Classica non viene mai sottoposto a verifica empirica e dunque la critica di Michell (2000), secondo cui  i modelli psicometrici sono basati su assunzioni che non vengono mai verificate, è sicuramente rilevante per la Teoria Classica e per la ricerca psicologica basata su di essa. 

\subsubsection{Implicazioni per la pratica dell'analisi dei dati psicologici}

Il fatto che i dati non siano a livello di scala ad intervalli ha importanti conseguenze per le inferenze statistiche. Maxwell e Delaney (1985), per esempio, hanno mostrato come i risultati di un test $t$ di Student per due gruppi indipendenti risultino distorti se i dati non sono a livello di scala ad intervalli.  Se i punteggi osservati non sono linearmente associati ai punteggi ``veri'', allora il risultato del test può risultare ``statisticamente significativo'' anche se i campioni sono estratti dalla stessa popolazione. Inoltre, Embretson (1997) ha mostrato che interazioni spurie possono risultare dall'analisi della varianza, se i punteggi grezzi sono associati solo in modo ordinale ai valori del tratto latente. 

Si può dunque facilmente intuire che, qualora vengano analizzati mediante la Teoria Classica o i modelli IRT, gli stessi dati possono portare a conclusioni diverse. Si noti infatti che, se i livelli di abilità stimati dal modello di Rasch sono su una scala ad intervalli, allora le proporzioni di risposte corrette (la tipica misura di prestazione per la Teoria Classica) non lo saranno. La relazione tra la proporzione di risposte corrette e i livelli del tratto latente, infatti, non è lineare. 


%------------------------------------------------------------------------
\section{Statistiche sufficienti}

I lavori di Rasch (1960, 1961 e 1977) hanno messo in evidenza la connessione tra l'oggettività specifica da un lato e le statistiche sufficienti dall'altro. Andersen (1977) ha poi dimostrato che l'oggettività specifica richiede l'esistenza di statistiche sufficienti per i parametri di un modello probabilistico. Il modello di Rasch, ammettendo l'esistenza di statistiche sufficienti, assicura la proprietà dell'oggettività specifica.  

L'esistenza di statistiche sufficienti, oltre la proprietà della oggettività specifica, garantisce al modello di Rasch anche la possibilità di ottenere stimatori con proprietà desiderabili come la correttezza e la consistenza, a patto di utilizzare metodi di stima adeguati (Hambleton e Swaminathan, 1985). Tali proprietà di correttezza e consistenza non sono invece garantite per altri modelli della classe IRT che nel ``generalizzare'' il modello di Rasch perdono le proprietà della sufficienza e dell'oggettività specifica, oltre a presentare problemi di stima dei parametri.

Per chiarire il concetto di ``statistica sufficiente'' consideriamo la matrice dei dati della Tabella~\ref{tab:matr_dati}.  Nelle righe sono riportati i rispondenti (indicizzati con $v = 1, \dots, n$) e nelle colonne sono riportati gli item (indicizzati con $i=1, \dots, P$). Ogni cella contiene la risposta del $v$-esimo rispondente all'$i$-esimo item. La risposta è codificata con $1 =$ corretta e $0 = $ errata.  

\begin{table}
\centering
\begin{tabular}{ccccccc}
\toprule
&\multicolumn{6}{c}{Item} \\
\cmidrule(l){2-7}
Rispondenti &  1  &   2 &   3  &  \dots & 5 \\
\midrule
1     &  $X_{11}$  &  $X_{12}$  &  $X_{13}$  &  \dots  &  $X_{1P}$ & $Y_{1.}$\\
2     &  $X_{21}$  &  $X_{22}$  &  $X_{23}$  &  \dots  &  $X_{2P}$ & $Y_{2.}$\\
3     &  $X_{31}$  &  $X_{32}$  &  $X_{33}$  &  \dots  &  $X_{3P}$ & $Y_{3.}$\\
\dots &  \dots    &  \dots    &  \dots    &  \dots  &  \dots   & \dots \\
n     &  $X_{n1}$  &  $X_{n2}$  &  $X_{n3}$  &  \dots  &  $X_{nP}$ & $Y_{n.}$\\
\midrule
      &  $R_{.1}$  &  $R_{.2}$  &  $R_{.3}$  &  \dots  &  $R_{.P}$ &       \\
\bottomrule
\end{tabular}
\caption{{\it Matrice dei dati relativa a $J$ item somministrati a $n$ rispondenti.} }
\label{tab:matr_dati}
\end{table}

Il totale marginale di riga rappresenta il numero complessivo di risposte corrette del $v$-esimo rispondente, $Y_{v.}=\sum_{i=1}^P  X_{vi}$. Il totale marginale di colonna rappresenta il numero complessivo di risposte corrette che l'$i$-esimo item ha ricevuto da tutti gli $n$ rispondenti, $R_{.i}=\sum_{v=1}^n  X_{vi}$. 
Il fatto che $Y_{v.}$ sia una statistica sufficiente per $\theta$ significa che, per stimare il parametro di abilità, non è necessario conoscere il livello di difficoltà degli item. La stima dell'abilità $\theta_v$ è dunque invariante rispetto alla difficoltà degli item.  Allo stesso modo, per stimare il livello di difficoltà dell'$i$-esimo item non è necessario conoscere i livelli di abilità dei rispondenti. La stima del parametro $\beta_i$ di difficoltà degli item è dunque invariante rispetto al livello di abilità dei rispondenti. Si dice dunque che $R_{.i}$ è una statistica sufficiente per il parametro $\beta_i$.  


%------------------------------------------------------------------------
\section{Excursus storico}

Il problema della costruzione di variabili psicologiche quantitative fu affrontato da Louis Thurstone in numerose pubblicazioni tra il 1925 e il 1932. La scala di Thurstone formalizzò per la prima volta in psicologia il problema della misurazione degli atteggiamenti riguardando, nello specifico, la misurazione degli atteggiamenti religiosi. Thurstone mise in evidenza alcuni dei requisiti, ancora validi, che sono necessari per ottenere delle misurazioni valide.
\begin{enumerate}
 \item \emph{Unidimensionalità}: ``The measurement of any object or entity describes only one attribute of the object measured. This is a universal characteristic of all measurement'' (Thurstone, 1931, p. 257). 
 \item \emph{Linearità}: ``The very idea of measurement implies a linear continuum of some sort such as length, price, volume, weight, age. When the idea of measurement is applied to scholastic achievement, for example, it is necessary to force the qualitative variations into a scholastic linear scale of some kind'' (Thurstone \& Chave, 1929, p. 11). 
 \item \emph{Invarianza}:  ``[\dots] which can be repeated without modification in the different parts of the measurement continuum'' (Thurstone 1931, p. 257).
 \item \emph{Calibrazione indipendente dal campione}: ``The scale must transcend the group measured. A measuring instrument must not be seriously affected in its measuring function by the object of measurement [\dots] Within the range of objects [\dots] intended, its function must be independent of the object of measurement'' (Thurstone, 1928, p. 547).
 \item \emph{Calibrazione indipendente dagli item}: ``It should be possible to omit several test questions at different levels of the scale without affecting the individual score (measure) [\dots] It should not be required to submit every subject to the whole range of the scale. The starting point and the terminal point [\dots] should not directly affect the individual score (measure)'' (Thurstone, 1926, p. 446).
 \end{enumerate}

Il metodo di scaling proposto da Thurstone consisteva nella conversione delle frequenze relative di una matrice di confusione ottenuta mediante il metodo dei confronti a coppie in punteggi della cumulativa di una normale standardizzata. Il limite di tale procedura è che produce risultati indeterminati nel caso di proporzioni pari a 0 e 1 (corrispondenti ai valori $+\infty$ e $-\infty$, rispettivamente); semplici varianti del metodo proposto da Thurstone consentono tuttavia di risolvere facilmente tale problema (Krus, \& Kennedy, 1977).  

La scala di Thurstone ha una stretta somiglianza concettuale con il modello di Rasch (Andrich, 1978), le principali differenze essendo  che (\emph{i}) per il modello di Rasch,  la forma funzionale della curva caratteristica degli item è quella  logistica piuttosto che la cumulativa  gaussiana, (\emph{ii}) per il modello di Rasch la curva caratteristica degli item dipende da due parametri: la difficoltà degli item e l'abilità dei rispondenti.
