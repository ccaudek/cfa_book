% DO NOT COMPILE THIS FILE DIRECTLY!
% This is included by the other .tex files.
%%------------------------------------------------------------
\chapter{Interpretazione e standardizzazione dei punteggi}
\label{ch:standardizzazione}
%%------------------------------------------------------------

Il modello di misurazione che è stato trattato in precedenza permette di ottenere misure accurate del punteggio di un soggetto in una variabile non osservabile. Le considerazioni relative all'errore standard di misurazione e alla stima del punteggio vero, inoltre, consentono di stimare il punteggio vero di un soggetto a partire dal suo punteggio osservato in base alla variabilità e all'attendibilità dei punteggi del test. Queste considerazioni, però, non forniscono ancora una risposta al problema fondamentale di interpretare il punteggio ottenuto al test. I punteggi dei test psicologici, presi da soli, non hanno infatti alcun significato, in quanto sono misure su scala ad intervalli.  La mancanza di uno zero assoluto (ovvero di un valore della scala che indica l'assenza della caratteristica misurata) fa sì che i punteggi grezzi siano interpretabili solo in relazione a un punteggio di riferimento. I punteggi di riferimento nella popolazione prendono il nome di \textit{punteggi nomativi}, o \textit{norme}. Tali valori corrispondono alle statistiche descrittive  dei punteggi del test calcolate sul \textit{campione di standardizzazione} o \textit{campione normativo}. 

Per eseguire correttamente la standardizzazione di un test è necessario:
\begin{enumerate}
\item identificare la popolazione target;
\item determinare il metodo di campionamento e l'ampiezza campionaria adeguata, e raccogliere i dati;
\item calcolare gli indici statistici idonei all'interpretazione dei punteggi grezzi; 
\item inserire le informazioni sulle norme nel manuale insieme alle procedure di somministrazione, scoring dei punteggi e interpretazione dei risultati.
\end{enumerate}

%--------------------------------------------------------------------------------
\section{Standardizzazione}
%--------------------------------------------------------------------------------

Tranne particolari eccezioni, le trasformazioni dei punteggi grezzi sono trasformazioni monotone crescenti, ovvero trasformazioni che non alterano l'ordine della graduatoria dei rispondenti espressa nei termini dei punteggi grezzi. Le trasformazioni rendono i punteggi grezzi interpretabili tramite il confronto tra il punteggio di un rispondente e i punteggi ottenuti dagli altri rispondenti, o tramite il confronto tra il punteggio di un rispondente e un criterio assoluto.


Nella standardizzazione \emph{riferita alle norme} il punteggio del rispondente è confrontato con i punteggi di un gruppo di riferimento. 
Nella standardizzazione \emph{riferita al criterio} il punteggio del rispondente è confrontato con un criterio assoluto di riferimento.
Nella standardizzazione \emph{ipsativa} si confrontano i punteggi su test diversi allo scopo di identificare ``i punti di forza'' e le ``i punti deboli'' del rispondente (talvolta chiamati ``training needs'').


%--------------------------------------------------------------------------------
\section{Test riferiti alle norme}
%--------------------------------------------------------------------------------




Quando la valutazione fa riferimento alle norme, il punteggio del rispondente viene confrontato con la prestazione di un gruppo di riferimento.
 Standardizzare un test significa stabilire delle norme, ossia degli insiemi di punteggi ottenuti da campioni chiaramente definiti.
 La standardizzazione conferisce un significato psicologico ai punteggi di un test e rende possibile la sua interpretazione psicologica stabilendo la posizione relativa del rispondente nei confronti della ``norma'' (gruppo di riferimento).

Se un individuo ottiene il punteggio $a$ in un test di intelligenza, il significato di questo punteggio è molto diverso se solo il 2\% della popolazione ottiene un punteggio maggiore di $a$, oppure se il 90\% della popolazione ha un punteggio maggiore di $a$.
 Senza il riferimento ad una norma è impossibile cogliere il significato del punteggio di un test se il test viene usato come procedura di selezione, se il test viene usato per prendere decisione sugli individui, nell'orientamento professionale o nella selezione del personale.

Il fatto che i risultati di un test siano standardizzabili, e quindi interpretabili mediante un confronto con un campione normativo, rende i test una forma di valutazione particolarmente efficace e vantaggiosa rispetto ad altri metodi che non possono essere standardizzati allo stesso modo, come ad esempio
le interviste o 
il colloquio clinico.

Se le norme vengono usate per fornire un'interpretazione ai punteggi di un test è necessario che le norme siano accurate.
    Le norme devono riflettere le caratteristiche della popolazione e non una peculiarità del campione normativo.
      Se le norme sono imprecise, le decisioni che vengono prese sulla base di esse saranno necessariamente sbagliate.

 Il campionamento è dunque un fattore cruciale nella standardizzazione di un test.
     La qualità delle norme dipende dall'adeguatezza dei campioni sui quali sono basate.
   Nel campionamento ci sono due variabili importanti:
  la numerosità e 
la rappresentatività.
Il campione deve essere sufficientemente ampio da rendere trascurabili gli errori standard delle statistiche descrittive.
     Per rendere sufficientemente piccoli gli errori standard, un campione di 500 soggetti è considerato adeguato (Kline, 1993).
La rappresentatività del campione si ottiene utilizzando le tecniche di campionamento messe a punto dalla teoria statistica, la cui scelta dipende dall'eterogeneità della popolazione (
campionamento casuale semplice, 
    campionamento casuale indipendente,
    campionamento casuale stratificato,
    campionamento a grappoli,
    campionamento multistadio),
 Se il test viene somministrato a buoni gruppi normativi, i risultati della somministrazione possono essere espressi in modi diversi.
     Tutti questi metodi consentono il confronto tra il punteggio di un individuo con il corrispondente gruppo normativo e introducono una qualche trasformazione del punteggio grezzo che consente di indicare la posizione dell'individuo in relazione al gruppo:
percentili, punteggi standard  e trasformazioni dei punteggi z (punteggi stanini, stens, ecc.).

\subsection{Percentili}

Si chiama punteggio percentile il punteggio al di sotto del quale ricade una determinata percentuale del gruppo normativo. I percentili stabiliscono dunque la posizione relativa del rispondente all'interno del campione normativo.

\begin{exmp}
Ad esempio, se il 75\% dei membri del gruppo normativo ottiene un punteggio minore o uguale a 24.6, allora il valore 24.6 indica il punteggio percentile pari al 75\%.
\end{exmp}
I percentili offrono una facile interpretazione ma hanno anche degli svantaggi:
 essendo punteggi ordinali non consentono analisi statistiche,
 la distribuzione dei percentili è rettangolare mentre la distribuzione dei punteggi grezzi è solitamente gaussiana.
 Dunque, una differenza costante (es. 5 punti percentili) riflette una differenza nelle prestazioni molto più grande agli estremi della distribuzione che in prossimità della media, se la distribuzione dei punteggi grezzi dell'abilità misurata dal test è gaussiana.
   Le norme basate sui percentili, dunque, non vengono raccomandate se non per dare spiegazioni ai risultati che siano facilmente comprensibili ai non addetti ai lavori.
 
\begin{figure}
\begin{center}
\includegraphics[width=\textwidth]{figure2-8}
\end{center}
\end{figure}

Talvolta i punteggi di un reattivo sono riportati nei termini di quelli che vengono chiamati ``age scores'' o ``grade scores.''  Per determinare l'age score (o grade score), viene calcolato il punteggio grezzo mediano per un determinato livello d'età (o per un determinato livello scolastico, es. terza elementare).

\begin{exmp}
 Per esempio, se la mediana del punteggio grezzo in un test di lettura assume il valore di 12.7 per i rispondenti di terza elementare, allora il grade score per la terza elementare assumerà il valore di 12.7.  E così via per la prima elementare, la seconda ecc.
In maniera corrispondente si calcolano i valori dell'age score.
\end{exmp}
Nonostante la loro popolarità, i punteggi di age score o grade score hanno molte limitazioni.
 Dato che sono punteggi ordinali,  manipolazioni aritmetiche non sono possibili.
Inoltre ci possono portare ad assumere erroneamente che due bambini aventi la stessa età mentale, o lo stesso age score, possiedano simili abilità cognitive, ma in generale questo non è vero: ci sono enormi differenze nelle abilità cognitive di bambini che hanno la stessa età mentale (age score).
L'uso degli age score o dei grade score può dunque incoraggiare interpretazioni fuorvianti.
Un ulteriore problema con l'uso degli age score è che le distribuzioni dei punteggi tendono a sovrapporsi sempre di più al crescere dei livelli d'età.
     Per esempio, per un bambino di 5 anni, avere un age score di due anni più elevato è eccezionale; per un bambino di 11 anni, avere un age score di due anni più elevato non è tanto sorprendente.
     Questo problema rende difficile il confronto tra le prestazioni di rispondenti con livelli d'età diversi.

\begin{figure}
\begin{center}
\includegraphics[width=\linewidth]{agegrade.pdf}
\end{center}
\end{figure}

Inoltre, non si deve dimenticare che, per come l'age score è stato definito (cioè, attraverso la mediana), in ciascuno dei gruppi identificati metà dei rispondenti avranno un punteggio minore dell'age score e metà dei rispondenti avranno un punteggio maggiore dell'age score.
  Quindi è probabile che solo pochi rispondenti abbiano un livello di prestazione simile a quello indicato dall'age score.

%--------------------------------------------------------------------------------
\subsection{Punti z}
%--------------------------------------------------------------------------------

La standardizzazione mediante punteggi standard avviene trasformando i punteggi grezzi in punti $z$ :
\[
z = \frac{x-M}{s},
\]
dove $x$ è il punteggio del soggetto, $M$ è la media dei punteggi del campione normativo e $s$ è la deviazione standard del campione normativo. 

Rispetto ai punteggi grezzi, i punti $z$ offrono una facile interpretazione: se i valori si distribuiscono in maniera gaussiana, un valore $z$ = 2 significa due deviazioni standard sopra la media.
  Questo consente di collocare facilmente la posizione di un rispondente all'interno della distribuzione del campione normativo e consente anche il confronto tra i punti $z$ ottenuti da un rispondente in test diversi.
  È importante però notare come la trasformazione in punti $z$ non alteri la forma della distribuzione (un esempio è fornito di seguito); dunque l'interpretazione precedente è giustificata solo se la distribuzione dei punteggi è effettivamente gaussiana.
I punteggi $z$ hanno anche degli svantaggi:
il significato di una scala che va da -3 a +3, con media 0, è difficile da afferrare per i non specialisti;
se la distribuzione dei punteggi grezzi non è gaussiana, i punti $z$ non consentono di stabilire la posizione relativa del punteggio del rispondente all'interno della distribuzione del campione normativo.


%--------------------------------------------------------------------------------
\subsection{Trasformazioni dei punti $z$ in altri punteggi standard}
%--------------------------------------------------------------------------------

I punteggi grezzi non vengono sempre trasformati in punti $z$. Altre forme di standardizzazione sono possibili.
 Tali forme di standardizzazione sono basate su una trasformazione lineare che rappresenta una trasformazione ammissibile per i valori della scala nel caso di misure espresse su una scala ad intervalli.
In altri casi, trasformazioni non lineari vengono usate.
Consideriamo la formula seguente
$$
z_t = a + bz,
$$
dove $z_t$ sono i punteggi standardizzati, $a$ è la media della distribuzione trasformata, $b$ è la deviazione standard della distribuzione trasformata e $z$ è la trasformazione in punti $z$ dei dati originari.
Mediante questa formula è possibile trasformare i punteggi grezzi nei punteggi di una nuova distribuzione avente media $a$ e deviazione standard $b$.
Si noti che la trasformazione va applicata ai punti $z$ e non direttamente ai dati grezzi.
Ad esempio, consideriamo i valori rappresentati dalla distribuzione della figura seguente.
I punteggi grezzi qui rappresentati hanno media 14.5 e deviazione standard 2.11.

\begin{figure}[h!]
\begin{center}
\includegraphics[width=7cm]{standardizz1.pdf}
\end{center}
\end{figure}
Se trasformiamo i punteggi grezzi in punti $z$ otteniamo la seguente distribuzione:
\begin{figure}[h!]
\begin{center}
\includegraphics[width=7cm]{standardizz3.pdf}
\end{center}
\end{figure}
Si noti che la forma della distribuzione è rimasta invariata.
 I punti $z$ hanno media 0 e deviazione standard uguale a 1.
Applichiamo ora una trasformazione lineare ai punti $z$, ovvero
$$
Y' = 50 + 10 z
$$
In questo modo otteniamo la seguente distribuzione:
\begin{figure}[h!]
\begin{center}
\includegraphics[width=7cm]{standardizz2.pdf}
\end{center}
\end{figure}
I punteggi standardizzati così ottenuti hanno media 50 e deviazione standarad pari a 10.
Si noti però che la forma della distribuzione è rimasta invariata, in quanto la trasformazione lineare $z_t = a + bz$ non  altera la forma della distribuzione.
Cambia invece l'origine della scala e l'unità di misura; di conseguenza la media di punteggi cambia, così come la deviazione standard.
Ma la forma della distribuzione resta immutata.
Cambiando i valori $a$ e $b$ nella formula $z_t = a + bz$ possiamo ottenere punteggi standardizzati aventi qualunque media e qualunque deviazione standard.


%--------------------------------------------------------------------------------
\paragraph{Punteggi T}
%--------------------------------------------------------------------------------

Secondo Crombach (1976) la trasformazione lineare più comune dei punteggi $z$ è quella che produce una distribuzione con media 50 e deviazione standard  pari a 10 (come nell'esempio precedente).
Tali punteggi standardizzati vanno sotto il nome di \emph{punteggi T}.
I punteggi-T si trovano calcolando  prima i punteggi $z$ e poi trasformando tali punteggi in maniera tale che abbiano  media media 50 e deviazione standard 10 con la formula seguente:
$$
T = 50 + 10 z.
$$
Anche se i punteggi-T costituiscono un miglioramento rispetto ai punteggi $z$, si prestano anch'essi ad interpretazioni sbagliate.
 Se i valori dell'abilità misurata dal test si distribuiscono in maniera gaussiana, allora i punteggi-T varieranno da circa 20 a circa 80 (media $\pm$ 3 deviazioni standard).
 Tuttavia, questo non è vero se la distribuzione non è gaussiana.
 Inoltre, è importante non interpretare i punteggi-T come se fossero delle proporzioni, dato che non hanno questo significato.


%--------------------------------------------------------------------------------
\paragraph{Punteggi stanini}
%--------------------------------------------------------------------------------

I punteggi stanzini (standard nine) sono dei numeri interi, da 1 a 9.
 Ad un individuo che ha un punteggio compreso nel secondo intervallo inferiore, per esempio, verrà attribuito il punteggio stanino di 2, indipendentemente dalla sua posizione all'interno dell'intervallo.
 I punteggi stanini hanno media 5 e deviazione standard pari a 2.
Partendo dai punteggi $z$, i punteggi stanini si ottengono nel modo seguente:
$$
\text{Punteggi stanini} = 5 + 2z.
$$
I punteggi stanzini dividono i punteggi grezzi, se sono distribuiti in maniera gaussiana, o una loro trasformazione distribuita in maniera gaussiana, in nove intervalli.
 L'area sottesa alla curva normale in ciascun intervallo non è però identica, come indicato nella figura seguente.
\begin{figure}[h!]
\begin{center}
\includegraphics[width=7cm]{figure2-8}
\end{center}
\end{figure}
I punteggi stanini venivano usati soprattutto quando i computer non erano disponibili.
 Al giorno d'oggi solo pochi test usano i punteggi stanini quale strumento di standardizzazione riferita alle norme.
Partendo dai punteggi $z$, i punteggi T, stanini, Sten e quelli simili al QI si ottengono nel modo seguente:
\begin{align*}
\text{Punteggi T} &= 50 + 10z\\
\text{Punteggi stanini} &= 5 + 2z\\
\text{Punteggi Sten} &= 5.5 + 2z\\
\text{Formato simile al QI} &= 100 + 15z
\end{align*}
\begin{figure}[h!]
\begin{center}
\includegraphics[width=7cm]{standardized-scores.pdf}
\end{center}
\end{figure}

Alcuni costruttori di test preferiscono i punteggi standard normalizzati.
 I punteggi grezzi del gruppo normativo vengono trasformati con una trasformazione non-lineare in maniera tale che la distribuzione dei punteggi trasformati diventi approssimativamente gaussiana.
\begin{figure}[h!]
\begin{center}
\includegraphics[width=7cm]{infant.pdf}
\end{center}
\end{figure}
\begin{figure}[h!]
\begin{center}
\includegraphics[width=7cm]{infantlog10.pdf}
\end{center}
\end{figure}
Una volta che la forma della distribuzione è stata alterata in modo da approssimarla alla normale, possono essere applicate  trasformazioni di diversi tipi.
 Ciascuna di queste trasformazioni consentirà una interpretazione basata sulla distribuzione gaussiana.
 I punteggi così derivati vanno sotto il nome di punteggi standard normalizzati (normalized standard scores) e sono talvolta chiamati ``punteggi basati su trasformazioni d'area'' anziché punteggi basati su trasformazioni lineari.




%%--------------------------------------------------------------------------------
%\subsection{Punteggi NCE}
%%--------------------------------------------------------------------------------
%
%\begin{frame}{Punteggi NCE}
%
%\begin{itemize}
%\item I punteggi NCE (Normal Curve Equivalent) indicano la posizione di un rispondente rispetto alla distribuzione normale.
%\item I punteggi NCE variano da 0 a 100 e, come i percentili, indicano quanti rispondenti ottengono un punteggio inferiore a quello considerato.
%\item Tuttavia, non vi è una relazione lineare tra i percentili e i punteggi NCE, come indicato nella figura seguente.
%\end{itemize}
%
%\end{frame}
%
%%--------------------------------------------------------------------------------
%
%\begin{frame}{Punteggi NCE}
%
%\begin{figure}
%\begin{center}
%\includegraphics[width=11cm]{nce.pdf}
%\end{center}
%\end{figure}
%
%\end{frame}
%
%
%%--------------------------------------------------------------------------------
%
%\begin{frame}{Punteggi NCE e percentili}
%
%\begin{itemize}
%\item I punteggi NCE rappresentano dunque i valori della funzione distribuzione (espressa in termini percentuali) associati ai quantili della distribuzione gaussiana.
%\item Ad esempio, un valore $z$ = 0 corrisponde ad un valore NCE = 50; il valore $z$ = 3 corrisponde circa a NCE = 100; il valore $Z$ = -3 corrisponde circa a NCE = 0.
%\end{itemize}
%
%\end{frame}
%
%%--------------------------------------------------------------------------------
%
%\begin{frame}{Punteggi NCE}
%
%\begin{itemize}
%\item I punteggi NCE hanno una media di 50 e una deviazione standard di 21.06.
%\end{itemize}
%
%\end{frame}


%%--------------------------------------------------------------------------------
%
%\begin{frame}{Interpretazione}
%
%\begin{itemize}
%\item Sapendo che il punteggio T = 40, come troviamo il punteggio percentile?
%\item Sappiamo che i punteggi T si distribuiscono con media = 50 e deviazione standard = 10.
%\item Dunque dobbiamo trovare l'area sottesa alla curva normale con media 50 e deviazione standard 10 nell'intervallo compreso tra $-\infty$ e 40.
%\end{itemize}
%
%\end{frame}
%
%%--------------------------------------------------------------------------------
%
%\begin{frame}{Interpretazione}
%
%\begin{itemize}
%\item Trasformando il punteggio T in punto $z$ troviamo che $z$ = -1 e l'area corrispondente è 0.16, il che corrisponde al 16\% percentile.
%\item Lo stesso ragionamento si segue per le altre forme di standardizzazione.
%\end{itemize}
%
%\end{frame}


%--------------------------------------------------------------------------------

Riassumendo:

\begin{itemize}
\item Scala espressa in puteggi $z$:\\ media = 0, deviazione standard = 1.
\item Scala espressa in puteggi T:\\ media = 50, deviazione standard = 10.
\item Scala espressa in puteggi Sten:\\ media = 5.5, deviazione standard = 2.
\item Scala espressa in puteggi stanini:\\ media = 5, deviazione standard = 2.
\end{itemize}
I punteggi T sono spesso usati (in alternativa ai percentili) nell'interpretazione dei testi di abilità in quanto consentono di esprimere il punteggio del candidato con un alto livello di precisione, solitamente nella gamma di punteggi compresa tra 20 e 80.
 Le misure di personalità tendono ad essere meno precise e per questa ragione i punteggi dei rispondenti sono solitamente  espressi nei termini di punteggi Sten (nella gamma 1 -- 10) o nei termini dei punteggi stanini (nella gamma 1 -- 9).



%--------------------------------------------------------------------------------
\subsection{Norme}
%--------------------------------------------------------------------------------

Abbiamo visto come le norme si riferiscano alle distribuzioni dei punteggi di un gruppo di rispondenti nel caso di uno specifico test. Le norme sono utilizzate per fornire informazioni a proposito della prestazione  che è stata osservata nel campione normativo esaminato. La media è una norma, così come il cinquantesimo percentile, ad esempio.
Uno degli aspetti più controversi nell'uso dei test standardizzati è che il livello di prestazione in molti test (es., i test di intelligenza) varia in popolazioni diverse definite da fattori quali quelli etnico-razziali, ad esempio.
 Un approccio per affrontare questo problema è quello di creare norme diverse per gruppi diversi.
 In base a questo approccio, le prestazioni di un individuo vengono confrontate con quelle di coloro che appartengono allo stesso gruppo.
 Questo però pone il problema che, ad esempio, diversi criteri di selezione vengono stabiliti per gruppi diversi, così da potenzialmente portare a preferire un individuo con un livello di abilità minore di un altro, solo perché appartiene ad un gruppo con norme ``meno stringenti.''
Alcuni test forniscono norme diverse per gruppi d'età diversi.
 Nello sviluppo storico dello Standford-Binet, ad esempio, vennero ottenute le distribuzioni dei punteggi di campioni casuali di bambini di età diverse.
 Lo scopo del test era quello di determinare ``l'età mentale'' del bambino sottoposto al test.
 Questo veniva ottenuto mettendo in relazione le prestazioni del bambino esaminato con quelle ottenute da gruppi di bambini di età diverse.
Uno degli usi comuni delle norme riferite all'età è quello dei diagrammi di crescita (growth charts) usati dai pediatri.
 L'altezza dei bambini, ad esempio, viene confrontata con l'altezza media dei bambini di età diverse.
 Ma i bambini seguono percorsi di sviluppo diversi: quelli che sono piccoli alla nascita tendono a rimanere più piccoli degli altri e crescono più lentamente.
 I pediatri devono dunque conoscere non solo l'età dei bambini ma anche il percentile all'interno di un gruppo d'età.
 Per molte caratteristiche fisiche, i bambini tendono a esibire caratteristiche simili a quelle dei bambini del loro stesso livello percentile.
La tendenza ad avere caratteristiche simili a quelle del gruppo percentile di appartenenza va sotto il nome di tracking.
Altezza e peso sono due variabili che si comportano in questo modo.

%
%%--------------------------------------------------------------------------------
%
%\begin{frame}{Tracking}
%
%\begin{figure}
%\begin{center}
%\includegraphics[width=6cm]{boys36}
%\end{center}
%\end{figure}
%
%\end{frame}
%
%%--------------------------------------------------------------------------------
%
%\begin{frame}{Tracking}
%
%\begin{figure}
%\begin{center}
%\includegraphics[width=6cm]{girls36}
%\end{center}
%\end{figure}
%
%\end{frame}
%
%
%%--------------------------------------------------------------------------------
%
%\begin{frame}{Tracking}
%
%\begin{itemize}
%\item Si noti che i bambini che sono più alti ad una certa età tendono a rimanere più alti degli altri anche ad età successive.
%\item Questi grafici forniscono dunque l'andamento atteso dello sviluppo nel tempo.
%\end{itemize}
%
%\end{frame}
%
%
%%--------------------------------------------------------------------------------
%
%\begin{frame}{Tracking}
%
%\begin{itemize}
%\item La figura successiva mostra l'andamento di crescita di un bambino che segue un percorso di sviluppo anomalo.
%\item I dati provengono da uno studio di Kaplan e Toshima (1992) e riguardano un bambino che seguiva una dieta inadeguata perché voleva ``perdere peso.''
%\end{itemize}
%
%\end{frame}
%
%
%%--------------------------------------------------------------------------------
%
%\begin{frame}{Tracking}
%
%\begin{figure}
%\begin{center}
%\includegraphics[width=6cm]{figure2-11}
%\end{center}
%\end{figure}
%
%\end{frame}
%
%
%%--------------------------------------------------------------------------------
%
%\begin{frame}{Tracking}
%
%\begin{itemize}
%\item Il punto evidenziato dalla freccia indica il momento in cui fu iniziata una psicoterapia.
%\item Da quel momento, una dieta bilanciata venne seguita e lo sviluppo potè proseguire.
%\item Comunque, a 18 anni, il bambino apparteneva al quinto percentile in termini di altezza e peso.
%\item Se lo sviluppo avesse proseguito normalmente, il bambino avrebbe dovuto avere peso e altezza tra il venticinquesimo e il cinquantesimo percentile.
%\end{itemize}
%
%\end{frame}


%--------------------------------------------------------------------------------
Il tracking ha senso dal punto di vista dello sviluppo fisico ma non è detto che un approccio simile abbia senso se consideriamo le prestazioni scolastiche, ad esempio.
 Alcuni ritengono che i bambini che imparano più lentamente diventeranno degli adulti con minori capacità intellettive di altri.
È stato anche proposto che alcuni bambini imparano più o meno velocemente di altri, e questo ha prodotto l'idea di separare i percorsi educativi di bambini con capacità diverse di apprendimento.
Questa proposta è stata criticata da molti per le ovvie implicazioni di discriminazione che possiede per i bambini ``più svantaggiati.''
 Si noti che è lo psicologo che deve decidere quali sono i bambini più o meno avvantaggiati, e che tale decisione solitamente viene fatta usando i test standardizzati.



%--------------------------------------------------------------------------------
\section{Test riferiti al criterio}
%--------------------------------------------------------------------------------

La valutazione basata sul criterio stabilisce se il rispondente risponde bene o male ad un test non in riferimento ad un gruppo, ma in assoluto.
     Il criterio è il dominio della materia la cui conoscenza viene valutata dal test.
     La maggior parte dei test sottoposti agli studenti possono essere considerati come valutazioni basate sul criterio.
Ad esempio, un test basato sul criterio può mostrare che un particolare bambino è in grado di eseguire correttamente alcune operazioni artimetiche (es., somma, sottrazione, moltiplicazione), ma ha difficoltà con altre (es., divisione).
   Non è necessario usare i risultati del test per un confronto tra le prestazioni di un bambino e quelle dei membri della sua classe, ad esempio.
   Invece, i risultati del test possono essere usati per mettere in evidenza i particolari bisogni educativi del bambino esaminato, in maniera da procedere ad un intervento in quella direzione.
   Quindi, i test riferiti al criterio possono essere usati a scopo diagnostico, ovvero per identificare i problemi su cui intervenire.



%
%%--------------------------------------------------------------------------------
%
%\begin{frame}{Esercizio}
%
%Se i punteggi di un test si distribuiscono in maniera gaussiana con media 70 e deviazione standard 15, si trovino i punteggi $z$, T, Sten e stanine in corrispondenza del punteggio grezzo pari a 55.
%
%\end{frame}
%
%
%%--------------------------------------------------------------------------------
%
%\begin{frame}{Punteggio $z$}
%
%
%\begin{align*}
%z &= \frac{X - \mu}{\sigma}\\[11pt]
%  &= \frac{55 - 70}{15}\\[11pt]
%  &= -1
%\end{align*}
%
%\end{frame}
%
%
%%--------------------------------------------------------------------------------
%
%\begin{frame}{Punteggio T}
%
%
%\begin{align*}
%T &= 50 + 10z\\[11pt]
%  &= 50 + 10 (-1)\\[11pt]
%  &= 40
%\end{align*}
%
%
%\end{frame}
%
%
%%--------------------------------------------------------------------------------
%
%\begin{frame}{Punteggio Sten}
%
%
%\begin{align*}
%\text{Sten} &= 5.5 + 2z\\[11pt]
%  &= 5.5 + 2 (-1)\\[11pt]
%  &= 3.5
%\end{align*}
%
%
%\end{frame}
%
%
%%--------------------------------------------------------------------------------
%
%\begin{frame}{Punteggio stanino}
%
%
%\begin{align*}
%\text{Punteggio stanino} &= 5 + 2z\\[11pt]
%  &= 5 + 2 (-1)\\[11pt]
%  &= 3
%\end{align*}
%
%
%\end{frame}
%
%
%%--------------------------------------------------------------------------------
%
%\begin{frame}{Interpretazione}
%
%Un punteggio grezzo di 55 in una distribuzione gaussiana con $\mu$ = 70 e $\sigma$ = 15 corrisponde ad un punteggio $z$ = -1, ad un punteggio T = 40, ad un punteggio Sten = 3.5 e ad un punteggio stanino = 3.
%
%\end{frame}
%
%
%
%%--------------------------------------------------------------------------------
%
%\begin{frame}{Relazione tra diverse forme di standardizzazione}
%
%\begin{figure}
%\begin{center}
%%\includegraphics[width=11cm]{relscores.pdf}
%\includegraphics[width=11cm]{figure2-8}
%\end{center}
%\end{figure}
%
%\end{frame}
%
%
%%--------------------------------------------------------------------------------
%
%\begin{frame}{Illustrazione}
%
%\begin{itemize}
%\item Most intelligence tests are transformed to have a mean of 100 and a standard deviation of 15.
%\item Thus, an IQ score of 115 is one standard deviation above the mean and an IQ score of 130 is two standard deviations above the mean.
%\item Using the information we have reviewed, you can determine that an IQ score of 115 is approximately in the 84th percentile, while an IQ score of 85 is approximately in the 16th percentile.
%\item Only some 0.13\% of the population obtains an IQ score of 145, which is three standard deviations above the mean.
%\end{itemize}
%
%\end{frame}
%
%
%%--------------------------------------------------------------------------------
%
%\begin{frame}{Illustrazione}
%
%\begin{itemize}
%\item Figure 2.8 shows the standard normal distribution with the Z scores, T scores, IQ scores, and stanines.
%\item Examining the figure, locate the point that is one standard deviation above the mean.
%\item That point is associated with a Z score of 1.0, a T score of 60, an IQ score of 115, and the seventh stanine.
%\item Using the figure, try to find the score on each scale for an observation that falls two standard deviations below the mean.
%\item You should get a Z score of -2.0, a T score of 30, an IQ score of 70, and a stanine of 1.
%\end{itemize}
%
%\end{frame}



