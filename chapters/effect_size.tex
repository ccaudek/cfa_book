\chapter{La dimensione dell'effetto}
\label{ch:effect_size}


\section{La dimensione dell'effetto del trattamento}

I risultati degli studi di ricerca sull'\emph{efficacy} e sull'\emph{effectiveness} degli interventi psicologici usano una metrica comune chiamata \emph{dimensione dell'effetto}.
La dimensione dell'effetto può essere calcolata per quasi tutti i tipi di disegni della ricerca e nel caso di quasi tutte le analisi statistiche. 
Le analisi statistiche basate sui confronti tra gruppi danno luogo a due tipi di indici che misurano la dimensione dell'effetto. 
Il primo tipo di dimensione dell'effetto riguarda le differenze tra le medie dei gruppi e si calcola nei termini della differenza tra le medie dei gruppi (per esempio gruppi con il trattamento e senza trattamento) divisa per la stima della deviazione standard raggruppata. 
Tali indici vanno sotto il nome di $d$, $g$ o $\eta^2$ -- si veda la~\ref{sec:cont_reponse_eff_size}. 

Il secondo tipo di dimensione dell'effetto comporta un confronto di gruppi in termini degli odds o della probabilità di un risultato -- si veda la~\ref{sec:binary_reponse_eff_size}.
Un \emph{odds ratio} (OR) viene calcolato per determinare l'associazione tra la condizione di gruppo (ad es., trattamento e non trattamento) e una variabile ad esito binario (ad esempio, la presenza o la non insorgenza di un evento, come la ricaduta). 
Un \emph{rischio relativo} (RR) viene calcolato per confrontare la probabilità che un evento si verifichi in base ai due livelli del fattore di rischio considerato (ad es., trattamento/non trattamento, esposizione/non esposizione ad un fattore di rischio).
%Sebbene comunemente usati nella ricerca epidemiologica, questi due indici della dimensione dell'effetto sono meno frequentemente utilizzati nella ricerca psicologica. 

Si noti inoltre un punto importante: non è possibile confrontare direttamente la dimensione dell'effetto ottenuta in studi sull'\emph{efficacy} e sull'\emph{effectiveness} del trattamento perché, quasi sempre, tali studi vengono svolti mediante disegni sperimentali molto diversi.
Gli studi sull'\emph{efficacy} sono generalmente studi randomizzati controllati nei quali i risultati del trattamento vengono confrontati con i risultati di una condizione di controllo (ad esempio, nessun trattamento o una forma alternativa di trattamento). 
In questo caso, la dimensione dell'effetto si basa sul cambiamento che può essere attribuito all'effetto causale del trattamento. 
Pochi studi sull'\emph{effectiveness} del trattamento sono invece studi randomizzati controllati: tali studi consistono tipicamente in un'indagine entro i gruppi -- cioè, in un confronto tra la condizione pre-trattamento e la condizione post-trattamento, senza alcuna condizione di controllo. 
Di conseguenza, la dimensione dell'effetto ottenuta in questo tipo di indagini è basata su un cambiamento dovuto a cause molteplici: oltre agli effetti del trattamento, ci sono agli effetti dovuti alla maturazione, alla regressione verso la media, alla remissione spontanea dei sintomi dovuta al passaggio di tempo e la reattività delle misure. 
Pertanto, è probabile che gli studi sull'\emph{efficacy} portino ad una stima della dimensione dell'effetto maggiore rispetto ai valori che tipicamente vengono ottenuti negli studi sull'\emph{effectiveness} del trattamento.


\subsection{Risposta continua}
\label{sec:cont_reponse_eff_size}

\subsubsection{Indici $d$ di Cohen e $g$ di Hedges}

Per valutare la dimensione dell'effetto, Cohen (1962, 1988) ha introdotto una misura simile a un punteggio $z$ in cui una di due medie campionarie viene sottratta dall'altra e il risultato è diviso per la deviazione standard della popolazione:
\begin{equation}
d = \frac{M_A - M_B}{\sigma},
\label{eq:d_cohen}
\end{equation}
laddove $M_A$ e $M_B$ sono le due medie campionarie e $\sigma$ è la media della popolazione.

Hedges (1982) ha proposto una piccola modifica alla~\eqref{eq:d_cohen} nella quale la deviazione standard raggruppata sostituisce il parametro ignoto $\sigma$, ottenendo in questo modo la statistica $g$:
\begin{equation}
g = \frac{M_A - M_B}{s}.
\label{eq:g_eff_size}
\end{equation}

Al fine di evitare una sistematica sovrastima della dimensione dell'effetto in piccoli campioni, Borenstein, Hedges, Higgins e Rothstein (2009) hanno proposto la seguente correzione:
\begin{equation}
d_{unb} = d \left(1 - \frac{3}{4 \cdot \text{df} - 1} \right).
\label{eq:d_unbiased}
\end{equation}
La correzione è molto piccola quando l'ampiezza campionaria è grande (solo il 3\% per 25 gradi di libertà) ma è sostanziale per piccoli campioni.

\subsection{Risposta binaria}
\label{sec:binary_reponse_eff_size}

Supponiamo che la variabile risposta  $Y$ abbia due modalità, convenzionalmente chiamate successo ($1$) e insuccesso ($0$).
Per esempio, un individuo con certi fattori di rischio può ammalarsi oppure no.
Supponiamo che le osservazioni siano indipendenti e che $P(Y=1) = \pi$ e $P(Y=0) = 1-\pi$.
Spesso, ogni unità di osservazione è associata a un vettore ($X_1, \dots, X_p$) di variabili esplicative. 
L'obbiettivo è studiare la relazione tra la probabilità $P(Y=1)$ e le variabili
esplicative.
Il caso più semplice si ha quando esiste un solo carattere esplicativo $X$, con $I$  modalità sconnesse (un fattore). 
In tal caso i dati binari sono raggruppati per modalità della variabile esplicativa e le risposte
$
Y_1, \dots, Y_I
$
sono il \emph{numero di successi} nelle $n_i$ prove indipendenti dell'$i$-esimo gruppo ($i = 1, \dots, I$).
Se i dati sono raggruppati si possono  presentare come una tavola di contingenza $I \times 2$:

\begin{center}
\begin{tabular}{cccc}
  \hline
  % after \\: \hline or \cline{col1-col2} \cline{col3-col4} ...
    & Successo & Insuccesso & Totale \\
  \hline
  1 & $Y_{11}$ & $Y_{12}$ & $Y_{1+}$ \\
  2 & $Y_{21}$ & $Y_{22}$ & $Y_{2+}$ \\
  $\vdots$ & $\vdots$ & $\vdots$ & $\vdots$ \\
  I & $Y_{I1}$ & $Y_{I2}$ & $Y_{I+}$ \\
  \hline
\end{tabular}
\end{center}
Nelle applicazioni che considereremo i totali di riga non sono variabili aleatorie, ma costanti fisse per disegno. 
Pertanto
\begin{center}
\begin{tabular}{cccc}
  \hline
  % after \\: \hline or \cline{col1-col2} \cline{col3-col4} ...
  X & Successo & Insuccesso & Totale \\
  \hline
  $x_1$    & $Y_{1}$ & $n_1-Y_1$ & $n_1$ \\
  $x_2$    & $Y_{2}$ & $n_2-Y_2$ & $n_2$ \\
  $\vdots$ & $\vdots$ & $\vdots$ & $\vdots$ \\
  $x_I$    & $Y_{I}$ & $n_I-Y_{I}$ & $n_I$ \\
  \hline
\end{tabular}
\end{center}
Le situazioni in cui i totali di riga sono v.a. non saranno qui considerate.

Il caso fondamentale è quello della tavola $2 \times 2$ in cui vi è una sola variabile esplicativa binaria.
Supponiamo che $X$ sia il trattamento (1: presente; 0: controllo) e $Y$ sia la risposta (1: successo; 0: insuccesso).
La situazione è riassunta nella tavola seguente:

\begin{center}
\begin{tabular}{cccc}
  \hline
  % after \\: \hline or \cline{col1-col2} \cline{col3-col4} ...
    & \multicolumn{2}{c}{Risposta} &   \\
  \hline
Trattamento & Successo & Insuccesso  \\\hline
Presente & $\pi_1$ & $1-\pi_1$  \\
Controllo & $\pi_2$ & $1-\pi_2$  \\\hline
\end{tabular}
\end{center}
dove $\pi_1$ e $\pi_2$ sono le probabilità condizionate $P(Y = \text{successo} \mid X=\text{trattamento presente})$ e $P(Y = \text{successo} \mid X=\text{controllo})$.
Le  proporzioni di successi nel campione forniscono delle stime di $\pi_1$ e $\pi_2$:
\begin{align}
\hat{\pi}_1 &= Y_1/n_1,\notag\\
\hat{\pi}_2 & = Y_2/n_2,\notag
\end{align}
dove $Y_i$ è il numero ottenuto di successi ed $n_i$ è la numerosità delle prove per $X = x_i$.
Per studiare l'\emph{efficacy} del trattamento si calcolano i seguenti indici:
\begin{itemize}
    \item la \emph{differenza delle probabilità} $D=\pi_{1} -
    \pi_{2}$;
    \item il \emph{rapporto delle probabilità}
    $RR=\pi_{1} / \pi_{2}$, il cosiddetto \emph{rischio relativo};
    \item il \emph{rapporto delle quote} (odds-ratio) detto anche \emph{rapporto
    crociato} $\theta = \frac{\pi_1/(1-\pi_1)}{\pi_2/ (1-\pi_2)}$;
    \item il \emph{logaritmo del rapporto delle quote} $\log_e \theta$.
\end{itemize}

\begin{exmp}
Il Physicians' Health Study è stata un'indagine prospettiva svolta per verificare se l'uso regolare di Aspirina riduce la mortalità per malattie cardiovascolari. 
I partecipanti allo studio (dei medici volontari) venivano assegnati in modo casuale al trattamento (uso regolare di Aspirina, $n_1=11037$) o al placebo ($n_2=11034$). I soggetti non erano a conoscenza del tipo di trattamento cui erano stati assegnati.
\begin{lstlisting}
aspirina
#>           Infarto miocardico
#> Gruppo      Si    No
#>   Aspirina 104 10933
#>   Placebo  189 10845
\end{lstlisting}
L'evento chiamato convenzionalmente successo è la presenza di un infarto.
La tabella precedente è stata generata con le seguenti istruzioni \R:
\begin{lstlisting}
aspirina <- matrix(
  c(104,10933,189,10845),
  nrow=2, 
  byrow=TRUE
)
dimnames(aspirina) <- list(
  c("Aspirina","Placebo"),
  c("Si","No")
)
names(dimnames(aspirina)) <- 
  c("Gruppo","Infarto miocardico")

aspirina
#>           Infarto miocardico
#> Gruppo      Si    No
#>   Aspirina 104 10933
#>   Placebo  189 10845
\end{lstlisting}
Calcolare la grandezza totale del campione e le proporzioni $\hat{\pi}_{ij}$ è facile:
\begin{lstlisting}
tot <- sum(aspirina)
tot
#> [1] 22071
aspirina/tot
#>           Infarto miocardico
#> Gruppo           Si     No
#>   Aspirina 0.004712 0.4954
#>   Placebo  0.008563 0.4914
\end{lstlisting}
Le stime delle probabilità condizionate di successo dato $X$ (proporzioni di riga) si ottengono nel modo seguente:
\begin{lstlisting}
rowtot <- apply(aspirina, 1, sum)
rowtot
#> Aspirina  Placebo
#>    11037    11034
rowprop <- sweep(aspirina, 1, rowtot, "/")
rowprop
#>           Infarto miocardico
#> Gruppo           Si     No
#>   Aspirina 0.009423 0.9906
#>   Placebo  0.017129 0.9829
\end{lstlisting}
quindi $\hat{\pi}_1=0.0094$ e $\hat{\pi}_2=0.0171$.
\end{exmp}


\subsubsection{Differenza tra due proporzioni}

Il modo più semplice per misurare l'effetto del trattamento sulla variabile risposta è la differenza delle probabilità
\[
D =\pi_1 - \pi_2.
\]
La differenza di probabilità assume valori compresi tra $-1$ e $1$ ed è nulla se la risposta non dipende da $X$, cioè se il trattamento non ha effetto.

\begin{exmp}
Per l'esempio dell'aspirina, $\hat{\pi}_1=104/11037 = 0.0094$ è la proporzione di attacchi di cuore (successi) tra gli individui trattati con aspirina e $\hat{\pi}_2=189/11034 = 0.0171$
è la proporzione di attacchi di cuore  tra gli individui trattati
con il placebo.
La differenza tra le proporzioni è $\hat{\pi}_1 - \hat{\pi}_2 =  0.0094 - 0.0171  = -0.0077$.
L'errore standard stimato della differenza tra le proporzioni è
\[
\sqrt{\frac{\hat{\pi}_1(1-\hat{\pi}_1)}{n_1} +
\frac{\hat{\pi}_2(1-\hat{\pi}_2)}{n_2}} = 0.0015.
\]
Un intervallo di confidenza al 95\% per la differenza tra le proporzioni è $-0.0077 \pm 1.96(0.0015)$ cioè ($-0.011, -0.005$).
Poiché l'intervallo contiene solo valori negativi si conclude che $\pi_1 < \pi_2$, per cui il trattamento con aspirina appare ridurre la probabilità di infarto.

Il risultato precedente si ottiene con \R nel modo seguente:
\begin{lstlisting}
p1 <- 104 / 11037
p2 <- 189 / 11034
n1 <- 11037
n2 <- 11034
se <- sqrt((p1 * (1 - p1) / n1) + (p2 * (1 - p2) / n2))
se
#> [1] 0.00154
p1 - p2 + c(-1, 1) * qnorm(0.975) * se
#> [1] -0.010724 -0.004688
\end{lstlisting}

\end{exmp}

\subsubsection{Rischio relativo}

La differenza tra due proporzioni può rivestire un'importanza maggiore quando entrambe le proporzioni sono vicine a 0 o a 1 che quando esse sono vicine a 0.5.
Per esempio, in uno studio che confronta la mortalità
associata a due trattamenti, la differenza $0.010 - 0.001 = 0.009$
può essere più importante della differenza tra $0.410 - 0.401 =
0.009$.
Infatti, la prima differenza coinvolge due proporzioni che
stanno in rapporto di 10 a 1, mentre la seconda riguarda due
proporzioni quasi uguali. Quindi la prima differenza appare di
maggiore rilevanza.
Nelle situazioni precedenti è utile calcolare il rischio
relativo, cioè il rapporto
\[
RR = \frac{\pi_1}{\pi_2}.
\]
Il $RR$ è sempre un numero non negativo. 
Se è uguale a 1 la risposta non dipende da $X$.

\begin{exmp}
Per le proporzioni precedenti, il rischio relativo è $0.010/0.001=10.0$ e $0.410/0.401=1.02$.
Nell'esempio dell'aspirina il $RR$ campionario è
\[
0.0171/0.0094 = 1.82.
\]
Quindi la proporzione campionaria di casi di infarto è più grande dell'82\% per il gruppo dei trattati col placebo.
\end{exmp}


\subsubsection{Odds ratio}

Invece del $RR$ si può misurare la dipendenza delle probabilità dal trattamento con il rapporto tra le \emph{quote di scommessa} (gli odds).
Gli odds di successo $\omega$ sono per definizione il rapporto tra la probabilità di successo ($\pi$) e la probabilità di insuccesso ($1 - \pi$):
\[
  \omega = \frac{\pi}{1 - \pi}.
\]
La quota di scommessa $\omega$ è un indice non negativo
che misura quanti successi ci si attendono per ogni insuccesso.
Nel campo delle scommesse  quando la quota dell'evento $Y =
    1$ contro $Y = 0$ è $\omega$, scomettendo su $Y = 0$
    si vince $\omega$ volte la posta.
Per esempio, se $\pi = 0.75$ la quota è $\omega = 0.75/0.25 = 3$; scommettendo su $Y = 0$ si riceve, in caso di vincita, 3 volte la posta.
Se la quota è minore di $1$, si può calcolare il reciproco ($1/\omega$) e interpretare il risultato come riferito all'evento complementare.
Per esempio, se $\omega$ è $0.3333$, uno si aspetta $0.3333$ successi per ogni insuccesso ossia
$1/0.3333=3$ insuccessi per ogni successo.
La relazione inversa tra  probabilità di successo e odds
è
$
\pi =\frac{\omega}{1 + \omega}.
$
Per esempio, se $\omega =3$, la probabilità di successo sarà
$
\pi = \frac{3}{1 + 3} 
$
uguale a 0.75.

\begin{exmp}
Nell'esempio dell'aspirina l'odds di infarto per il gruppo aspirina è
\[
\omega_1 = 104/10933 = 0.0095
\]
e per il gruppo placebo è 
\[
\omega_2 = 189/10845 = 0.0174.
\]
Dato l'odds $\omega_1 = 0.0095$, la probabilità di infarto per il gruppo aspirina è
\[
\pi_1 =\frac{\omega_1}{1 + \omega_1} = \frac{0.0095}{1 + 0.0095}= 0.0094
\]
 ovvero
\[
\pi_1 = \frac{104}{104+10933}= 0.0094.
\]
\end{exmp}


\subsubsection{Logit}

Il logaritmo della quota ($\phi$) o \emph{logit} definito da
    \[
    \phi = \log_e \omega = \log_e \frac{\pi}{1 - \pi} = \ln (\pi) - \ln (1-\pi)
    \]
trasforma la probabilità $\pi \in (0, 1)$ in un numero $\phi \in \Re$.
Il logit è simmetrico attorno allo 0 ed è privo di limite superiore e inferiore.

Per esempio, se $\pi = 0.75$ la quota è $\omega = 0.75/0.25 = 3$.
Il logaritmo della quota (logit) è
$
\log_e 3 = 1.0986.
$
Vediamo di seguito alcuni valori rappresentativi che illustrano la realzione tra $\pi$ e $\phi$:
\begin{center}
\begin{tabular}{clc}
  \hline
  % after \\: \hline or \cline{col1-col2} \cline{col3-col4} ...
  Probabilità & Odds & logit \\
  $\pi$ & $\omega= \frac{\pi}{1-\pi}$ & $ \phi = \ln \frac{\pi}{1-\pi}$ \\
 \hline
  0.01 & 1/99=0.0101 & -4.60 \\[5pt]
  0.05 & 5/95=0.0526 & -2.94 \\[5pt]
  0.10 & 1/9=0.1111 & -2.20 \\[5pt]
  0.30 & 3/7=0.4286 & -0.85 \\[5pt]
  0.50 & 5/5=1 & 0.00 \\[5pt]
  0.70 & 7/3 = 2.333 & 0.85 \\[5pt]
  0.90 & 9/1 = 9 & 2.20 \\[5pt]
  0.95 & 95/5 = 19 & 2.94 \\[5pt]
  0.99 & 99/1 = 99 & 4.60 \\[5pt]
  \hline
\end{tabular}
\end{center}

La trasformazione inversa del logit è
\[
\pi = \frac{e^{\phi}}{1+e^{\phi}},
\]
dove $e \simeq 2.718$.

A un logit $\phi = 1.0986$, per esempio, corrisponde una probabilità
\[
\pi =
\frac{e^{\phi}}{1+e^{\phi}}=\frac{e^{1.0986}}{1+e^{1.0986}}=0.75
\]
uguale a 0.75.

Si noti che il logit della probabilità complementare $1 -\pi = 0.25$ è -1.0986, cioè l'opposto.
Se $\pi = 0.25$ la quota è $\omega = 0.25/0.75 = 0.3333$.
Il logit di $\pi = 0.25$ è
\[
\log_e (0.25/0.75) = \log_e 0.3333 = -1.0986.
\]

La distribuzione campionaria della proporzione $\hat{\pi}$ è esattamente una Binomiale.
Asintoticamente, è normale con media $\pi$ e varianza asintotica stimata $\hat{\pi}(1 - \hat{\pi})/n$.
Lo stimatore del logit ha una distribuzione asintotica normale con media $\pi/(1 -\pi)$ e varianza asintotica stimata $1/(n \hat{\pi}(1 - \hat{\pi}))$.

\begin{exmp}
Per il campione di 11037 volontari sottoposti al trattamento aspirina la probabilità stimata di infarto miocardico è
\[
\hat{\pi} = 104/11037 = 0.0094
\]
Il logit empirico è
\[
ln \left( \frac{0.0094}{1- 0.0094} \right) =-4.655
\]
con errore standard asintotico
\[
\sqrt{\frac{1}{11037 \times 0.0094 \times (1-0.0094)}}= 0.0985.
\]
\end{exmp}


\subsubsection{Rapporto delle quote}

Una volta chiarito il concetto di odds, consideriamo ora il \emph{rapporto delle quote} (\emph{odds-ratio}) detto anche \emph{rapporto crociato} (\emph{cross-product ratio}).
In una tabella $2 \times 2$ gli odds di successo nella riga $i$-esima sono $\omega_i = \pi_i / (1-\pi_i)$.
Il rapporto degli odds $\omega_1$ e $\omega_2$ nelle due righe
    \[
     \theta = \frac{\omega_1}{\omega_2}=\frac{\pi_1 / (1-\pi_1)}{\pi_2 / (1-\pi_2)}
    \]
è chiamato rapporto delle quote.

Un rapporto delle quote può assumere solo valori non negativi.
Se è uguale a 1 gli odds sono uguali e quindi sono uguali anche le probabilità, cioè la risposta è indipendente dal trattamento. 
I rapporti degli odds si valutano in rapporto a 1:
\begin{itemize}
    \item se $1 < \theta < \infty$ gli odds sono più grandi nel gruppo 1
che nel gruppo 2, e quindi anche $\pi_1 > \pi_2$;
    \item se $0 <  \theta
< 1$ gli odds sono più piccoli nel gruppo 1 che nel gruppo 2 e
$\pi_1 < \pi_2$.
\end{itemize}
Il rapporto degli odds non cambia se si permutano la variabile risposta e la variabile esplicativa; quindi $\theta$ tratta le variabili in modo simmetrico.

\begin{exmp}
Nell'esempio dell'aspirina il rapporto degli odds stimato è
\[
\hat{\theta} = \frac{189/10845}{104/10933}= 1.832,
\]
cioè gli odds a favore dell'infarto sono più grandi dell'83\% per il gruppo placebo.
Un rapporto degli odds di $1.832$ non significa che $\pi_2$ è 1.832 volte $\pi_1$, ma che gli odds $\pi_2/(1 - \pi_2)$ sono 1.832 volte gli odds $\pi_1/(1 - \pi_1)$.
Tuttavia,
\[
 \theta = RR \frac{1-\pi_1}{1-\pi_2}
\]
Perciò quando la proporzione di successi è prossima a zero in entrambi i gruppi $\theta$ e il RR hanno valori simili. 
Si osservi infatti che il RR nell'esempio dell'aspirina è 1.83.
\end{exmp}


\subsubsection{Logaritmo del rapporto delle quote}

La distribuzione del rapporto degli odds è molto asimmetrica ed è conveniente usare la distribuzione del suo logaritmo $\log_e \theta$. 
Tale distribuzione è meno asimmetrica e più vicina alla normalità.
Il logaritmo di $\theta$ è 0 in caso di indipendenza e l'interpretazione è simmetrica rispetto allo zero. 
Cioè se si invertono le righe o le colonne della tavola $\log_e \theta$ cambia il segno.
Due valori di $\log_e \theta$ diversi solo per il segno rappresentano due livelli di associazione uguali. 
Raddoppiando $\log_e \theta$ corrisponde ad elevare al quadrato il rapporto delle quote.

L'errore standard asintotico del logaritmo del rapporto degli odds ha una formula semplice
\[
\sqrt{\frac{1}{Y_{11}} + \frac{1}{Y_{12}} + \frac{1}{Y_{21}} +
\frac{1}{Y_{22}}},
\]
dove $Y_{ij}$ sono le frequenze nelle celle della tavola di
contingenza.

Un intervallo di confidenza al 95\% per $\ln \theta$ è dato dalla stima $\ln \hat{\theta} \pm 1.96$ l'errore standard stimato.
Per ottenere l'intervallo di confidenza al 95\% per l'odds ratio esponenziamo i limiti $L$ dell'intervallo di confidenza:
$e^L$.

\begin{exmp}
Nell'esempio dell'aspirina il rapporto degli odds è
\[
\hat{\theta} = \frac{189/10845}{104/10933}= 1.832
\]
e il logaritmo del rapporto degli odds è
\[
\log_e 1.832 = 0.605.
\]

Il logaritmo del rapporto delle quote ha la seguente proprietà: se invertiamo l'ordine delle categorie di una delle variabili, il logaritmo del rapporto delle quote cambia semplicemente di segno:
\[
\hat{\theta} = \frac{104/10933}{189/10845}= 0.5458
\]
e quindi
\[
\log_e 0.5458 = -0.605.
\]

L'errore standard stimato del logaritmo dell'odds ratio è
\[
\sqrt{\frac{1}{189} + \frac{1}{10845} + \frac{1}{104} + \frac{1}{10933}} = 0.1228.
\]
Un intervallo di confidenza al 95\% per $\log_e \theta$ è
\begin{lstlisting}
0.6054377 + c(-1, 1)* 1.96 * se
#> [1] 0.3646681 0.8462073
\end{lstlisting}
e il corrispondente intervallo di confidenza per $\theta$ è
\begin{lstlisting}
exp(0.8462073)
#> [1] 2.33079
exp(0.3646681)
#> [1] 1.440036
\end{lstlisting}
Poiché non contiene il valore 1, i veri valori degli odds per il gruppo placebo e per il gruppo aspirina sono significativamente diversi: gli odds per l'infarto sono almeno il 44\% in più rispetto al gruppo aspirina.

\end{exmp}


\subsection{Tipi fondamentali di indagine}

Si distinguono due tipi fondamentali di indagine: gli esperimenti e gli studi osservazionali.
Negli esperimenti si studia l'effetto di uno o più trattamenti sulle risposte delle unità sperimentali.
Negli esperimenti randomizzati ogni unità sperimentale viene assegnata casualmente dal ricercatore a una delle possibili modalità di trattamento.
L'iportanza della randomizzazione consiste nel produrre dei sottogruppi a seconda dei livelli del trattamento, in cui tutte le altre variabili, anche quelle non misurate, hanno approssimativamente la stessa distribuzione e quindi sono comparabili.

Negli studi osservazionali invece il ricercatore non può assegnare le unità ai trattamenti. 
Le indagini osservazionali si possono distinguere in studi longitudinali prospettici,  trasversali (cross-sectional) e caso-controllo.

\subsubsection{Studi prospettici}

In un studio prospettico viene seguito un numero fisso di unità per ciascuna modalità della variabile esplicativa ($X = x_1, \dots, X = x_I$) e dopo un periodo prefissato si rilevano le proporzioni di successi negli $I$ gruppi.

\begin{exmp}
Il Physicians' Health Study è stata un'indagine prospettica svolta per verificare se l'uso regolare di Aspirina riduce la mortalità per malattie cardiovascolari.
I partecipanti allo studio (dei medici volontari) venivano assegnati in modo casuale al trattamento (uso regolare di Aspirina, $n_1=11037$) o al placebo ($n_2=11034$).
Per cinque anni, i medici che parteciparono allo studio assunsero giornalmente una pastiglia di aspirina o un placebo.
Alla fine dello studio l'incidenza di infarti miocardici venne misurata nei due gruppi.
\end{exmp}


\subsubsection{Studi trasversali}

Se si estrae un campione ad un tempo prefissato e si classificano le unità nella tabella $I \times 2$ a seconda delle modalità dei due caratteri si ha uno studio trasversale.

\begin{exmp}
Sulla base del General Social Survey (1984), 901 individui sono stati classificati in base alla soddisfazione lavorativa (soddisfatto verso insoddisfatto) e a due categorie di reddito ($< \$ 15.000, \geq \$ 15.000$).
\begin{center}
\begin{tabular}{ccc}
  \hline
  % after \\: \hline or \cline{col1-col2} \cline{col3-col4} ...
   & Soddisfatto & Insoddisfatto \\
  \hline
 $< \$ 15.000$      & 391 & 104 \\
  $\geq \$ 15.000$ & 340 & 66 \\
  \hline
\end{tabular}
\end{center}
Ci si chiede: la soddisfazione lavorativa dipende dal reddito?
\end{exmp}


\subsubsection{Studi longitudinali retrospettivi}

Se si estraggono due campioni per $Y = 0$ e $Y = 1$ e si controlla nel passato se i soggetti appartengono al gruppo $x_1, x_2, \dots, x_I$ allora si ha un studio retrospettivo.
Un esempio tipico di indagine basata su un disegno retrospettivo è lo studio caso-controllo. 
Si considerino i dati seguenti:
\begin{center}
\begin{tabular}{rcc}
  \hline
  % after \\: \hline or \cline{col1-col2} \cline{col3-col4} ...
    & \multicolumn{2}{c}{Cancro ai polmoni}  \\
  \hline
  Mai fumato? & Casi & Controlli \\
 S{\`\i} & 688 & 650 \\
 No & 21 & 59 \\\hline
  Totale & 709 & 709 \\
  \hline
\end{tabular}
\end{center}
La prima colonna si riferisce a $709$ pazienti ricoverati in
20 ospedali londinesi per cancro polmonare.
Ogni caso è stato appaiato a un controllo, cioè a un
paziente ricoverato nello stesso ospedale per disturbi diversi dal
cancro polomonare.
Casi e controlli sono stati classificati poi a seconda che
siano o siano stati fumatori oppure no.

Sarebbe naturale considerare il cancro polmonare come variabile risposta  e il comportamento relativo al fumo quale variabile esplicativa, e confrontare le proporzioni di cancro
polmonare tra fumatori e non fumatori, $P(cancro \mid fumo)$.
In questo studio, tuttavia, questo non ha senso dato che la distribuzione marginale della variabile risposta è fissa per disegno.
\begin{itemize}
    \item Nello studio Aspirina erano i totali di riga a essere fissi -- avevamo cioè un campione indipendente per ciascuna modalità della variabile esplicativa.
    \item Nel caso presente, invece, sono i totali di colonna ad essere fissi -- ovvero, abbiamo un campione indipendente per ciascuna modalità della variabile risposta.
\end{itemize}
Invece è sensato usare le proporzioni nell'altro verso, cioè $688/709 = 0.970$ e $650/709=0.917$ come stime delle probabilità $P(X = fumatore \mid Y = caso)$ e $P(X = fumatore \mid Y = controllo)$.
Se conoscessimo la proporzione di individui nella popolazione con cancro polmonare, potremmo stimare
$P(Y = \text{cancro polmonare} \mid X = \text{fumatore})$ e $P(Y = \text{non cancro polmonare} \mid X = \text{fumatore})$.
Ponendo $C$ = cancro e $F$ = fumatore, con Bayes avremo
\[
P(C \mid F)=\frac{P(C)P(F \mid C)}{P(C)P(F \mid C) + P(C^c)P(F \mid C^c)}.
\]
Non possiamo però procedere in questo modo, dato che la probabilità $P(C)$ è ignota.
Con questo disegno, inoltre, dalle probabilità $P( X = \text{fumatore} \mid Y = \text{cancro polmonare})$ e $P(X = \text{fumatore} \mid Y = \text{non cancro polmonare})$ non è neppure possibile  calcolare  il rischio relativo per la risposta $Y$.

Possiamo però stimare il rapporto degli odds
\[
\frac{(688/709)/(21/709)}{(650/709)/(59/709)}=\frac{688 \times
59}{650 \times 21}=  2.97
\]
e utilizzarlo per l'interpretazione di interesse (anche se lo studio è retrospettivo): gli odds per il cancro polmonare sono circa 3 volte più grandi tra i fumatori.

Se le probabilità condizionate di $Y = \text{cancro polmonare}$ dato $X = \text{fumatore}$ e $X = \text{non fumatore}$ sono vicine a zero, il $RR$ ha un valore simile a quello di
$\theta$.
Nell'esempio possiamo attenderci che tali probabilità siano piccole e quindi possiamo considerare il rapporto degli odds come un indicatore grezzo del $RR$.
Possiamo perciò concludere dicendo che la frequenza relativa del cancro polmonare è circa 3 volte più grande per gli individui che hanno fumato rispetto a quelli che non hanno mai
fumato.

Un intervallo di confidenza si calcola come indicato in precedenza.
\begin{lstlisting}
se <- sqrt(1 / 688 + 1 / 650 + 1 / 21 + 1 / 59)
se
#> [1] 0.2599234
theta <- (688 * 59) / (650 * 21)
theta
#> [1] 2.973773
log(theta)
#> [1] 1.089831
log(theta) + c(-1, 1)* 1.96 * se
#> [1] 0.5803817 1.5992813
exp(0.5803817)
#> [1] 1.786720
exp(1.5992813)
#> [1] 4.949474
\end{lstlisting}
Poiché non contiene il valore 1, i veri valori degli odds per il gruppo fumatori e per il gruppo non fumatori sono significativamente diversi: gli odds per il cancro ai polmoni sono almeno il 78\% in più rispetto al gruppo non fumatori.


\subsection*{Considerazioni conclusive}

Nel caso di una variabile risposta binaria e una variabile esplicativa con due sole modalità i dati si possono rappresentare in una tavola di contingenza $2 \times 2$.
Negli studi sperimentali l'efficacia del trattamento si può stabilire calcolando
la differenza tra le proporzioni,
il rapporto tra le proporzioni,
il rapporto tra le quote e
il logaritmo del rapporto tra le quote.
I dati raccolti mediante un esperimento e mediante un'indagine osservazionale prospettico o trasversale hanno la stessa struttura. 
Le differenze nella risposta possono dunque essere analizzate mediante gli stessi indici descritti in precedenza.
Negli studi osservazionali, però, le conclusioni sono molto meno stringenti.
Nell'esperimento i gruppi sono in tutto confrontabili tranne per la modalità del trattamento e quindi eventuali differenze nella risposta non possono essere dovute ad altro che al trattamento.
Negli studi osservazionali, invece, la mancanza di controllo sull'assegnazione dei trattamenti fa sì che  i gruppi di unità non siano mai totalmente comparabili.
Negli studi retrospettivi, come ad esempio i disegni caso-controllo, non ha senso calcolare la differenza tra le proporzioni condizionate alla variabile esplicativa e il rischio relativo.
Possiamo però calcolare il rapporto tra le quote e, per la proprietà simmetrica di questo indice ($\theta$ non cambia se vengono invertite la variabile sulle righe e quella sulle colonne), procedere poi all'interpretazione nella direzione di interesse.
Dato che il logaritmo dell'odds ratio ha una distribuzione asintotica normale, un intervallo di confidenza per $\ln \theta$ può essere facilmente calcolato.
Esponenziando i limiti dell'intervallo di confidenza per $\ln \theta$ si trovano i limiti dell'intervallo di confidenza per $\theta$.




