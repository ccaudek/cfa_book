%%%%%%%%%%%%%%%%%%%%%% pref_it.tex %%%%%%%%%%%%%%%%%%%%%%%%%%%%%%%%%%%%%
%
% Esempio di prefazione
%
% Usare questo file come template per il vostro documento.
%
%%%%%%%%%%%%%%%%%%%%%%%% Springer-Verlag %%%%%%%%%%%%%%%%%%%%%%%%%%

%-----------------------------------------------------------------------------
% Beginning of preface.tex
%-----------------------------------------------------------------------------


\chapter*{Prefazione}
\label{chapter:prefazione}

%% Scrivere qui la prefazione

Queste dispense contengono il materiale delle lezioni dell'insegnamento \emph{Costruzione e validazione di strumenti di misura dell'efficacia dell'intervento psicologico in neuropsicologia} - B020881 (B213) rivolto agli studenti del secondo anno del Corso di Laurea Magistrale in Psicologia Clinica e della Salute e Neuropsicologia dell'Università di Firenze.

Le dispense includono la discussione degli argomenti della parte centrale del corso, ovvero quella relativa alla costruzione e validazione di strumenti di misura psicologici e neuropsicologici.  
Per quel che riguarda la discussione dell'assessment neuropsicologico, si rimanda a: M.D. Lezak, D.B. Howieson, E.D. Bigler, D. Tranel (2004). \emph{Neuropsychological Assessment} (fifth edition). New York: Oxford University press.

%%Queste dispense rappresentano gli appunti relativi al corso Psicometria dell'A.A. 2016/2017. 
%Sono pensate, come l'insegnamento del resto, per studenti in possesso di conoscenze matematiche quali quelle fornite mediamente dalle scuole secondarie 
%superiori. Alcuni richiami del linguaggio insiemistico sono fatti nelle Appendici che consiglio di leggere come prima cosa. Le Appendici forniscono inoltre un breve ripasso di alcune nozioni di matematica di base, di un po' di simbologia, e introducono anche qualche argomento più avanzato.
%
%
%%Un certo numero di esercizi sono forniti in queste dispense e sulla piattaforma Moodle. 
%
%%Queste dispense contengono anche un certo numero di esercizi. Questi problemi rappresentano la difficoltà richiesta per superare l'esame di Psicometria. Le soluzioni degli esercizi sono fornite alla fine delle dispense.
%
%%Per seguire con profitto questo insegnamento è richiesta un'adeguata preparazione iniziale e le conoscenze richieste sono quelle fornite dalla scuola superiore, con particolare riferimento alle conoscenze matematiche di base e alle capacità di ragionamento logico. Le Appendici forniscono un ripasso di alcuni concetti preliminari elementari, di un po' di simbologia altrettanto elementare su cui è importante avere le idee chiare, ma a volte introducono qualche argomento più avanzato.
%
%Uno degli obiettivi centrali dell'insegnamento di Psicometria è quello di mostrare allo studente come si applicano e si utilizzano le tecniche statistiche di base per l'analisi dei dati psicologici. A tale scopo nelle esercitazioni incorporate nelle lezioni verrà usato il software open source \R. Un'introduzione che consente di farsi un'idea generale di come utilizzeremo il software \R\, per calcolare statistiche descrittive e inferenziali univariate e bivariate e per la loro presentazione grafica è fornita nel primo capitolo delle dispense.
%
%Il corso si propone di rendere accessibili agli studenti i concetti di base dell'analisi dei dati e la relazione tra questi e la metodologia della ricerca psicologica. Allo studente non viene chiesto di memorizzare equazioni e formule, né di imparare a risolvere in maniera meccanica problemi scolastici. L'obiettivo è invece quello di contribuire a sviluppare un modo di pensare computazionale e algoritmico che può essere utilizzato per risolvere il tipo di problemi che si incontrano nel mondo reale, sia nella ricerca psicologica sia nella pratica della professione dello psicologo.
%
%Il giusto metodo di studio per prepararsi all'esame di Psicometria è quello di seguire attivamente le lezioni, assimilare i concetti via via che essi vengono presentati, controllare e verificare in autonomia le procedure presentate a lezione, fare domande dimostrando la capacità di mettere in relazione tra loro i diversi argomenti trattati, rivolgersi al docente durante il ricevimento per chiarire ciò che non si è capito appieno, partecipare attivamente alle esercitazioni organizzate dai Peer Tutor e ai forum attivi su Moodle e, soprattutto, svolgere gli esercizi proposti nelle dispense e su Moodle. Tali problemi rappresentano la difficoltà richiesta per superare l'esame e consentono allo studente di capire se le competenze sviluppate risultino essere sufficienti rispetto alle richieste dell'insegnamento. 



%Nella mia esperienza didattica nei Corsi di Laurea in Psicologia la cosa che ho trovato più difficile non è ``spiegare le formule,'' ma spiegare a cosa servono, agli psicologi, la misurazione e l'analisi dei dati. Credo fortemente che la risposta a questa domanda possa essere  data solo dal coinvolgimento degli studenti nell'attività di ricerca, in modo tale che essi si rendano conto di come vengono ottenute le osservazioni psicologiche, di come tali osservazioni producano dei ``numeri'' e di come l'analisi di queste misurazioni conduca a delle affermazioni di carattere generale sui fenomeni psicologici. Nessuna descrizione può sostituire un'esperienza in prima persona dell'applicazione del metodo scientifico. 
%
%Il livello di difficoltà degli argomenti qui discussi si richiede agli studenti di possedere una preparazione matematica del livello di quella offerta dalle scuole medie superiori. Tale preparazione, però, non è in generale sufficiente per una comprensione che non sia solo superficiale delle tematiche della statistica applicata all'analisi dei dati psicologici.  A questo limite si può ovviare acquisendo un'adeguata padronanza delle procedure e delle tecniche di analisi e simulazione numerica offerte dai software statistici. In particolare, qui si farà uso del linguaggio di programmazione \textsf{R}, che è Open Source e gratuito (\url{https://www.r-project.org/}). Non è molto utile, infatti conoscere i teoremi della statistica applicata, senza sapere in che modo tali conoscenze possano essere usate concretamente. Per questa ragione, in queste dispense, la discussione di argomenti teorici si accompagnerà a numerosi richiami alle applicazioni pratiche. 
%
%In conclusione, queste dispense intendono offrire alcune informazioni sulla misurazione e sull'analisi dei dati psicologici. Le altre componenti necessarie per   \emph{interpretare} in modo sensato i risultati degli esperimenti andranno cercate altrove.


\vspace{\baselineskip}
\begin{flushright}
\noindent
{\it Corrado Caudek}\\
Firenze, \today\\
\end{flushright}



