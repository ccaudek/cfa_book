% DO NOT COMPILE THIS FILE DIRECTLY!
% This is included by the other .tex files.
%--------------------------------------------------------------------
\chapter{Test psicometrici e analisi fattoriale}
\label{ch:sviluppo}
%--------------------------------------------------------------------

%\begin{chapquote}{R. A. Fisher}
%``The best time to plan an experiment is after you've done it.''
%\end{chapquote}


%------------------------------------------------------------
\section{Utilizzo e costruzione di test psicometrici}
%------------------------------------------------------------

La maggior parte degli psicologi che utilizzano i test non costruiscono dei test ad hoc ma utilizzano i test già validati e interpretano i punteggi attenuti sulla base delle norme fornite nel manuale del test.  Nella selezione del test che è più appropriato per il problema che lo psicologo deve affrontare, si devono considerare domande quali: Qual è il valore minimo di attendibilità che è richiesto? Ovvero, a che livello di precisione dobbiamo essere in grado di differenziare tra i rispondenti? Che tipo di validità è importante?  Quali sono le caratteristiche della popolazione che il gruppo normativo deve rappresentare? Qual è il livello di istruzione necessario per completare il test? Qual è il tempo disponibile per la somministrazione del test e per lo scoring dei risultati? Quali sono i costi necessari per la somministrazione del test e per lo scoring dei risultati? Avendo dato una risposta a tali domande, la selezione del test solitamente si riduce ad una scelta tra poche alternative. Per giungere ad una decisione tra tali alternative è importante consultare la letteratura specialistica che discute le proprietà psicometriche dei test e la loro validità. 

Ciascuno psicologo ha l'obbligo di dimostrare che il test che usa per un certo scopo costituisca lo strumento migliore, tra quelli disponibili, per giungere ad una decisione razionale e obiettiva relativamente al problema che si trova ad affrontare (si veda il Codice Deontologico). Se nessuno dei test disponibili si dimostra appropriato per misurare un determinato tratto psicologico, si procede alla costruzione di un nuovo reattivo. La costruzione di un test richiede sia conoscenze specialistiche di tipo psicometrico, sia una conoscenza specialistica del fenomeno considerato.

Le fasi di costruzione dei test comprendono la definizione delle aree di contenuto che andranno misurate dal test; la generazione degli item per ciascuna area in un numero di circa tre volte superiore a quello che ci si aspetta farà parte della versione finale del test; la somministrazione degli item ad un campione sufficientemente numeroso (centinaia di rispondenti selezionati in modo tale che il campione sia rappresentativo della popolazione di interesse); l’analisi degli item che ci consente di selezionare gli item migliori; infine, la somministrazione della versione revisionata del test ad un nuovo campione per stabilire se la versione finale del test sia soddisfacente dal punto di vista psicometrico. In caso affermativo, il campione esaminato fornisce le norme del test e questa fase va sotto il nome di \emph{standardizzazione del test}.


%------------------------------------------------------------
\section{Caratteristiche dell'analisi fattoriale}
%------------------------------------------------------------

La teoria psicometria è l'unico strumento che abbiamo a disposizione per creare strumenti di misurazione validi e attendibili per l'assessment psicologico e neuropsicologico. 
La psicometria comprende due diversi approcci metodologici per la costruzione dei reattivi psicologici: la teoria classica dei test e la teoria di risposta all'item (\emph{item response theory}). Esamineremo qui gli aspetti di base della teoria classica dei test e ci focalizzeremo, in particolare, sull'analisi fattoriale.

Se vogliamo costruire un test psicometrico per la valutazione di un particolare deficit psicologico o neuropsicologico, prima di iniziare lo studio, dobbiamo fornire una risposta ad una serie di domande. Per esempio, come dobbiamo selezionare gli item? Gli item scelti coprono tutto il dominio del fenomeno considerato? Quanti item dobbiamo usare? A quanti soggetti dobbiamo somministrare lo strumento? Quali analisi statistiche dobbiamo svolgere sui dati raccolti? 

L'analisi fattoriale è uno strumento statistico che può essere utilizzato per trovare una  risposta a queste domande. Chiariamo subito che l'analisi fattoriale è una tecnica statistica complessa che richiede l'uso di un software. Per gli scopi di questo insegnamento useremo il pacchetto \texttt{lavaan} del linguaggio statistico \R.

L'analisi fattoriale può essere pensata come una tecnica statistica per la ricerca di variabili latenti a partire da alcune variabili osservate.  La distinzione tra variabili latenti e variabili osservate si basa sulla osservabilità, ossia sulla possibilità di rilevazione empirica. Le prime sono variabili non direttamente osservabili in quanto rappresentano concetti molto generali o complessi, mentre le seconde sono facilmente rilevabili. In ogni caso, entrambe possono essere operazionalizzate, per cui anche nel caso delle variabili latenti c'è una sostanziale differenza con i concetti. 

Ma che cos'è una variabile latente o fattore? Un fattore può essere descritto come una combinazione lineare di  variabili manifeste tra loro associate le quali rappresentano una specifica dimensione di un costrutto psicologico, la quale si distingue da altre dimensioni dello stesso costrutto o dalle dimensioni di costrutti diversi  (Tabachnick \&
Fidell, 2001). Per esempio, la Wechsler Adult Intelligence Scale – Fourth Edition (WAIS-IV) dà una valutazione complessiva delle capacità cognitive di adolescenti e adulti e distingue tra quattro dimensioni dell'intelligenza: comprensione verbale, ragionamento visuo-percettivo, memoria di lavoro e velocità di elaborazione. 

L'analisi fattoriale viene usata per lo sviluppo e la validazione di un test psicometrico e per stabilire la validità di costrutto di uno strumento per una specifica popolazione. Una volta che la struttura interna di un costrutto è stata chiarita, l'analisi fattoriale  può essere usata per identificare le variabili esterne (per esempio, il genere o il livello di istruzione) che sono associate alle varie dimensioni del costrutto di interesse (Nunnally \& Bernstein, 1994).


%------------------------------------------------------------
\subsection{L'analisi fattoriale esplorativa e confermativa}
%------------------------------------------------------------

Nei suoi primi modelli matematici, l’analisi fattoriale era una tecnica di analisi esplorativa appunto perché permetteva di ``esplorare'' le relazioni nascoste
fra un gran numero di variabili. 
In tempi più recenti, la tecnica delle equazioni strutturali ha permesso di sviluppare una tecnica di analisi fattoriale di tipo confermativo, che cioè permette di verificare se effettivamente i fattori ipotizzati servono a spiegare le variabili misurate. 
In tempi ancora più recenti è stato notato che i vincoli posti dai modelli SEM sono, alle volte, troppo restrittivi per cui le tecniche dell'analisi fattoriale esplorativa vengono integrate con i modelli di equazioni strutturali in maniera tale da produrre un nuovo approccio alla definizione delle variabili latenti, ovvero quelli che si chiamano modelli ESEM (Exploratory Structural Equation Modeling).
 

%------------------------------------------------------------
\subsection{Assunzioni dell'analisi fattoriale}
%------------------------------------------------------------

L'assunzione di base dell'analisi fattoriale esplorativa è che, dato un insieme di variabili osservate, esista un insieme più piccolo di fattori latenti i quali siano in grado di spiegare i legami, le interrelazioni e le dipendenze tra le variabili statistiche osservate. Dato che l'analisi fattoriale si propone di spiegare una matrice di correlazioni, molte delle assunzioni che stanno alla base del calcolo delle correlazioni si applicano anche all'analisi fattoriale: campioni di grande numerosità, il fatto che le variabili considerate abbiano una distribuzione di probabilità continua, la linearità nella relazione tra le variabili. In generale, queste assunzioni si riducono al requisito della normalità multivariata, ovvero il requisito per il quale tutte le variabili considerate e tutte le combinazioni lineari tra tali variabili sono distribuite normalmente.

Molto spesso, però, l'assunzione di normalità multivariata è violata a causa del fatto che, anziché essere delle variabili continue, gli item sono spesso variabili (discrete) che derivano da risposte a questionari fornite utilizzando scale di tipo Likert. 
Vedremo come questo problema può essere superato utilizzando le correlazioni policoriche. 

%------------------------------------------------------------
\subsection{Sviluppo storico dell'analisi fattoriale}
%------------------------------------------------------------

L'analisi fattoriale è un metodo di analisi multivariata che, dal punto di vista storico, risulta fortemente interconnessa con i modelli psicometrici. Alcuni concetti basilari per i successivi sviluppi dell'analisi fattoriale si possono trovare in Galton (1888, 1889) con il concetto di \textit{fonte comune} (\emph{common source}) e in Pearson (1901) con l'introduzione delle \textit{componenti principali} come metodo di riduzione dei dati.
 Si deve però a Spearman (1904) l'elaborazione del primo \textit{modello di analisi fattoriale} secondo il quale le risposte fornite ad un insieme di test di abilità sono riconducibili ad un unico fattore generale di intelligenza.
 
L'analisi fattoriale è stata sviluppata da diversi gruppi di psicologi inglesi e americani che, tra l'inizio del '900 e il 1930, svilupparono una serie di tecniche statistiche per affrontare problemi quali quello di determinare le dimensioni dell'intelligenza (Garnett, 1919; Pearson, 1901; Spearman, 1904, 1923, 1927, 1929; Thurstone, 1935, 1937a, 1937b; Wilson, 1928). 

In Inghilterra, Spearman e i suoi colleghi (Burt, 1939, 1941;
Garnett, 1919; Ledermann, 1937, 1938; Spearman, 1904, 1922, 1923, 1927,
1928, 1929, 1930a, 1930b; Thomson, 1934, 1936, 1938) svilupparono il concetto della teoria dei due fattori dell'intelligenza umana secondo cui esiste un fattore generale, \emph{g}, che esprime l'intelligenza generale, è ereditario e compare in maggiore o minor misura in tutti i test, e esistono inoltre tanti fattori specifici, \emph{s}, che rappresentano l'acquisizione specifica attraverso l'apprendimento e l'esperienza, e sono presenti in  modo differenziato in ogni singolo test. Dato che i fattori specifici sono incorrelati tra loro, il fattore \emph{g} spiega la maggior parte delle correlazioni tra i punteggi delle abilità mentali (Nunnally \& Bernstein, 1994). 
Anche se successive ricerche non confermarono totalmente l'ipotesi di Spearman dell'esistenza di un fattore \emph{g}, e particolarmente vennero alla luce dei fattori di gruppo, che potevano spiegare le correlazioni esistenti tra test simili (Garnett, 1919; Spearman, 1927), la maggior parte dello sviluppo iniziale dell'analisi fattoriale fu dedicata alla ricerca di un singolo fattore per l'intelligenza.

Spearman analizzò molti insiemi di dati nel tentativo di trovare conferma al modello uni-fattoriale: alcuni di questi, tuttavia, non risultarono compatibili con il modello dei due fattori.  La generalizzazione del modello fattoriale al caso in cui sono presenti più fattori comuni si deve a Thurstone (1938, 1947).  Nel modello di Thurstone non viene fatto più riferimento ad un fattore generale che influenza tutte le variabili osservate: si ipotizza invece la presenza di una molteplicità di fattori comuni i quali determinano le variabili osservate. I contributi di Thurstone riguardano alcuni degli aspetti fondamentali dell'analisi fattoriale esplorativa, quali il concetto di struttura semplice e i fattori obliqui.

Thurstone e collaboratori applicarono la teoria dei fattori multipli a vari problemi psicometrici (Thurstone, 1931, 1940, 1947, 1948, 1954) diventando ben presto il riferimento più importante per lo sviluppo delle procedure dell'analisi fattoriale. La centralità della scuola statunitense continuò fino alla fine degli anni '60, quando un numero di importanti psicometristi europei contribuirono a spostare il focus dell'assessment psicologico dall'approccio esplorativo a quello confermativo (J\"oreskog, 1967, 1969, 1970; Jöreskog \& Goldberger, 1972; Sörbom, 1974). I contributi statistici di Jöreskog e Sörbom hanno contribuito a sviluppare quelli che oggi chiamiamo i modelli di equazioni strutturali (Structural Equation Models, SEM), i quali costituiscono una delle tecniche più utilizzate per l'analisi dei dati nelle discipline
psicologiche e sociali. 

Un'ulteriore linea di ricerca venne sviluppata nel campo della biometria, ad opera di Sewall Wright (1934). 
In un tale approccio vengono utilizzati modelli di equazioni simultanee, che includono però solo variabili osservabili, per analizzare particolari schemi di rappresentazione dei nessi di influenza tra le variabili noti come \emph{analisi dei percorsi} (\emph{path analysis}). 

A partire dalla metà degli anni cinquanta, con la diffusione dei centri calcolo, l'analisi fattoriale è diventata uno dei metodi di analisi dei dati più largamente utilizzati in psicologia e, in particolare, nelle ricerche volte alla costruzione dei reattivi psicologici.  
Le tradizioni di ricerca dei modelli di equazioni simultanee con variabili osservate, da una parte, e dei modelli di equazioni strutturali con variabili osservate e latenti, dall'altra, sono rimaste sostanzialmente indipendenti fino agli anni '60 dello scorso secolo, quando metodologi delle scienze sociali, come Blalock (1961, 1963), Boudon (1965) e Duncan (1966) hanno cominciato a sottolineare i vantaggi derivanti dalla possibilità di combinare la semplicità di rappresentare nessi di influenza tra le variabili tramite i diagrammi tipici della \emph{path analysis}, con il rigore derivante dalla specificazione delle equazioni simultanee, in modo tale da includere sia variabili osservate sia variabili latenti. 
La disponibilità di programmi per computer che permettono di tradurre in un linguaggio non matematico le complesse operazioni matematiche connesse alla risoluzione simultanea di sistemi di equazioni lineari con variabili latenti ha poi facilitato enormemente la diffusione delle tecniche dell'analisi fattoriale confermativa. 
A partire dalla fine degli anni sessanta, soprattutto grazie a Karl J\"oreskog, il modello di analisi fattoriale si è ulteriormente raffinato e sviluppato verso un approccio confermativo orientato prevalentemente all'esame di ipotesi teoriche.

Sebbene l'analisi fattoriale confermativa (CFA) possa essere vista come una motivazione per la creazione di modelli di misurazione parsimoniosi, tali modelli spesso includono un certo livello di errore di misurazione sistematico che deriva dal fatto che gli item sono raramente indicatori puri dei corrispondenti costrutti, e quindi ci si può aspettare un certo grado di associazione rilevante per il costrutto anche tra item e costrutti non target, ma concettualmente associati (Morin et al., 2016). 
Quando tali associazioni restano inespresse nel modello, a causa dei vincoli restrittivi dell'approccio CFA (cioè, del fatto che gli item possono essere indicatori di un unico fattore), allora i risultati dell'analisi possono risultare sistematicamente distorti.
In particolare, si può osservare una sovrastima delle associazioni tra i fattori. 
Inoltre, i vincoli eccessivamente restrittivi dei modelli CFA possono anche minare la bontà di adattamento dei modelli e la validità discriminante dei fattori (Marsh et al., 2010, 2014).
I contemporanei modelli ESEM si pongono l'obiettivo di affrontare questo tipo di problemi.

\subsection{Fasi dell'analisi fattoriale}

Indipendentemente dal software statistico che viene utilizzato e dal livello di esperienza del ricercatore, lo sviluppo di un test psicometrico inizia con l'analisi fattoriale esplorativa. 
In essa si possono distinguere otto stadi successivi: 
\begin{enumerate}
\item la specificazione del problema, 
\item la scelta degli item, 
\item la verifica dell'adeguatezza della matrice di correlazione, 
\item l'estrazione iniziale dei fattori, 
\item la rotazione dei fattori, 
\item l'ulteriore selezione degli item, 
\item l'interpretazione dei risultati,
\item  la costruzione di un report della ricerca,
\item  la validazione dei risultati. 
\end{enumerate}
Queste diverse fasi della costruzione e validazione di un reattivo psicologivo verranno discusse nel seguito di queste dispense.

% ----------------------------------------------------------------
% ----------------------------------------------------------------
