\chapter{La specificazione del problema}
\label{ch:problema}


\section{Tipologie dei test psicometrici}

Prima di iniziare lo sviluppo di un test psicometrico, il ricercatore deve decidere quale tipologia di strumento sia più utile per affrontare il problema che ha di fronte. Possiamo infatti distinguere tra test orientati al criterio e test riferiti alla norma. 

\subsection{Test orientati al criterio} 

I test orientati al criterio hanno quale scopo il confronto fra gruppi precostituiti di individui. Gli item del test vengono selezionati in base alla loro capacità empirica di discriminare fra gruppi criterio, ad esempio, malati/sani, bocciati/promossi, schizofrenici/depressi. I test orientati al criterio sono costruiti utilizzando metodi empirici e non teorici. Al vantaggio di una chiara utilità pratica si accompagna il grande svantaggio di identificare fattori aventi una scarsa validità di costrutto, i quali risultano inutili per la comprensione dei processi psicologici. 

Il processo di sviluppo della scala è semplice, in quanto si devono selezionare gli item che mostrano punteggi differenti in gruppi-criterio noti. Se i gruppi criterio possono essere individuati con chiarezza è sempre teoricamente possibile sviluppare test in grado di discriminarli.

Tuttavia, non sempre i gruppi possono essere definiti in modo attendibile; oppure, la definizione dei gruppi criterio potrebbe avere senso solo all'interno di una teoria, ma non sia generalizzabile ad altre tradizioni teoriche. In questo caso, il test rischia di essere eccessivamente specifico, dimostrandosi utile solo nelle condizioni per cui è stato sviluppato, ma con scarsa capacità di potere essere utilizzato in condizioni diverse. 

Lo svantaggio principale dei test orientati al criterio è che il significato psicologico dei punteggi è ignoto. Non avendo una teoria sulle variabili psicologiche che
distinguono due gruppi, un buon test
discriminante non ci aiuta a capire perché tali gruppi siano diversi. Non è possibile sapere quanti costrutti siano coinvolti nella
determinazione di un punteggio. Inoltre, due punteggi uguali non implicano la presenza dei medesime meccanismi psicologici. Date queste ambiguità, utilizzando questi test non è possibile neppure aumentare le nostre conoscenze in maniera incrementale.

Il problema maggiore per lo sviluppo di questi strumenti è la definizione del criterio: qual è la variabile numerica che discrimina nella maniera maggiore tra i gruppi in esame (malati/sani)? 

La batteria di item iniziale deve essere sufficientemente grande e non è necessario che gli item abbiano validità di contenuto o validità di facciata. In generale, la batteria di item deve essere più grande di quelle usate per il metodo fattoriale o di analisi degli item: mancando criteri teorici per la scelta degli item, la scelta iniziale degli item è molto arbitraria ed è dunque necessario partire da un numero molto elevato di item. Ciò è meno vero quando gli item hanno una certa validità di facciata o di contenuto. In seguito, semplicemente si selezionano gli item che discriminano efficacemente fra i gruppi, o gli item fortemente associati con il punteggio criterio. È necessario poi replicare su un diverso campione la capacità discriminativa degli item selezionati.

\subsection{Test basati sulla norma}  

Le misure riferite ad una norma indicano la posizione del  rispondente in riferimento alla distribuzione di punteggi ottenuti nello stesso test da un campione di grandi dimensioni e rappresentativo della popolazione
di riferimento. La maggioranza dei test di personalità, attitudinali e cognitivi sono test basati sulla norma.

La metrica utilizzata per tale confronto può avere caratteristiche diverse. I punteggi standardizzati (con media $0$ e varianza unitaria) calcolati rispetto al gruppo di riferimento sono spesso convertiti  in una scala diversa, per esempio aventi media pari a $500$ e deviazione standard di $100$ (punteggi SAT), o aventi media pari
  a $100$ e deviazione standard di $15$ (es. punteggi WAIS-VI). È facile operare tale trasformazione. Il punteggio $Y$ di un rispondente può essere trasformato nel modo seguente in un  punteggio standard $X_i$, avente una media target pari a $\mu_s$ e una deviazione standard $\sigma_s$
\begin{equation}
X_i = \mu_s + z_i \sigma_s\notag
\end{equation}
dove $z$ è il punteggio standardizzato $z=\frac{Y-\bar{Y}}{s_Y}$.  


\section{Variabili latenti e sviluppo di uno strumento psicometrico}

Quando uno psicologo sviluppa una scala di misura è meno interessato agli item della scala che ai costrutti che si intendono misurare: ``Scale items are usually a means to the end of construct assessment \dots they are necessary because many constructs cannot be assessed directly'' (p. 2). Dato che non sono osservabili direttamente, i costrutti sono detti \emph{variabili latenti}. I costrutti sono interpretati come le cause (non visibili) che fanno in modo che gli item assumano un determinato valore per un determinato rispondente in un certo momento del tempo. Mentre alcune variabili, quali l'altezza, il peso, il battito cardiaco, la temperatura, possono essere misurate direttamente, i costrutti psicologici, quali l'ansia, la personalità, la qualità della vita, possono solo essere misurati indirettamente esaminando gli effetti che hanno sugli indicatori osservabili del costrutto. Gli item che vengono misurati tramite uno strumento sono gli indicatori osservati o empirici degli attributi del costrutto. Il dolore, ad esempio, è un costrutto psicologico non direttamente osservabile. Tuttavia, al dolore sono associati molteplici indicatori che sono direttamente osservabili, quali il pallore, la sudorazione profusa, ecc.

Allo scopo di misurare le variabili latenti del costrutto di interesse, lo psicologo deve identificare gli indicatori empirici del costrutto che possono essere direttamente osservabili. L'identificazione di tali indicatori empirici avviene attraverso (1) la chiarificazione del costrutto di interesse, (2) 	l'operazionalizzazione del costrutto, (3) la rassegna della letteratura rilevante, (4) l'analisi concettuale del costrutto.

%------------------------------------------------------------
\subsection{Chiarificazione del costrutto di interesse}
%------------------------------------------------------------

Vi sono diverse domande a cui lo psicologo deve rispondere prima di iniziare la selezione degli item, altrimenti rischia di produrre uno strumento con una scarsa validità di costrutto. 
\begin{enumerate}
\item
Qual è lo scopo dello strumento? Che cosa lo strumento dovrà misurare?
\item
Quali altri costrutti sono associati al costrutto di interesse? In che misura essi si distinguono dal costrutto di interesse? A tali domande non è semplice dare una risposta se il costrutto di interesse è complesso e astratto.
\item 
Lo strumento da costruire intende misurare le caratteristiche generali del costrutto di interesse, o intende focalizzarsi su alcuni specifici aspetti del costrutto?
\end{enumerate}

\begin{exmp}
Watson et al. (2007) si sono posti il problema di costruire uno strumento atto misurare la depressione superando i limiti degli strumenti già esistenti, quali il Beck Depression Inventory—II (BDI–II; Beck, Steer, \& Brown, 1996) e il Center for Epidemiological Studies Depression Scale (CES–D; Radloff, 1977). La scala costruita dagli autori prende il nome di Inventory of Depression and Anxiety Symptoms (IDAS). 

Per rispondere alla prima domanda, Watson et al. (2007) fanno notare che gli strumenti esistenti comprendono contenuti non specifici, ovvero non direttamente associati alla depressione. Infatti, sia il BDI-2 sia il CES–D contengono item che fanno riferimento a vari tipi di ansia. Di conseguenza, la validità discriminante di questi strumenti risulta compromessa. Inoltre, gli strumenti esistenti non contengono item che coprono tutto il dominio del costrutto della depressione maggiore, così come specificato dal Diagnostic and Statistical Manual of Mental Disorders (4th ed.). Infine, un'altra limitazione degli strumenti esistenti è il fatto che essi sono stati creati per produrre un singolo item della severità dei sintomi e quindi ignorano l'eterogeneità e la multidimensionalità del fenomeno depressivo. Questo si riflette sul fatto che gli strumenti esistenti manifestano una struttura fattoriale poco chiara, nel senso che autori diverse hanno trovato soluzioni fattoriali diverse. Lo strumento che Watson et al. (2007) intendono sviluppare vuole superare queste difficoltà costruendo una scale che direttamente rifletta, in ciascuna delle sue sottoscale, gli aspetti distintivi della depressione, a differenza di quanto accade per gli strumenti BDI–II e CES–D. 

Per rispondere alla seconda domanda, Watson et al. (2007) fanno notare come la depressione sia inserita in una rete nomologica di costrutti che include, in primo luogo, l'ansia. Diversamente dagli strumenti già esistenti, BDI–II e CES–D, Watson et al. (2007) si propongono espliciatamente di creare scale che riflettano gli aspetti specifici della depressione, distinti dall'ansia. Per fare questo, Watson et al. (2007) iniziano con il considerare un ampio insieme di item che rappresentano sintomi associati all'ansia. In questo modo viene perseguito l'obiettivo, all'interno dello strumento, di esaminare la relazione tra i sintomi d'ansia e quelli della depressione in modo da creare scale distinte per tali dimensioni così da aumentare a validità discriminante dello strumento.

Per rispondere alla terza domanda, Watson et al. (2007) affermano di volere sviluppare uno strumento che, nel suo punteggio generale, rifletta le caratteristiche generali della depressione mentre, quando vengono considerate le varie sottoscale che lo costituiscono, consente di misurare con precisione ciascuna delle dimensioni del costrutto esaminato.
\end{exmp}


%------------------------------------------------------------
\subsection{Operazionalizzazione del costrutto di interesse}
%------------------------------------------------------------

La definizione concettuale fornisce il significato teorico generale del costrutto. L'operazionalizzazione è invece una definizione del costrutto che ne consenta la misurazione (Vogt, 1993). Gli indicatori osservabili o empirici sono il prodotto finale di tale processo di operazionalizzazione (Keck, 1998) e diventano gli item dello strumento. Se il costrutto di interesse è stato sviluppato all'interno di un approccio teorico ben articolato, allora diventa più semplice stabilire quali siano le dimensioni che caratterizzano il costrutto, in che modo esse si possano manifestare, e in che modo possano essere misurate. Tuttavia, molti costrutti psicologici vengono spesso caratterizzati in maniera diversa da approcci teorici differenti. 

\begin{exmp}
Per chiarire il costrutto di depressione, Watson et al. (2007) fanno riferimento al DSM–IV il quale elenca nove criteri sintomatici per un episodio depressivo maggiore: (1) umore depresso per la maggior parte del giorno, quasi ogni giorno, come riportato dal soggetto o come osservato dagli altri, (2) marcata diminuzione di interesse o piacere per tutte, o quasi tutte, le attività per la maggior parte del giorno, quasi ogni giorno (come riportato dal soggetto o come osservato dagli altri), (3) significativa perdita di peso, senza essere a dieta, o aumento di peso, oppure diminuzione o aumento dell'appetito quasi ogni giorno, (4) insonnia o ipersonnia quasi ogni giorno, (5) agitazione o rallentamento psicomotorio quasi ogni giorno (osservabile dagli altri, non semplicemente sentimenti soggettivi di essere irrequieto o rallentato), (6) faticabilità o mancanza di energia quasi ogni giorno, (7) sentimenti di autosvalutazione o di colpa eccessivi o inappropriati (che possono essere deliranti), quasi ogni giorno, (8) ridotta capacità di pensare o di concentrarsi, o indecisione, quasi ogni giorno (come impressione soggettiva o osservata dagli altri), (9) pensieri ricorrenti di morte, ricorrente ideazione suicidaria senza un piano specifico, o un tentativo di suicidio, o l'ideazione di un piano specifico per commettere suicidio.

Per massimizzare l'utilità dell'IDAS, Watson et al. (2007) includono item molteplici per ciascuno dei nove criteri sintomatici per un episodio depressivo maggiore. Allo scopo di assicurare che un numero sufficiente di indicatori venga incluso nello strumento per ciascuna di queste dimensioni potenziali, nell'insieme di item preso in considerazione inizialmente, Watson et al. (2007) organizzano gli item potenziali in gruppi chiamati \emph{homogeneous item composites} (HIC). Essi fanno comunque notare come la costruzione di questi HIC non forza l'emergenza di un corrispondente fattore, ma soltanto consente di campionare tutto il dominio potenziale del costrutto.

\end{exmp}


%------------------------------------------------------------
\subsection{Rassegna della letteratura rilevante}
%------------------------------------------------------------

È importante per lo psicologo conoscere la maggior parte possibile della letteratura rilevante prima di iniziare il processo di costruzione di un nuovo strumento. Una sistematica rassegna della letteratura consente allo psicologo di valutare e organizzare i risultati empirici provenienti da fonti diverse che sono utili per individuare i potenziali indicatori empirici del costrutto. La rassegna della letteratura consente di sintetizzare le scoperte in un campo di ricerca, mette in evidenza gli aspetti metodologici associati al costrutto di interesse, chiarisce quali siano gli approcci teorici all'interno dei quali il costrutto è stato discusso e consente di mettere in evidenza, quando è opportuno, la ``dimensione dell'effetto'' attraverso le meta-analisi.


\begin{exmp}
Nel caso dell'articolo di Watson et al. (2007), gran parte dell'introduzione è dedicata alla rassegna della letteratura che viene discussa allo scopo di mettere in evidenza i limiti degli strumenti esistenti, considerare quali sono le caratteristiche degli item utilizzati, mettere in relazione gli indicatori utilizzati dagli strumenti esistenti con gli approcci teorici disponibili in relazione alla depressione e all'ansia, discutere le soluzioni fattoriali che sono state ottenute dai dati raccolti tramite gli strumenti esistenti, considerare quali aree di contenuto del costrutto non sono state adeguatamente indagate dagli strumenti esistenti.
\end{exmp}

%------------------------------------------------------------
\subsection{Analisi concettuale del costrutto}
%------------------------------------------------------------

L'analisi concettuale del costrutto è un altro metodo che può essere usato per determinare gli indicatori empirici del costrutto di interesse. 

È necessario stabilire quali siano gli attributi del costrutto di interesse, includendo la specificazione degli antecedenti e delle conseguenze che derivano da esso. Si devono identificare tutti gli usi che, nella letteratura specialistica, sono stati fatti del costrutto in esame. Infine, è necessario elencare tutti gli indicatori empirici che siano mai stati utilizzati per il costrutto esaminato. 

\begin{exmp}
Allo scopo di campionare efficacemente l'intero dominio del costrutto, Watson et al. (2007) hanno definito 20 HIC: 
Depressed Mood, Loss of Interest or Pleasure, Appetite Disturbance, 
Sleep Disturbance, 
Psychomotor Problems, 
Fatigue/Anergia, 
Worthlessness/Guilt, 
Cognitive Problems, 
Suicidal Ideation, 
Hopelessness, 
Melancholic Depression, 
Angry/Irritable Mood, 
High Energy/High Positive Affect, 
Anxious Mood, 
Worry, 
Panic, 
Agoraphobia, 
Social Anxiety, 
Traumatic Intrusions, 
Obsessive-Compulsive Symptoms. 

Tredici HIC (per un totale di 117 item) raggruppavano gli indicatori rilevanti per la depressione.  Tra questi, nove HICs (per un totale di 79 items) facevano riferimento ai sintomi di base della depressione maggiore così come descritta nel DSM–IV (depressed mood, loss of interest or pleasure, appetite disturbance, sleep disturbance, psychomotor problems, fatigue/anergia, worthlessness and guilt, cognitive problems, suicidal ideation). I quattro rimanenti HIC facevano riferimento alla presenza di sintomi della Hopelessness (Abramson, Metalsky, \& Alloy, 1989), ai sintomi specifici della depressione malinconica (Joiner et al., 2005), allo stato d'animo di rabbia/irritabilità  
(il quale rappresenta una forma alternativa di depressione tra gli adolescenti; DSM–IV, American Psychiatric Association, 1994, p. 327), e infine ad indicatori di energia e affetto positivo (i quali sono stati specificamente associati alla depressione; Mineka et al., 1998).

Gli altri sette HIC (per un totale di 63 item) sono stati introdotti per valutare sintomi associati all'ansia. Essi sono stati raggruppati nei termini dello stato d'animo ansioso, della worry, del panico, dell'agorafobia, dell'ansia sociale e delle intrusioni traumatiche associate al PTSD. 

\end{exmp}


%------------------------------------------------------------
\subsection{Metodi di ricerca qualitativi}
%------------------------------------------------------------

Metodi di ricerca qualitativi posso anche essere usati allo scopo di identificare i potenziali indicatori empirici del costrutto.  In particolare, possono essere usati i metodi della ricerca fenomenologica, dell'indagine naturalistica, i focus group e lo studio del caso singolo. 

L'indagine fenomenologica is pone l'obiettivo di descrivere il costrutto dal punto di vista di chi fa di esperienza di esso 
(Carpenter, 1999). Utili a questo proposito sono ovviamente le descrizioni che i soggetti forniscono della propria esperienza.

Nell'indagine naturalistica, lo psicologo osserva le conseguenze del costrutto così come si manifestano nel mondo naturale. Uno strumento possibile di raccolta dati è l'intervista con il paziente. 

Il focus group, originariamente sviluppato in ambito economico per ottenere opinioni su un determinato prodotto (Morse \& Field, 1995), ha le caratteristiche di  `a semi-structured group session, moderated by a group leader, held in an informal setting with the purpose of collecting information on a designated topic'' (Carry, 1994, p. 226). 

Un'altra possibile fonte di informazioni è costituita dagli studi sul caso singolo. 

%------------------------------------------------------------
\section{Lo sviluppo dello strumento}
%------------------------------------------------------------

Una volta selezionati gli indicatori empirici del costrutto, deve essere scelta una modalità di presentazione che consenta la raccolta efficiente dei dati. Ciascuno strumento può essere descritto in base a sei caratteristiche: (1)  formato, (2) composizione tipografica, (3) istruzioni ai soggetti, (4) la costruzione degli item, (5) formato di risposta, e (6) numero di item.

%------------------------------------------------------------
\subsection{Formato}
%------------------------------------------------------------

I formati di scala più usati sono lo scaling Thurstoniano, lo scaling di Guttman, le scale a differenziale semantico, le scale di valutazione grafica, 
semantic differential scales, graphic rating scales, le  scale visive di tipo analogico (visual analog scales) e le scale Likert.  Ci concentriamo qui sulle scale Likert per la loro importanza nei test psicometrici basati sull'analisi fattoriale. 

\subsubsection{Scala Likert} 

Sviluppata nel 1932 da Rensis Likert per misurare gli atteggiamenti, una scala Likert è una scala ordinale usata dai rispondenti per valutare il grado di accordo o disaccordo con l'affermazione che viene loro proposta. Di solito le alternative di risposta sono cinque o sette, da ``molto d’accordo'' a ``fortemente contrario.''

Essendo una scala ordinale, le risposte possono essere ordinate, ma le distanze tra i livelli della scala non sono quantificabili. Quindi le distanze tra i livelli ``sempre,'' ``spesso'' e ``talvolta'' non sono necessariamente uguali. In altri termini, non possiamo assumere che le differenze tra i livelli di risposta siano equidistanti anche se le differenze tra i valori numerici assegnati ai livelli della scala lo sono. 

C'è una lunga controversia sulla possibilità di trattare i valori numerici di una scala ordinale come se essi provenissero da una scala ad intervalli. In altri termini, ci si è chiesti se sia appropriato usare statistiche descrittive quali la media e la deviazione standard per i dati a questo livello di scala, e ci si è chiesti se sia appropriato usare i test parametrici per dati a livello di scala Likert. È risaputo che i test non parametrici, i quali non fanno assunzioni sulla forma della distribuzione della popolazione da cui abbiamo campionato i dati, hanno una potenza statistica nettamente inferiore ai test parametrici. Inoltre, concetti quali quelli di media e varianza non hanno senso se i livelli di una scala Likert non vengono considerati a livello di scala ad intervalli. Per queste ragioni alcuni autori ritengono problematico non potere trattare i dati provenienti da scale di tipo Likert come se fossero a livello di scala ad intervalli.

È stato risposto a tali difficoltà che sufficienti evidenze mostrano come risulti giustificato trattare i dati a livello di scala Likert come se fossero a livello di scala ad intervalli quando la numerosità campionaria è sufficientemente grande e quando i dati si distribuiscono in maniera approssimativamente normale. Altri autori (es. Jöreskog \& Sörbom, 1996) ritengono invece che le scale tipo Likert vadano considerate in ogni caso come ordinali, e debbano essere analizzate di conseguenza. Nel caso dell'analisi fattoriale e dei modelli di equazioni strutturali questo significa semplicemente che l'analisi si deve basare sul calcolo delle correlazioni policoriche.

In conclusione, la procedura che sta alla base delle scale Likert consiste nella somma dei punti attribuiti ad ogni singola domanda. I vantaggi della scala Likert consistono nella sua semplicità e applicabilità, mentre i suoi svantaggi sono il fatto che i suoi elementi vengono trattati come scale cardinali pur essendo ordinali e il fatto che il punteggio finale non rappresenta una variabile cardinale. 

%------------------------------------------------------------
\subsection{Composizione tipografica}
%------------------------------------------------------------

Criteri da considerare nella formattazione tipografica del test di un test psicometrico sono la facilità di lettura, la chiarezza e l'organizzazione. La formattazione dovrebbe tenere in considerazione l'età dei rispondenti e la potenziale difficoltà di lettura.

%------------------------------------------------------------
\subsection{Istruzioni ai soggetti}
%------------------------------------------------------------

Le istruzioni devono essere chiare e concise. Oltre ad illustrare la consegna, esse forniscono una cornice di riferimento che deve essere comune a tutti i rispondenti. Le istruzioni seguono un formato simile al seguente:

\begin{quote}
Lo studio ha come obiettivo generale [\dots]. In particolare, con la ricerca che qui presentiamo, si intendono ottenere dati relativi a [\dots]. Nel caso  tu decida di partecipare allo studio, questa ricerca prevede l'attuazione dei seguenti trattamenti [\dots]. La ricerca durerà [\dots] e vi parteciperanno [\dots] individui. Dalla partecipazione a questa ricerca sono prevedibili i seguenti benefici [\dots]. La partecipazione allo studio non comporta alcun rischio. Sei del tutto libero/a di non  partecipare allo studio. La tua adesione a questo programma di ricerca è completamente volontaria e potrà essere ritirata in qualsiasi momento. 

Ai sensi  del Decreto Legislativo 30 giugno 2003 n. 196 in materia di protezione dei dati personali, tratteranno i tuoi dati esclusivamente in funzione della realizzazione dello studio. Lo psicologo che ti seguirà nello studio ti identificherà con un codice: i dati che ti riguardano raccolti nel corso dello studio, ad eccezione del nominativo, saranno registrati, elaborati e conservati unitamente a tale codice, alla data di nascita, al genere. Soltanto il supervisore del progetto di ricerca potrà collegare questo codice al tuo nominativo. I dati, trattati mediante strumenti anche elettronici, saranno diffusi solo in forma rigorosamente anonima, ad esempio attraverso pubblicazioni scientifiche, statistiche e convegni scientifici. La tua partecipazione allo studio implica che il gruppo di ricerca che organizza lo studio e il Comitato etico potranno conoscere i dati che ti riguardano solo attraverso modalità tali da garantire la riservatezza della tua identità.

Potrai esercitare i diritti di cui all'art. 7 del Codice in materia di protezione dei Dati Personali (es. accedere ai tuoi dati personali, integrarli, aggiornarli, rettificarli, opporsi al loro trattamento per motivi legittimi, ecc.) rivolgendoti direttamente al responsabile della ricerca. Potrai interrompere in ogni momento e senza fornire alcuna giustificazione la tua partecipazione allo studio: in tal caso, i dati raccolti verranno distrutti. 

Se lo richiederai, alla fine dello studio potranno esserti comunicati i risultati ottenuti in generale e, in particolare, quelli che ti riguardano. Per ulteriori informazioni e comunicazioni durante la ricerca puoi rivolgerti a [\dots]. 

Potrai segnalare qualsiasi fatto che riterrai opportuno evidenziare, relativamente alla ricerca che ti riguarda, al Comitato Etico dell'Università degli Studi di Firenze. La segnalazione dovrà essere inoltrata all'attenzione di [\dots].

\end{quote}

È inoltre necessario che i partecipanti completino una dichiarazione di assenso (consenso informato). Ad esempio:

\begin{quote}
Io sottoscritto [\dots] dichiaro di aver ricevuto dal Dottor [\dots]
esaurienti spiegazioni in merito alla richiesta di partecipazione alla ricerca in oggetto, secondo quanto riportato nella scheda informativa qui allegata, copia della quale mi è stata prima d'ora consegnata (indicare data e ora della consegna).

Dichiaro altresì di aver potuto discutere tali spiegazioni, porre tutte le domande che ho ritenuto necessarie e di aver ricevuto risposte soddisfacenti, come pure di aver avuto la possibilità di informarmi in merito ai particolari dello studio con persona di mia fiducia.

Accetto dunque liberamente di partecipare alla ricerca, avendo compreso completamente il significato della richiesta e i rischi e benefici che possono derivare da questa partecipazione.

Acconsento al trattamento dei miei dati personali per gli scopi della ricerca nei limiti e con le modalità indicate nell'informativa fornitami con il presente documento.

Sono stato informato/a, inoltre, del mio diritto ad avere libero accesso alla documentazione relativa alla ricerca ed alla valutazione espressa dal Comitato Etico dell'Università degli Studi di Firenze.
\end{quote}

%------------------------------------------------------------
\subsection{La costruzione degli item}
%------------------------------------------------------------

La scelta di item tecnicamente adeguati sul piano strutturale e linguistico non è un problema statistico. La formulazione verbale degli item è molto importante in quanto essa contribuisce all'errore di misura. Per ridurre gli errori di misura, gli item devono essere formulati nella forma più chiara e meno ambigua possibile. È ovviamente necessario impiegare contenuti coerenti con la definizione del costrutto, ma non ci sono regole semplici per generare item che fanno emergere il costrutto che si cerca di misurare. Vanno certamente evitati contenuti che inducano atteggiamenti difensive e/o ostili nei rispondenti. La formulazione verbale deve inoltre essere appropriata al livello di scolarità dei rispondenti. Pett, Lackey e Sullivan (2003) forniscono le  raccomandazioni seguenti.

\begin{itemize}
\item Evitare affermazioni che si riferiscono al passato a meno che il costrutto faccia direttamente riferimento al passato.

\item Evitare affermazioni fattuali. Evitare affermazioni su cui quasi tutti (o
quasi nessuno) sono d'accordo.

\item Evitare l'uso di pronomi personali con un significato ambiguo.

\item Selezionare item che potenzialmente possano coprire l'intera gamma delle possibili risposte concernenti il costrutto di interesse.  

\item Se viene fatto riferimento ad argomenti sensibili, la formulazione verbale deve essere la più neutra possibile.

\item Utilizzare un linguaggio chiaro, semplice, diretto. Utilizzare frasi corte, altrimenti non ne è chiaro il senso. 

\item Evitare affermazioni ambigue o interpretabili in più modi. 

\item Evitare formulazioni sintattiche complesse. 

\item Evitare l'uso di parole a bassa frequenza o l'uso di una terminologia che potrebbe non essere capita dai rispondenti.

\item Disporre gli item aventi un contenuto sensibile verso la fine dello strumento.

\item Fare riferimento a comportamenti specifici e
non generali.

\item Evitare la duplicazione delle domande.
\end{itemize}

%Ulteriori raccomandazioni sono fornite della tabella~\ref{tab:item_wording}.
%
%\begin{table}
%\caption{Raccomandazioni per la scelta delle parole nella costruzione degli item.}
%\label{tab:item_wording}
%\centering
%\begin{tabulary}{\linewidth}{LLL}
%\hline
%Guida & Cattivo item & Buon item \\
%\hline
%Inserire un solo concetto per frase.  & I worry about being faced
%with an uncertain diagnosis, future, and insurance coverage.   & I worry about being
%faced with an uncertain diagnosis. \\
%
%Usare frasi affermative e negative, ma non nello stesso item.  & While genetic testing can help me plan for the future, it can result in decreased health insurance coverage. & (Positive) Genetic testing helps me plan for the future. (Negative) Genetic
%testing can be responsible for decreased health insurance coverage.\\
%
%Evitare espressioni gergali, colloquialismi e idiomi  & I try to eat as many
%phytoestrogens as possible in my diet to prevent breast cancer. & I try to eat a healthy
%diet on a daily basis.\\
%
%Evitare l'uso della negazione per cambiare il significato di un item.  & I am not worried about my
%future life.  & I am at peace when I think about my future life.\\
%
%Evitare item troppo lunghi.  & I might be helped to make important future life
%decisions (e.g., getting married and having children) by knowing I carry the gene. & I might be helped to make important future life decisions by knowing I carry the gene.\\
%
%Evitare l'uso del doppio negativo.  & I am not in favor of the federal government stopping funding for genetic research.  & I am in favor of federal funding for genetic research.\\
%
%Evitare item che hanno un doppio contenuto. & I do not support genetic testing because the results could make me ineligible for health insurance. & I am worried about being able to maintain health and life insurance coverage.\\
%
%Evitare item che suggeriscono una determinata risposta. & I agree that nurses play an indispensable role in genetic testing. & Nurses play an indispensable role in
%genetic testing.\\
%
%Evitare termini quali ``dovrebbe,'' ``è la causa di,'' ``sempre,'' ``mai,'' ``nessuno.''  & Genetic testing should be required for all women who have a familial risk for breast cancer. & I would recommend that all women who have a familial risk for breast cancer have genetic testing.\\
%\hline
%\end{tabulary}
%\end{table}

\subsubsection{Desiderabilità sociale} Quando si sviluppa lo strumento è necessario tenere in considerazione il fatto che i rispondenti tendono a fornire risposte socialmente desiderabili piuttosto che risposte veritiere (Rosenthal \& Rosnow, 1991; Waltz et al.,
1991). La Desiderabilità sociale non soltanto introduce dei bias nello strumento ma può anche comprometterne la validità.

La Desiderabilità Sociale si riferisce al bisogno provato da alcuni individui di approvazione sociale e accettazione, e alla credenza di poterle ottenere attraverso comportamenti appropriati e culturalmente accettati (Marlowe \& Crowne, 1961). La Desiderabilità Sociale consiste nella tendenza a fornire risposte molto positive quando vengono poste domande su di sé, con l'obiettivo di risultare positivamente agli occhi dell'altro. Marlowe e Crowne (1960) hanno proposto la scala di valutazione MC-SCS (Marlowe-Crowne Social Desirability Scale), largamente utilizzata per indagare questo costrutto. Un'altra scala di valutazione molto utilizzata è la BIDR (Balanced Inventory of Desirable Responding, 1991) proposta da Paulhus: tale scala contiene 40 item, volti a rilevare la gestione delle impressioni e l'autoinganno. 


\subsubsection{Item marker}

Quando si anticipa la presenza di più costrutti latenti, è utile utilizzare nell'insieme degli item alcuni item marker, ovvero item che correlano molto con un solo fattore e pochissimo con altri. Questo facilità l'interpretazione dei fattori.  I  marker consentono infatti di attribuire ai fattori un nome (etichetta) coerente con l'area semantica cui i maker fanno riferimento. 

\subsubsection{Campionamento del dominio}

Il campionamento del dominio può essere inteso sia come campionamento del contenuto o come campionamento del comportamento. L'adeguatezza del campionamento del contenuto riguarda il fatto che l'insieme degli item sia o meno in grado di rappresentare il dominio di contenuto di interesse. Questa caratteristica è un indice dell'adeguatezza del test nel misurare ciò che intende misurare e dovrebbe garantire che le risposte agli item possano rappresentare una stima della quantità di costrutto posseduta dal rispondente. 

Il campionamento del comportamento riguarda invece il grado in cui le risposte a un test costituiscono un campione adeguato dei comportamenti che il test intende misurare. In questo caso ci si chiede se il test riflette i comportamenti  che intende valutare e possiede dunque un valore descrittivo  del comportamento del rispondente.

Un item mal formulato determina una distorsione delle risposte e non può essere considerato rappresentativo di nessun dominio di contenuto né di nessun universo di comportamenti.

Per la generazione iniziale degli item è molto importante considerare il parere della popolazione target e degli esperti. Interviste accuratamente strutturate e a risposta aperta con esperti o potenziali soggetti permettono non solo di verificare che gli item siano rappresentativi o rilevanti per il costrutto, ma anche che siano formulati correttamente. Questo processo può anche suggerire sfaccettature ulteriori rispetto a quelle progettate inizialmente e la necessità di raffinare il costrutto. Nello sviluppo di un test è molto utile ascoltare il parere di persone inserite nel contesto applicativo del test anche per sapere qual è la terminologia specifica da utilizzare nella formulazione degli item, o se gli item sono chiari, o se la scala di risposta è di facile compilazione.

È anche utile rivolgersi a giudici esterni aventi una conoscenza approfondita del dominio di contenuto per ottenere una prospettiva esterna e autorevole che aiuti nell'individuazione degli item da eliminare  e di quelli che richiedono un raffinamento.

Gli item di un test dovrebbero essere distribuiti in modo che riflettano la relativa importanza delle varie sfaccettature del costrutto target (Nunnally \& Bernstein, 1994). Se gli item per una certa sfaccettatura sono troppi o troppo pochi, i punteggi e le inferenze ottenute da questi punteggi saranno distorte.


\section{Numero delle opzioni di risposta}

Un item è costituito da due parti: l'item stem, cioè il testo che contiene la domanda o l'affermazione da valutare e le alternative di risposta. 

In una scala di tipo Likert, le categorie di risposta si dicono a parziale autonomia semantica, ovvero sono tali per cui le modalità di risposta devono essere confrontate con le altre affinché il rispondente sia in grado di stabilire il loro valore.   A ciascuna modalità di risposta viene attribuito un punteggio (4, 3, 2, 1 oppure 3, 2, 1, 0), e la somma (o media) dei punteggi alle risposte di ciascun individuo sull'intera batteria rappresenta la posizione dell'individuo sul concetto indagato. 

Per esempio,
\[
\text{Fortemente d'accordo} \quad 7\quad 6 \quad 5 \quad 4 \quad 3 \quad 2 \quad 1 \quad \text{Fortemente in disaccordo}
\]
oppure,
\[
\text{Molto} \quad \text{Abbastanza}\quad \text{Poco} \quad \text{Per niente}
\]

Il numero ottimale delle opzioni di risposta è stato dibattuto a lungo. Per esempio, Schutz e Rucker (1975) hanno trovato che ``the number of available response categories does not materially affect the cognitive structure derived from the results'' (p. 323), il che suggerisce che il numero di opzioni di risposta ha poco effetto sui risultati ottenuti. Tale conclusione, tuttavia, è stata contraddetta da altri ricercatori.  Per esempio, Garner (1960) ha suggerito risultati massimamente informativi si ottengono utilizzando più di 20 opzioni di risposta. D'altra parte, Green e Rao (1970), hanno trovato che i risultati migliori si ottengono con sei o sette alternative di risposta, con un guadagno molto piccolo all'aumentare delle categorie di risposta al di là di sette. In un articolo molto citato, Preston e Colman (2000) hanno esaminato le risposte fornite da un campione di rispondenti variando il numero di opzioni di risposta pari a 2, 3, \dots, 11, e 101. Dopo avere calcolato l'attendibilità test-retest e la validità dello strumento, oltre al potere discriminante degli item, hanno concluso che le scale a 2, 3 e 4 passi hanno prestazioni piuttosto basse, avendo gli indici calcolati valori molto maggiori per le scale di risposta con un numero maggiore di opzioni di risposta. In particolare, i risultati dello studio suggeriscono che scale di valutazione con 7, 9 o 10 opzioni di risposta sono da preferire rispetto ad altri numeri di alternative di risposta.

Oltre alle scale Likert è possibile usare le risposte auto-ancoranti, ovvero quelle in cui gli item prevedono solo due aggettivi di risposta, estremi (per esempio, ``per niente'' e ``molto''), legati da un segmento continuo in cui il rispondente deve scegliere la propria posizione. Un esempio è la Visual Analogue Scale usata nella misura dell'umore. Tali scale sono molto più rare delle scale Likert.


\subsection{Item a codifica inversa}

Alcuni item correlano fortemente in maniera negativa con gli altri item e con il punteggio totale del test. Tali item richiedono una codifica inversa. Ad esempio, due item del questionario S.T.A.I per la valutazione dell'ansia sono codificati nel modo seguente.

\noindent ``Sono preoccupata.''
\[
\text{Per nulla} \quad \text{Un po'}\quad \text{Abbastanza} \quad \text{Moltissimo}
\]
con valori 1, 2, 3 e 4, rispettivamente.

\noindent ``Mi sento bene.''
\[
\text{Per nulla} \quad \text{Un po'}\quad \text{Abbastanza} \quad \text{Moltissimo}
\]
con valori 4, 3, 2, e 1, rispettivamente. 

In ambito psicometrico si è soliti ritenere che due proprietà contrarie giacciano sullo stesso continuum latente.  Nella costruzione di un test psicologico viene dunque consigliato di utilizzare sia item con contenuto orientato nella direzione del costrutto (per cui punteggi alti nell'item sono il riflesso di un alto livello del costrutto) sia nella direzione opposta (per cui punteggi alti nell'item sono il riflesso di un basso livello del costrutto). Nel primo caso si parla di \emph{straight item}, nel secondo di \emph{reverse item}. Lo scopo centrale degli item reverse è quello di contrastare l'acquiescenza, ovvero di rallentare il soggetto nella compilazione del test, evitando di rispondere in maniera automatica, così da prestare maggiore attenzione al contenuto degli item.


\section{Numero di item}

Un test psicometrico, oltre ad essere valido, deve minimizzare l'errore di misura. L'attendibilità di uno strumento sia dall'attendibilità di ciascun item, sia dal numero di item che lo compongono. Tratteremo di questo argomento nella \ref{sec:reliability_number_item}.

Kline (1986) suggerisce di costruire un numero di item almeno doppio del numero di item che andranno a costituire il test finale. La lunghezza del test dipende dal suo scopo.  Un  Test di valutazione delle abilità per la scuola primaria non deve richiedere più di trenta minuti per essere completato, altrimenti la fatica e la noia finiscono per distorcere i risultati dello strumento. Lo stesso si può dire per un test di personalità per soggetti adulti.  Idealmente, un test dovrebbe essere il più breve possibile, a patto di raggiungere un livello adeguato di validità. Come regola euristica, Kline (1986) suggerisce la soglia di almeno 50 item nella forma finale del test.

\section{Numero di soggetti}

Vi è poco accordo su quale sia la grandezza del campione necessaria per lo svolgimento dell'analisi fattoriale. Nunnally (1978) ha suggerito che il campione deve essere costituito da almeno 10 soggetti per ciascun item. Comrey e Lee (1992) hanno fornito le seguenti indicazioni: 50—very poor, 100—poor, 200—fair, 300—good, 500—very good, 1,000 or more—excellent. Secondo altri autori 
\begin{quote}
 as a general rule of thumb, it is comforting to have at least 300 cases for factor analysis (Tabachnick e Fidell, 2001). 
\end{quote}

% ----------------------------------------------------------------
% ----------------------------------------------------------------
