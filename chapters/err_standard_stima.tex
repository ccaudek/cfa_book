% DO NOT COMPILE THIS FILE DIRECTLY!
% This is included by the other .tex files.
%%------------------------------------------------------------
\chapter{La stima del punteggio vero}
\label{ch:err_stnd_stima}
%%------------------------------------------------------------

Uno degli scopi principali della valutazione psicologica è quello di stimare il punteggio vero del rispondente. Il punteggio osservato $X$ differisce dal punteggio vero $T$ a causa della presenza dell'errore della misurazione:
$X = T + E$. Poniamoci ora il problema di utilizzare i concetti della Teoria Classica  per  stimare il punteggio vero di un rispondente utilizzando il suo punteggio osservato  e l'attendibilità del test. Questa stima è utile soprattutto quando è necessario costruire un intervallo di confidenza per il punteggio vero.

Per costruire l'intervallo di confidenza del punteggio vero dobbiamo utilizzare due quantità: 
\begin{itemize}
\item 
una stima del punteggio vero, e 
\item  l'errore standard della stima (ovvero, una stima della deviazione standard della distribuzione delle stime del punteggio vero che si otterrebbe se il test venisse somministrato infinite volte sotto le stesse condizioni). 
\end{itemize}
Iniziamo con il problema della stima del punteggio vero.


%-------------------------------------------------------------------
\section{Il paradosso di Kelley}

Nella sua monografia del 1954, \enquote{Clinical versus statistical prediction: A theoretical analysis and a review of the evidence}, Paul Meehl suscitò un grande scalpore con una convincente dimostrazione del fatto che metodi meccanici di combinazione dei dati, come ad esempio la regressione multipla, sono in grado di fornire delle predizioni migliori di quanto sia in grado di fare la diagnosi clinica eseguita da esperti.
L'enorme quantità di letteratura che è stata prodotta in seguito a tale contributo ha fornito forti e univoche evidenze a sostegno di questa osservazione. 

È interessante notare che Robyn Dawes (2005) ha pubblicato un articolo su \emph{Journal of Clinical Psychology} (61, 1245--1255) dal titolo seguente: \enquote{The ethical implications of Paul Meehl's Work on comparing clinical versus actuarial prediction methods}. 
L'argomento principale sostenuto da Dawes è che, date le evidenze molto convincenti che sono disponibili, non è etico usare il giudizio clinico in preferenza all'uso di modelli statistici di previsione. 
Citiamo dall'abstract:

\begin{displayquote}
Whenever statistical prediction rules [\dots] are available for making a relevant prediction, they should be used in preference to intuition. [\dots] Providing service that assumes that clinicians \enquote{can do better} simply based on self-confidence or plausibility in the absence of evidence that they can actually do so is simply unethical.
\end{displayquote}

Sulla base di quanto detto sopra, e in riferimento ai nostri scopi presenti, si pone dunque il problema di capire come sia possibile utilizzare il modello di regressione per ottenere una stima del punteggio vero di un rispondente.
A questo proposito si deve notare che è necessario tenere in considerazione il fatto che le nostre variabili indipendenti sono corrotte dall'errore di misurazione, mentre  il modello di regressione tradizionale presuppone che le variabili indipendenti siano misurate senza errori.
%Secondo la CTT, il punteggio osservato $X$ differisce dal punteggio vero $T$ a causa dell'errore della misurazione. 
%Ricordiamo che, secondo Lord e Novick (1968), il punteggio vero è il valore atteso del punteggio osservato, ovvero la  media dei valori ottenuti a seguito di ipotetiche somministrazioni del test nelle identiche condizioni. 
%Sulla base di quando detto sopra facciamo ora riferimento ad un tema che è stato discusso negli anni '20 Kelley, e che riguarda la possibilità di utilizzare il modello di regressione per ottenere una stima del punteggio vero di un rispondente.
Le considerazioni seguenti sono state proposte da Kelly negli anni '20.

Come dimostrato in seguito, la formula di Kelley si basa sull'equivalenza algebrica secondo la quale l'attendibilità è uguale al quadrato del coefficiente di correlazione tra i punteggi osservati e i punteggi veri.
In base alla formula di Kelley, il punteggio vero di un rispondente può essere stimato   nel modo seguente mediante il modello di regressione:
\begin{equation}
\hat{T} = \mu_x + \rho  (X - \mu_x),
\label{eq:true_score}
\end{equation}
laddove $X$ è il punteggio osservato, $\mu_x$ è la media dei punteggi ottenuti da tutti i rispondenti di un campione e $\rho$ è l'attendibilità del test. 

Quando l'attendibilità è perfetta ($\rho = 1$), il punteggio vero è uguale al punteggio osservato.  Quando l'attendibilità è zero (tutta la varianza è dovuta all'errore della misurazione), allora la stima migliore del punteggio vero è data dalla media del campione. Quando $0 < \rho < 1 $, la stima del punteggio vero corrisponde ad un valore che si discosta dal punteggio osservato nella direzione della media del campione. La stima del punteggio vero, dunque, esibisce la proprietà della regressione verso la media del punteggio osservato, in funzione dell'attendibilità del test\footnote{Si possono fare molti esempi di questo fenomeno nello sport, dove gli atleti che fanno  bene in una stagione solitamente nella stagione successiva ottengono risultati meno buoni (Bernstein, 1996; Stigler, 1999). Un esempio famoso è stato discusso da Tversky e Kahneman (1974): fare i complimenti ad un allievo pilota per un buon  atterraggio solitamente porta ad una prestazione peggiore nell’atterraggio successivo; sgridare  in modo fermo un allievo pilota per un cattivo atterraggio porta solitamente ad un miglioramento negli atterraggi successivi.}.

La formula~\ref{eq:true_score} può essere interpretata dicendo che, per stimare il punteggio vero di un rispondente, partiamo dalla media della distribuzione della  popolazione dei rispondenti e ci spostiamo nella direzione del punteggio osservato di un rispondente. Tuttavia, non raggiungiamo il valore del punteggio osservato: la quantità di cui ci spostiamo è proporzionale all'attendibilità.
In altre parole, a seconda della dimensione di $\rho$, la stima del punteggio vero di un individuo è dovuta, in parte, a dove si trova l'individuo in relazione al gruppo di appartenenza: la stima del punteggio vero dell'individuo si sposterà verso l'alto se l'individuo è collocato sotto la media del gruppo di appartenenza e si sposterà verso il basso se l'individuo è collocato al di sopra della media del gruppo di appartenenza.
Questa equazione è stata chiamata il \emph{paradosso di Kelley}.

È importante sottolineare che l'interpretazione precedente rivela che la formula di Kelley contraddice la nozione intuitiva secondo cui il punteggio osservato può essere utilizzato quale stima del punteggio vero (cioè, $\hat{T} = X$). Tale ragionamento ingenuo sarebbe corretto se l'attendibilità del test fosse perfetta ($\rho = 1$). All'altra estremità dello spettro, quando $\rho = 0$, la formula di Kelley ci suggerisce $\mu_x$ quale stima del punteggio vero, il che è equivale a dire che il punteggio osservato deve essere ignorato -- infatti se la varianza di $X $ è solamente dovuta all'errore di misurazione, allora il test è del tutto inutile quale strumento inferenziale per differenziare le abilità dei rispondenti. Fortunatamente, in pratica è molto improbabile che $\rho = 0$. Se $\rho$ cade tra gli estremi di 0 e 1, allora il punteggio vero stimato sarà compreso tra il punteggio osservato e $\mu_x$. Per capire cosa esso catturi, possiamo citare Kelley(1947), che osservò:
\begin{displayquote}
This is an interesting equation in that it expresses the estimate of true ability as the weighted sum of two separate estimates, -- one based upon the individual's observed score, $X_1$ [$X$ nella notazione corrente], and the other based upon the mean of the group to which he belongs, $M_1$ [$\mu_x$ nella notazione corrente]. 
If the test is highly reliable, much weight is given to the test score and little to the group mean, and vice versa. 
\end{displayquote}


\begin{proof}

Come si arriva all'equazione di Kelley? 
Abbiamo visto in precedenza come l'equazione che mette in relazione il punteggio osservato con il punteggio vero non è altro che un modello di regressione con intercetta nulla e pendenza unitaria: $X = 0 + 1 \cdot T + E$.
In questo caso, però, il problema è diverso, in quanto noi vogliamo \emph{predire} il punteggio vero sulla base del punteggio osservato per mezzo di un modello di regressione (Nunnally, 1978). 
Avendo quale scopo quello di \enquote{predire} il punteggio vero $T$ sulla base del punteggio osservato $X$, il modello di regressione diventa
$$
T = \alpha + \beta X + \varepsilon.
$$
Se esprimiamo le variabili come deviazioni dalla  media, $x = X - \bar{X}$ e $\tau = T - \Ev(T)$, allora l'intercetta diventa uguale a zero e il modello diventa
$
\tau = \beta x + \varepsilon
$, 
ovvero
$
\hat{\tau} = \beta x. 
$
Il problema è quello di calcolare il coefficiente $\beta$.
 
Nel modello $\hat{\tau} = \beta x$, la pendenza della retta di regressione è uguale a
$\beta = \frac{\sigma_{\tau x}}{\sigma^2_x}$. 
Possiamo dunque scrivere il modello di regressione nel modo seguente:
\begin{equation}
\hat{\tau} = \frac{\sigma_{\tau x}}{\sigma^2_x} x.\notag
\label{eq:hat_t_1}
\end{equation}
La correlazione tra $x$ (o $X$) e $\tau$ (o $T$) è uguale a $\rho_{\tau x} =
  \frac{\sigma_{\tau x}}{\sigma_x \sigma_{\tau}}$. 
Dunque  $\sigma_{\tau x} =
  \rho_{\tau x}\sigma_x \sigma_{\tau}$ e l'equazione precedente diventa
\begin{align}
\hat{\tau} %&= \frac{\sigma_{TX}}{\sigma^2_X} X  \notag\\[10pt]
&= \frac{\rho_{\tau x}\sigma_x \sigma_{\tau}}{\sigma^2_x} x  \notag\\
&= \rho_{\tau x}\frac{\sigma_{\tau}}{\sigma_x} x. \notag
\label{eq:hat_t_2}
\end{align}
In base alla definizione di attendibilità, la varianza del punteggio vero è  
$
\sigma^2_{\tau} = \sigma^2_x \rho_{xx'}
$. 
Dunque, la deviazione standard del punteggio vero diventa 
$
\sigma_{\tau} = \sigma_x \sqrt{\rho_{xx'}}
$. 
Sostituendo questo risultato nell'equazione precedente otteniamo
\begin{align}
\hat{\tau} &= \rho_{\tau x}\frac{\sigma_x \sqrt{\rho_{xx'}}}{\sigma_x} x
\notag\\[8pt]
&=  \rho_{\tau x}  \sqrt{\rho_{xx'}} x. \notag
\end{align}
In precedenza abbiamo visto che $\rho^2_{\tau x} = \rho_{xx'}$, dunque
\begin{align}
\hat{\tau} &= \rho_{\tau x} \sqrt{\rho_{xx'}} x \notag\\
        &= \sqrt{\rho_{xx'}} \sqrt{\rho_{xx'}} x \notag\\
        &= \rho_{xx'} x.
%\label{eq:hat_t_part}
\end{align}
In conclusione, una stima del punteggio vero si ottiene moltiplicando il punteggio osservato, espresso come deviazione dalla media, per il coefficiente di attendibilità.

Riscriviamo ora la formula appena ottenuta nei termini del punteggio grezzo $X$ (non in termini di deviazioni dalla media. 
Per fare ciò, sommiamo $\bar{X}$ così da ottenere 
\begin{equation}
\hat{T} = \rho_{XX'} (X - \bar{X}) + \bar{X}, \notag
\end{equation}
laddove $\hat{T}'$ è la stima del punteggio vero grezzo. Sviluppando otteniamo
\begin{align}
\hat{T} &= \rho_{XX'} (X - \bar{X}) + \bar{X}\notag\\
 &=  X\rho_{XX'}  - \bar{X} \rho_{XX'} + \bar{X}\notag\\
&= \bar{X} (1 - \rho_{XX'}) + X\rho_{XX’}\notag\\
&= \bar{X} - \bar{X}\rho_{XX'} + X\rho_{XX’}\notag\\
&= \bar{X} + \rho_{XX'} (X - \bar{X}).\notag
\end{align}
Per i dati campionari, la formula diventa:
\begin{equation}
\hat{T} = \bar{X} + r_{XX'}  (X - \bar{X}),\notag
\end{equation}
dove $X$ è il punteggio (grezzo) osservato, $\bar{X}$ è la media dei punteggi osservati di un campione di rispondenti e $r_{XX'}$ è il coefficiente di attendibilità. 
\end{proof}


\begin{exmp}
Posto un coefficiente di attendibilità pari a 0.80 e una media del test pari a $\bar{X} = 100$, si trovi una stima del punteggio vero per un rispondente con un punteggio osservato uguale a $X$ = 115.
\end{exmp}
\begin{solu}
La stima del punteggio vero  $\hat{T}$ è uguale a 
\begin{align}
\hat{T} &= \bar{X} + r_{XX'}  (X - \bar{X})\notag\\
&= 100 + 0.80 \cdot (115 - 100) = 112.\notag
\end{align}
\end{solu}


\section{L'errore standard della stima}

Oltre a ottenere una stima del punteggio vero da un punteggio osservato, il modello di regressione di Kelley ci fornisce anche l'errore standard della stima.
È chiaro che la stima del punteggio vero è difficile da interpretare se non è accompagnata da una qualche indicazione sulla precisione della stima. 
Tale informazione viene appunto fornita dall'\emph{errore standard della stima}.

Se il test potesse essere somministrato ad un rispondente più volte sotto le identiche condizioni, sarebbe possibile ottenere in ciascuna somministrazione una stima del valore vero $\hat{T}$. 
A causa dell'errore della misurazione, il punteggio osservato non può che variare in ciascuna ipotetica somministrazioni del test e, di conseguenza, in ciascuna ipotetica somministrazione varierà anche la stima di $\hat{T}$.
La deviazione standard di tali (ipotetiche) stime di $\hat{T}$ è chiamata \textit{errore standard della stima}. 
L'errore standard della stima, $\sigma_{\hat{T}}$, si calcola con la formula seguente:
\begin{equation}
\sigma_{\hat{T}} = \sigma_X \sqrt{\rho_{XX'} (1 -\rho_{XX'})}.
\label{eq:std_err_estimate}
\end{equation}

\begin{proof}
Per ricavare l'equazione~\ref{eq:std_err_estimate}, si definisce $\varepsilon$ l'errore che si commette quando si stima il punteggio vero $\hat{T}$ con il punteggio osservato $T$ (si veda Lord e Novick, 1968):
\begin{equation}
\varepsilon = T - \hat{T}.
\end{equation}
Si presti attenzione alla notazione: $E = X - T$ indica l'errore della misurazione, ovvero la differenza tra il punteggio osservato e il punteggio vero. 
Invece $\varepsilon = T - \hat{T}$ indica la differenza tra il punteggio vero e la stima del punteggio vero. 
Avendo che $\hat{T} = \bar{X} + \rho_{XX'} (X - \bar{X})$, la varianza di $\varepsilon = T - \hat{T}$ si può scrivere come
\begin{align}
\var(\varepsilon) &=  \var(T - \hat{T})\notag\\
&= \var(T - \bar{X} - \rho_{XX'} X + \rho_{XX'}\bar{X}).\notag
\end{align}
Dato che la varianza di una variabile aleatoria non cambia sommando a tale variabile una costante, dobbiamo semplicemente calcolare 
\begin{align}
\var(\varepsilon) &= \var(T - \rho_{XX'}X).
\end{align}
Dobbiamo trovare la varianza della somma di due variabili aleatorie, una delle quali moltiplicata per una costante. Dunque: 
\begin{align}
\var(\varepsilon) &= \var(T) + \rho_{XX'}^2 \var(X) - 2  \rho_{XX'} \cov(X,T),\notag
\end{align}
ovvero, semplificando la notazione, 
\begin{align}
\sigma^2_{\varepsilon} &= \sigma^2_T + \rho_{XX'}^2 \sigma^2_X - 2  \rho_{XX'} \sigma_{XT}.\notag
\end{align}
La quantità $\rho_{XX'}$ è il coefficiente di attendibilità.  Quindi
\begin{align}
\sigma^2_{\varepsilon} &= \sigma^2_T + \left(\frac{\sigma_T^2}{\sigma_X^2}\right)^2 \sigma^2_X - 2  \frac{\sigma_T^2}{\sigma_X^2} \sigma_{XT}.\notag
\end{align}
Semplificando otteniamo
\begin{align}
\sigma^2_{\varepsilon} &= \sigma^2_T + \frac{\sigma_T^4}{\sigma_X^4}
\sigma^2_X - 2  \frac{\sigma_T^2}{\sigma_X^2} \sigma_{XT}\notag\\ 
&= \sigma^2_T + \sigma^2_T\frac{\sigma_T^2}{\sigma_X^2} -  \sigma_T^2 2
\frac{\sigma_{XT}}{\sigma_X^2} \notag\\ 
&= \sigma^2_T \left(1 + \frac{\sigma_T^2}{\sigma_X^2} - 2
  \frac{\sigma_{XT}}{\sigma_X^2}\right).\notag 
\end{align}
Daro che $\sigma_{XT}=\sigma^2_T$, l'equazione precedente diventa uguale a
\begin{align}
\sigma^2_{\varepsilon} &= \sigma^2_T \left(1
  +\frac{\sigma_T^2}{\sigma_X^2} - 2
  \frac{\sigma_{T}^2}{\sigma_X^2}\right)\notag\\
&= \sigma^2_T \left(1 - 
  \frac{\sigma_{T}^2}{\sigma_X^2}\right).\notag
\end{align}
L'errore standard della stima è dunque uguale a 
\begin{align}
\sigma_{\varepsilon} 
&=\sigma_T \sqrt{1-\frac{\sigma^2_T}{\sigma^2_X}}\notag\\
&=\sigma_T \sqrt{\frac{\sigma^2_X - \sigma^2_T}{\sigma^2_X}}\notag\\
&=\frac{\sigma_T}{\sigma_X} \sqrt{\sigma^2_X - \sigma^2_T}.\notag
\end{align}
Dato che $\sigma^2_X=\sigma^2_T+\sigma^2_E$, abbiamo 
\begin{align}
\sigma_{\varepsilon} 
 &= \frac{\sigma_T}{\sigma_X} \sqrt{\sigma^2_E }\notag\\
&=  \frac{\sigma_T}{\sigma_X} \sigma_E \notag\\
&= \sqrt{\rho_{XX'}} \sigma_E. \notag
\end{align}
Ricordando che l'errore standard della misurazione è
  $\sigma_E = \sigma_X \sqrt{1 - \rho_{XX'}}$, 
  possiamo scrivere
\begin{align}
\sigma_{\varepsilon}  &= \sqrt{\rho_{XX'}} \sigma_E \notag\\
&= \sqrt{\rho_{XX'}} \sigma_X
\sqrt{1-\rho_{XX'}} \notag\\
&= \sigma_X \sqrt{\rho_{XX'} (1 - \rho_{XX'})}.\notag
\end{align}
\end{proof}

Per dati campionari, l'errore standard della stima si calcola nel modo seguente:
$$
s_{\hat{T}} = s_X \sqrt{r_{XX'} (1-r_{XX'})}, 
$$
dove $s_X$ è deviazione standard del campione e $r_{XX'}$ è il
coefficiente di attendibilità.


\section{Intervallo di confidenza per il punteggio vero}

L'errore standard della stima $\sigma_{\hat{T}}$ viene usato per 
calcolare l'intervallo di confidenza per il punteggio vero\footnote{
L'intervallo di confidenza per il punteggio vero definisce un campo di variazione all'interno del quale ci si aspetta di trovare il punteggio vero nel caso in cui il rispondente ripetesse infinite volte il test. Ad ogni intervallo di confidenza viene associato un livello di confidenza pari a 1 - $\alpha$. Il valore 1 - $\alpha$ indica il livello di copertura fornito dall'intervallo: esiste sempre una probabilità pari ad $\alpha$ che il punteggio vero si trovi al di fuori dell'intervallo.
}:
\begin{equation}
\hat{T} \pm z  \sigma_{\hat{T}},
\end{equation}
laddove $\hat{T}$ è la stima del punteggio vero e $z$ è il quantile della normale standardizzata al livello di probabilità desiderato. Se il campione è piccolo (minore di 30) è opportuno usare $t$ anziché $z$. 

Si osservi che l'intervallo $\hat{T} \pm z  \sigma_{\hat{T}}$ è centrato sulla 
\textit{stima puntuale del valore vero} e ha una ampiezza che dipende sia dal livello di copertura desiderato (da cui dipende il quantile $z_{\frac{\alpha}{2}}$), sia dal grado di precisione dello stimatore misurato dall'errore standard della stima, $\sigma_{\hat{T}} = \sigma_X \sqrt{\rho_{XX'} (1 -\rho_{XX'})}$.  L'errore standard della stima diventa tanto più grande quanto minore è l'attendibilità $\rho_{XX'}$ del test.
 
L'intervallo di confidenza `ricorda' allo psicologo quanto sia imprecisa la misura che utilizza: tanto più grande è l'intervallo di confidenza, tanto maggiore è l'incertezza dell'interpretazione. L'intervallo di confidenza è lo strumento che consente allo psicologo di giungere ad una conclusione sapendo qual è la probabilità che tale conclusione sia sbagliata. 
Se la decisione è basata su un intervallo di confidenza al 95\%, la probabilità di sbagliare è 0.05. 
Se lo psicologo vuole che la probabilità d'errore sia più piccola, può costruire un intervallo di confidenza utilizzando un valore $\alpha$ minore. 
La diminuzione di $\alpha$, però, produce un aumento dell'ampiezza dell'intervallo di confidenza. 
Valori accettabili per $\alpha$ sono 0.1 e 0.05.

\begin{exmp}
Charter (1996) discute l'effetto della variazione dell'attendibilità del test sull'ampiezza dell'intervallo di confidenza per il punteggio vero. 
Nell'esempio considera i punteggi del QI ($\mu$ = 100, $\sigma$ = 15) immaginando di variare il coefficiente di attendibilità del test tramite il quale il QI viene misurato.
I valori esaminati sono 0.55, 0.65, 0.75, 0.85 e 0.95. 
Consideriamo, ad esempio, il caso di un punteggio osservato pari a QI = 120 e poniamo che $\rho_{xx'}$ = 0.65.
In tali circostanze, la stima del punteggio vero è pari a
\begin{align}
\hat{T} &= \bar{X} + r_{XX'}  (X - \bar{X}) \notag\\
&= 100 + 0.65 (120 - 100)\notag\\
&= 113.\notag
\end{align}
L'errore standard della stima è uguale a
\begin{align}
\sigma_{\hat{T}} &= \sigma_{X} \sqrt{r_{XX'} (1 - r_{XX'})} \notag\\
&= 15 \sqrt{0.65 (1 - 0.65)}\notag\\
&= 7.15.\notag
\end{align}
L'intervallo di confidenza al 95\% per la stima del punteggio vero diventa pertanto uguale a
\begin{align}
113 \pm 1.96 \cdot 7.15 = [98.98, 127.02].\notag
\end{align}
Nella figura successiva sono riportati gli intervalli di confidenza per il valore vero a livelli diversi della scala del QI, i quali sono stati calcolati supponendo valori diversi di attendibilità del test.
\end{exmp}

\begin{figure}[h!]
\label{fig:charter_96}
\centering
\includegraphics[width=\textwidth]{charter96}
\caption{Intervalli di confidenza basati sulla stima dell'errore standard della stima  e centrati sul punteggio vero stimato, per diversi livelli di attendibilità del test (Charter, 1996).}
\end{figure}


Lo stesso risultato si ottiene con la funzione \texttt{CI.tscore()} del pacchetto \texttt{psychometric} che richiede i seguenti argomenti: il punteggio osservato, la media del campione, la deviazione standard del campione e l'attendibilità del test:
\begin{lstlisting}
> CI.tscore(120, 100, 15, 0.65)
    SE.Est      LCL T.Score      UCL
1 7.154544 98.97735     113 127.0226
\end{lstlisting}


\section{Cut-off}

Uno degli usi possibili degli intervalli di confidenza per il punteggio vero è quello di confrontare i limiti dell'intervallo di confidenza con un cut-off. Sono possibili tre alternative: il limite inferiore dell'intervallo di confidenza è maggiore del cut-off, il limite superiore dell'intervallo è minore del cut-off, oppure il valore del cut-off è contenuto all'interno dell'intervallo.
Nel primo caso, lo psicologo afferma, con un grado di certezza $1 -\alpha$, che il valore vero del rispondente è superiore al cut-off. Nel secondo caso, lo psicologo afferma, con un grado di certezza $1 -\alpha$, che il valore vero del rispondente è inferiore al cut-off. Nel terzo caso lo psicologo non può concludere né che il valore vero sia inferiore né che sia superiore al cut-off.



%
%%------------------------------------------------------------
%
%\begin{frame}
%\frametitle{Illustrazione}
%
%\begin{itemize}
%\item Si considerino i punteggi del QI, per cui $\bar{X}$ = 100 e $s_X$
%  = 15. 
%\item Sia l'attendibilità del test $\rho_{XX'}$ = .95.  
%\item Si calcoli l'intervallo di confidenza al 95\% per un punteggio
%  osservato uguale a  110.
%
%\begin{lstlisting} 
%
%> xm <- 100
%> sx <- 15
%> rho <- .95
%> x <- 110
%\end{lstlisting}
%
%\end{itemize}
%
%\end{frame}
%
%%------------------------------------------------------------
%
%\begin{frame}
%\frametitle{Illustrazione}
%
%\begin{itemize}
%\item La stima del punteggio vero
%
%$$
%\hat{T} = \bar{X} + r_{XX'}  (X - \bar{X})
%$$
%
%\begin{lstlisting} 
%> t.hat <- xm + rho * (x - xm)
%> t.hat
%[1] 109.5
%\end{lstlisting}
%è uguale a 109.5.
%
%\item Si noti che la stima del punteggio vero è un valore che si
%  allontana dal valore osservato nella direzione della
%  media del campione.
%\end{itemize}
%
%\end{frame}
%
%%------------------------------------------------------------
%
%\begin{frame}
%\frametitle{Illustrazione}
%
%\begin{itemize}
%\item L'errore standard della stima 
%
%$$
%\sigma_{\hat{T}} = \sigma_X \sqrt{\rho_{XX'} (1 -\rho_{XX'})}
%$$
%
%\begin{lstlisting} 
%> se.t <- sx * sqrt(rho * (1 - rho))
%> se.t
%[1] 3.269174
%\end{lstlisting}
%è uguale a 3.27.
%
%\end{itemize}
%
%\end{frame}
%
%%------------------------------------------------------------
%
%\begin{frame}
%\frametitle{Illustrazione}
%
%\begin{itemize}
%\item Per questo rispondente, l'intervallo di confidenza al 95\% per il punteggio vero 
%
%$$
%\hat{T} \pm z  \sigma_{\hat{T}} 
%$$
%
%\begin{lstlisting} 
%> t.hat + c(1, -1) * qnorm(.025, 0, 1) * se.t
%[1] 103.0925 115.9075
%\end{lstlisting}
%è uguale a [103.09, 115.91].
%
%\end{itemize}
%
%\end{frame}

%%------------------------------------------------------------
%
%\begin{frame}
%\frametitle{Interpretazione dell'intervallo di confidenza}
%
%\begin{itemize}
%\item Nel lungo periodo le realizzazioni campionarie (\textit{Limite
%    Inferiore}, \textit{Limite Superiore}) dell'intervallo di confidenza
%  contengono il parametro incognito T con frequenza 1 - $\alpha$,
%\item nel lungo periodo la frequenza di intervalli campionari
%  (\textit{Limite Inferiore}, \textit{Limite Superiore}) che non
%  contengono il parametro incognito T è $\alpha$.
%\end{itemize}
%
%\end{frame}

%------------------------------------------------------------

\begin{exmp}
Si considerino i punteggi del QI, per cui $\bar{X}$ = 100 e $s_X$ = 15.  Sia l'attendibilità del test $\rho_{XX'}$ = 0.95.  
 Supponiamo che il rispondente abbia un QI = 130.   
 Poniamo che il cut-off per ammettere il rispondente ad un corso avanzato sia
120.
 Ci sono tre alternative:
il valore vero del rispondente è sicuramente maggiore di 120;
 il valore vero del rispondente è sicuramente inferiore di 120;
 le evidenze disponibili ci lasciano in dubbio se il punteggio
  vero sia maggiore o minore di 120.
Svolgiamo i calcoli per trovare l'intervallo di confidenza al
  livello di certezza del 95\%:
\begin{lstlisting} 
xm <- 100
sx <- 15
rho <- .95
x <- 130
t.hat <- xm + rho * (x - xm)
t.hat
#> [1] 128.5
se.t <- sx * sqrt(rho * (1 - rho))
se.t
#> [1] 3.269174
t.hat + c(1, -1) * qnorm(.025, 0, 1) * se.t
#> [1] 122.0925 134.9075
\end{lstlisting}
Dato che il limite inferiore
  dell'intervallo di confidenza è maggiore del cut-off, lo psicologo
  conclude che il punteggio vero del rispondente è maggiore di 120.
 Quindi, raccomanda che il rispondente sia ammesso al corso avanzato.

Continuiamo con l'esempio precedente, ma supponiamo che
  l'attendibilità del test abbia un valore simile a
  quello che solitamente si ottiene empiricamente, ovvero
  0.80. 
\begin{lstlisting} 
xm <- 100
sx <- 15
rho <- .8
x <- 130
t.hat <- xm + rho * (x - xm)
t.hat
#> [1] 124
se.t <- sx * sqrt(rho * (1 - rho))
se.t
#> [1] 6
t.hat + c(1, -1) * qnorm(.025, 0, 1) * se.t
#> [1] 112.2402 135.7598
\end{lstlisting}
In questo secondo esempio, l'intervallo di confidenza al 95\% è [112.24, 135.76] e
  \textit{contiene il valore del cut-off}.  
Dunque, la decisione dello psicologo è che non vi sono evidenze sufficienti che il  vero valore del rispondente sia superiore al cut-off. Si noti come la diminuzione dell’attendibilità del test porta all'aumento delle dimensioni dell’intervallo di confidenza. 
\end{exmp}

%-------------------------------------------------------------------
\section{Procedure alternative}

Non vi è un unico modo per costruire gli intervalli di confidenza per il punteggio vero. Charter e Feldt (1991) descrivono altri quattro approcci possibili, oltre a quello discusso qui, per costruire gli intervalli di confidenza per il punteggio vero. L'approccio che abbiamo descritto è accettato da tutti gli autori; le procedure alternative descritte da Charter e Feldt (1991),  non sono invece accettate come valide da tutti gli autori. 

La più comune delle procedure alternative descritte da Charter e Feldt (1991), che rappresenta l'approccio tradizionale a questo problema, centra l'intervallo di confidenza sul punteggio osservato di un rispondente e utilizza l'\textit{errore standard della misurazione} per calcolare i limiti dell'intervallo di confidenza:
\[
X_j \pm z_{\frac{\alpha}{2}} \sigma_E,
\]
dove $\sigma_E = \sigma_X \sqrt{1 -\rho_{XX'}}$. Tale procedura è però stata criticata da diversi autori (es., Dudek, 1979).


%%-------------------------------------------------------------------
%\section{Interpretazioni assegnate all'intervallo di confidenza}
%
%L'approccio su cui ci siamo focalizzati consente la seguente interpretazione (e.g., Charter e Feldt, 1991):
%\begin{quote}
%il 95\% degli intervalli di confidenza costruiti in questo modo contengono il valore vero.
%\end{quote}
%La procedura alternativa che abbiamo menzionato brevemente, invece, richiede la seguente interpretazione:
%\begin{quote}
%il 95\% degli studenti con un punteggio osservato pari a $X$ hanno un punteggio vero compreso tra $L_1$ e $L_2$ (dove $L_1$ e $L_2$ sono il limite inferiore e il limite superiore dell'intervallo di confidenza).
%\end{quote}
%
%%This interval can be expected to contain a given student’s true score 95\% of the time when the intervals are constructed using the observeds cores that are the result of repeated independent testings of the student with the same or parallel tests. Or the intervals can be expected to cover 95\% of the students’ true scores when many students are tested with the same test and a confidence interval is calculated for each student.





%
%% ----------------------------------------------------------------
%\section{Esercizi per casa}
%% ----------------------------------------------------------------
%
%
%
%% ----------------------------------------------------------------
%\begin{frame}{Esercizio 1}
%
%Come si calcola la stima migliore del punteggio vero di un
%rispondente? Perché non è sensato utilizzare il
%punteggio osservato quale stima del punteggio vero? 
%
%\end{frame}
%
%% ----------------------------------------------------------------
%\begin{frame}{Esercizio 2}
%
%Avendo stimato il punteggio vero di un rispondente, come si
%calcola l'intervallo di confidenza per il punteggio vero, ad un
%dato livello di fiducia?  Come si interpreta?
%
%\end{frame}
%
%% ----------------------------------------------------------------
%\begin{frame}{Esercizio 3}
%
%Che differenza c'è tra l'errore standard della misura e l'errore
%standard della stima del punteggio vero? Come si calcolano questi due
%stimatori? 
%
%\end{frame}
%
%
%% ----------------------------------------------------------------
%\begin{frame}[fragile]
%\frametitle{Esercizio 4}
%
%Un rispondente ottiene un punteggio di 88 in un test di intelligenza.
%Per il campione di standardizzazione del test, il punteggio
%medio è 100, con una deviazione standard di 15.
%L'attendibilità del test è 0.72.  Si calcoli l'intervallo di
%confidenza al 95\% per il punteggio vero del rispondente. 
%
%
%% \begin{align}
%% \left[(1-r_{XX'}) \bar{X} + r_{XX'}X \right] &\pm z \times \left[s_X
%% \sqrt{r_{XX'}(1-r_{XX'})}\right]\notag 
%% \end{align}
%
%% \begin{lstlisting}
%% > ((1-.72)*100 + .72*88)+
%% +     c(-1,1)*qnorm(.975)*(15*sqrt(.72*(1-.72)))
%% [1]  78.15968 104.56032
%% \end{lstlisting}
%
%\end{frame}
%
%% ----------------------------------------------------------------
%\begin{frame}[fragile]
%\frametitle{Esercizio 5}
%
%Un test per la selezione del personale viene somministrato a 500
%candidati.  La media delle risposte corrette nel campione è 76\%, con
%una deviazione standard di 12\%. L'attendibilità del test è uguale a
%0.82.  Mario Rossi ha osservato un punteggio pari al 82\% di risposte
%corrette.  Il cut-off per la selezione dei candidati è il 80\% di
%risposte corrette. Dopo avere calcolato l'intervallo di confidenza per
%il valore vero, qual è la raccomandazione dello psicologo?
%
%\end{frame}
%
%
%% % ----------------------------------------------------------------
%% \begin{frame}[fragile]
%% \frametitle{Esercizio 14}
%
%
%% \begin{align}
%% \left[(1-r_{XX'}) \bar{X} + r_{XX'}X \right] &\pm z \times \left[s_X
%%   \sqrt{r_{XX'}(1-r_{XX'})}\right]\notag 
%% \end{align}
%
%% Considerando che circa il 68\% dei casi è contenuto nell'intervallo compreso tra $\pm 1$ deviazioni standard dalla media, l'intervallo di confidenza diventa:
%
%% \begin{lstlisting}
%% > ((1-.82)*76 + .82*64) +
%% +      c(-1,1)*1*(12*sqrt(.82*(1-.82)))
%% [1] 61.54975 70.77025
%% \end{lstlisting}
%
%% Mario Rossi ottiene il lavoro.
%
%% \end{frame}
%
