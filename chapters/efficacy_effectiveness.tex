% !TEX encoding = UTF-8 
% !TeX program = pdflatex
% !TeX spellcheck = en-US

\chapter{La valutazione dell'intervento psicologico }
\label{ch:effect_size}


\section{La valutazione dell'esito dei trattamenti psicologici}

%Lo scopo della valutazione neuropsicologica è di dimostrare la presenza o l'assenza di un declino cognitivo mediante misure oggettive del funzionamento cognitivo. 
La valutazione neuropsicologica dei pazienti che presentano disturbi cognitivi fornisce informazioni cruciali su cui basare la diagnosi e valutare se ci sono stati cambiamenti nelle condizioni di un individuo. 
%L'utilità dei test neuropsicologici per la diagnosi si basa sull'identificazione di un pattern di prestazioni deficitarie in diversi domini cognitivi, piuttosto che sul risultato di un particolare test o di un particolare dominio cognitivo a sé stante. 
%Per tale ragione, la valutazione neuropsicologica include tipicamente una serie di misure sul funzionamento intellettivo generale corrente e test di prestazione nei principali domini cognitivi: memoria, comprensione e produzione linguistica, funzione esecutiva, funzione visuo-percettiva e visuo-spaziale, attenzione e velocità di elaborazione. 
%Si noti inoltre che, anche se una valutazione neuropsicologica deve essere completa, essa deve anche essere sufficientemente agile, nell'interesse del benessere e della cooperazione del paziente, da un lato, e della praticità, dall'altro.
%%
%Ad esempio, mentre la ``demenza'' è definita come una sindrome cognitiva che influenza la memoria, il pensiero, il comportamento e la capacità di svolgere le attività quotidiane, la ricerca neuropsicologica ha portato all'identificazione di modelli differenziali e patognomonici di declino cognitivo in diverse condizioni.  
%In particolare nelle prime fasi della malattia, le sindromi di demenza con diverse eziologie sottostanti spesso influenzano selettivamente specifici sistemi funzionali e aree cognitive coerenti con la patologia in questione. 
%Un certo numero di sindromi da demenza hanno quindi quella che potrebbe essere considerata una specifica firma cognitiva. 
%La valutazione neuropsicologica è diretta a (i) stabilire se i disturbi del paziente sono più probabilmente  neurogenici che psicogeni; e (ii) caratterizzare lo stato cognitivo del  paziente al fine di determinare se il suo profilo cognitivo sia coerente con l'una o l'altra di tali modalità. 
%Questo capitolo affronterà come questa valutazione è fatta e includerà alcuni casi studio. Cerca anche di espandere il "psicologico" in "neuropsicologico" per affrontare, anche se brevemente, l'impatto di quei cambiamenti cognitivi sul senso di sè stessi dei pazienti e sulle loro relazioni con gli altri.
%Nel contesto della demenza, nonostante i progressi nella neuroimaging, la documentazione dei cambiamenti cognitivi sulla valutazione neuropsicologica non raramente precede i risultati positivi su altre misure investigative. 
%La valutazione del trattamento ha a che fare con 
%\begin{enumerate}[(a)]
%\item Confronto (controllo o altri trattamenti)
%\item Confrontabilità gruppi
%\item Confrontabilità rilevazioni (es. test prima-dopo)
%\item Rilevanza risultati (significatività clinica vs statistica)
%\item Adeguata descrizione dei trattamenti
%\end{enumerate}
Essa costituisce uno degli strumenti che possono essere usati per la valutazione gli effetti di un trattamento.

Ma come si misurano gli effetti di un trattamento psicologico, in generale, o neuropsicologico, in particolare?
Consideriamo il caso di un intervento psicologico.
La valutazione di un intervento psicologico è molto diversa rispetto alla valutazione sull'effetto dei farmaci, ad esempio, perché c'è un numero quasi infinito di fattori che possono essere esaminati per decidere se un determinato tipo di intervento psicologico ha portato ad un esito positivo, e semplicemente non c'è modo di controllarli tutti. 
I ricercatori non concordano neppure su cosa debba essere valutato quando si valutano gli effetti di un trattamento.  
Miglioramento nel benessere soggettivo? 
Diminuzione dei sintomi?
Cambiamento di personalità? 
Miglioramento nei rapporti interpersonali? 
Miglioramento nelle capacità lavorative?
Crescita personale, realizzazione e miglioramento della persona? 
Tutti i precedenti?

Molto spesso gli studi sugli effetti del trattamento considerano punteggi medi di misure di esito basate sui sintomi. 
Questo modo di quantificare l'esito del trattamento non tiene in considerazione il fatto che, in qualunque trattamento, si osserva un cambiamento in alcuni pazienti ma non in altri.
Ciò è ulteriormente complicato dai problemi della comorbidità.
È possibile infatti che un trattamento possa produrre un esito favorevole rispetto ad alcune dimensioni di un deficit ma, nel contempo, un esito negativo per altre.
Anche se la maggior parte degli studi sugli effetti di un trattamento psicologico esamina la riduzione dei sintomi, il fine ultimo del trattamento dovrebbe essere quello di un miglioramento del funzionamento dell'individuo in contesti sociali e del livello di benessere percepito. 
Ma non è chiaro come ciò possa essere misurato.


\section{Efficacia dell'intervento ed efficienza clinica}

Indipendentemente dal problema di cosa deve essere misurato, è anche necessario chiedersi quali siano le finalità della valutazione.
La letteratura psicologica distingue tra efficacia ed efficienza clinica del trattamento, laddove per efficacia (\emph{efficacy}) si intende la capacità del trattamento di produrre un cambiamento, mentre l'efficienza clinica (\emph{effectiveness}) può essere definita ``la capacità di un intervento di produrre gli effetti benefici desiderati nella pratica clinica corrente'' e fa anche riferimento al rapporto costi/benefici dell'intervento.
Come descritto da Hunsley (2007)
\begin{quote}
Treatment efficacy studies involve methodological efforts to maximize the internal validity of a study. 
This commonly includes the use of design features, such as random assignment to treatment and control conditions, training of therapists to a specified level of competence in providing the treatment, and ensuring that all participants have the condition that the treatment was designed to address. 
Treatment effectiveness studies, on the other hand, strive to maximize external validity while maintaining an adequate level of internal validity (without which, of course, no viable conclusions could be drawn about the impact of the treatment). 
Most commonly, efforts to enhance external validity involve locating the treatment study within clinical service sites that provide ongoing health services, thus using clinicians who are routinely providing psychological services and patients who have been referred to the clinical settings (p. 117).
\end{quote}

I dati degli studi sull'\emph{efficacy} e sull'\emph{effectiveness} sono fondamentali per una piena comprensione dell'impatto potenziale di un trattamento. 
Una volta che un trattamento abbia dimostrato la sua \emph{efficacy} attraverso la riproducibilità dei dati di un esperimento, il passo successivo è determinare l'impatto effettivo del trattamento nella pratica clinica.
Deve essere infatti dimostrato che i trattamenti esaminati nelle condizioni controllate degli studi di ricerca (studi sull'\emph{efficacy}) possano anche avere un adeguato impatto clinico quando vengono utilizzati nell'effettiva pratica clinica (studi sull'\emph{effectiveness}). 

Nel concetto di \emph{effectiveness} si possono individuare quattro componenti: efficacia del trattamento, tollerabilità e sicurezza del trattamento, funzionamento del paziente, accettabilità dell'intervento da parte del paziente. 
Tutte le componenti interagiscono tra di loro e forniscono informazioni sull'esito di un determinato intervento.
Per quanto riguarda l'efficacia del trattamento, le variabili più importanti sono la scomparsa dei sintomi, il tasso di ricadute, la comorbidità. 
Gli elementi più importanti che incidono sulla tollerabilità e sicurezza del trattamento sono gli effetti collaterali, la sicurezza e la facilità di assunzione della terapia. 
Il funzionamento del paziente, il terzo dominio, include le normali attività quotidiane e la qualità della vita, elementi che possono essere quantificati e valutati in studi clinici randomizzati o in studi osservazionali. 
L'accettabilità di un intervento da parte del paziente infine, è il dominio della compliance, ed è strettamente legato alle altre tre componenti.

Sebbene ciascun dominio possa essere quantificabile singolarmente, è anche possibile misurare l'\emph{effectiveness} in maniera globale. 
Uno strumento che può essere usato a tale scopo è la scala \emph{Clinical Global Impression} (CGI), che rappresenta il modo più semplice per valutare e quantificare, in generale, l'\emph{effectiveness} di un intervento. 
Inoltre, la scala della \emph{Valutazione Globale del Funzionamento} (GAF) fornisce un punteggio singolo, composito della funzione psicologica, sociale ed occupazionale su un ipotetico continuum che va dal benessere mentale al funzionamento più deficitario causato dalla malattia. 
%Le scale GAF e CGI, comunque, non esaminano la progressione dei singoli domini. 
%Per esempio, i sintomi di malattia possono progredire differentemente rispetto alla capacità di vivere in maniera indipendente.


%\section{La dimensione dell'effetto}
%
%I risultati degli studi di ricerca sull'efficacia e l'efficienza clinica degli interventi psicologici usano una metrica comune chiamata \emph{dimensione dell'effetto}.
%La dimensione dell'effetto può essere calcolata per quasi tutti i tipi di progetti di ricerca e in tutte le analisi statistiche. 
%Le analisi basate sui confronti tra gruppi danno luogo a due tipi di indici che misurano la dimensione dell'effetto. 
%Il primo indice riguarda le differenze tra le medie dei gruppi, laddove la dimensione dell'effetto è data dalla differenza tra le medie dei gruppi (per esempio gruppi con il trattamento e senza trattamento) divisa per la stima della deviazione standard raggruppata. 
%Tali indici vanno sotto il nome di $d$, $g$ o $\eta^2$ -- si veda la~XX. 
%Il secondo tipo di dimensione dell'effetto comporta un confronto di gruppi in termini degli odds o della probabilità di un risultato. 
%Un \emph{odds ratio} (OR) viene calcolato per determinare l'associazione tra la condizione di gruppo (ad es., trattamento e non trattamento) e una variabile ad esito binario (ad esempio, la presenza o la non insorgenza di un evento, come la ricaduta). 
%Un \emph{rischio relativo} (RR) viene calcolato per confrontare la probabilità che un evento si verifichi in base ai due livelli del fattore di rischio considerato (ad es., trattamento/non trattamento, esposizione/non esposizione ad un fattore di rischio).
%Sebbene comunemente usati nella ricerca epidemiologica, questi due indici della dimensione dell'effetto sono meno frequentemente utilizzati nella ricerca psicologica. 
%
%Si noti inoltre un punto importante: non è possibile confrontare direttamente la dimensione dell'effetto ottenuta in studi sull'efficacia del trattamento e in studi sull'efficenza clinica perché, quasi sempre, tali studi sono stati svolti mediante disegni sperimentali diversi.
%Gli studi di efficacia sono generalmente studi randomizzati controllati nei quali i risultati del trattamento vengono confrontati con i risultati di una condizione di controllo (ad esempio, nessun trattamento o una forma alternativa di trattamento). 
%In questo caso, la dimensione dell'effetto si basa sul cambiamento che può essere attribuito all'effetto causale del trattamento. 
%Pochi studi sull'efficacia clinica sono invece studi randomizzati controllati: gli studi sull'efficacia clinica coinvolgono tipicamente un'indagine entro i gruppi (cioè, un confronto tra la condizione pre-trattamento e la condizione post-trattamento, senza alcuna condizione di controllo). 
%Di conseguenza, la dimensione dell'effetto ottenuta in questo tipo di indagini è basata su un cambiamento dovuto a cause molteplici: oltre agli effetti del trattamento, ci sono agli effetti dovuti alla maturazione, alla regressione verso la media, alla remissione spontanea dei sintomi dovuta al passaggio di tempo e la reattività delle misure. 
%Pertanto, è probabile che la gli studi sull'efficacia del trattamento portino ad una stima maggiore della dimensione dell'effetto rispetto ai valori che vengono ottenuti negli studi sull'efficienza clinica.
%





